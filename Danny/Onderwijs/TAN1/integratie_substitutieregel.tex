\documentclass[a4paper,12pt]{article}
\usepackage{amsmath}
\usepackage{geometry}
\usepackage[dutch]{babel}
\geometry{margin=2.5cm}
\title{Uitwerkingen voor ``Integratietechnieken: Substitutieregel''}
\author{Dr. ir. D.A.M.P. Blom}
\date{\today}

\begin{document}

\maketitle

\section*{$\int x^4 \sin(x^5 + 3)\, dx$}

We gebruiken substitutie op basis van de binnenfunctie die in de sinus staat:
\[
    u = x^5 + 3 \quad \Rightarrow \quad du = 5x^4\, dx \quad \Rightarrow \quad \frac{1}{5}du = x^4\, dx
\]
Substitueren in de integraal geeft ons:
\begin{align*}
    \int x^4 \sin(x^5 + 3)\, dx &= \int \frac{1}{5} \sin(u)\, du \\
                                &= -\frac{1}{5}\cos(u) + C \\
                                &\boxed{=-\frac{1}{5} \cos(x^5 + 3) + C}
\end{align*}

\section*{$\int \frac{x}{1 + x^2}\, dx$}

We gebruiken substitutie op basis van de noemer (merk op dat de afgeleide van $1+x^2$ gelijk is aan $2x$, oftewel 2 keer de teller):
\[
u = 1 + x^2 \quad \Rightarrow \quad du = 2x\, dx \quad \Rightarrow \quad \frac{1}{2}du = x\, dx
\]
Substitueren in de integraal geeft ons:
\begin{align*}
    \int \frac{x}{1 + x^2}\, dx &= \int \frac{x\, dx}{1 + x^2} \\
                                &= \int \frac{\frac{1}{2}\,du}{u} \\
                                &= \frac{1}{2} \int \frac{1}{u}\, du \\
                                &= \frac{1}{2} \ln|u| + C \\
                                &\boxed{=\frac{1}{2} \ln|1 + x^2| + C}
\end{align*}

\section*{3. $\int \frac{1}{x \ln(x)}\, dx$}

Deze integraal is al iets lastiger. Merk echter op dat de afgeleide van $\ln(x)$ gelijk is aan $\frac{1}{x}$.
Gebruik nu de volgende substitutie:
\[
    u = \ln(x) \quad \Rightarrow \quad du = \frac{1}{x}\, dx
\]

Substitueren in de integraal geeft ons:
\begin{align*}
    \int \frac{1}{x \ln(x)}\, dx    &= \int \frac{1}{\ln(x)} \cdot \frac{1}{x} \, dx \\
                                    &= \int \frac{1}{u}\, du \\
                                    &= \ln|u| + C \\
                                    &\boxed{= \ln|\ln(x)| + C}
\end{align*}

\section*{4. $\int x \sqrt{1 + x^2}\, dx$}

We substitueren opnieuw $u = 1 + x^2$, omdat de afgeleide van $1 + x^2$ gelijk is aan $2x$, oftewel 2 keer de $x$ die buiten de wortel staat:
\[
    u = 1 + x^2 \quad \Rightarrow \quad du = 2x\, dx \quad \Rightarrow \quad \frac{1}{2}du = x\, dx
\]

Substitueren in de integraal geeft ons:
\begin{align*}
    \int x \sqrt{1 + x^2}\, dx  &= \int \frac{1}{2} \sqrt{u}\, du\\
                                &= \int \frac{1}{2} u^{\frac{1}{2}}\, du \\
                                &= \frac{1}{2} \cdot \frac{2}{3} u^{3/2} + C \\
                                &= \frac{1}{3} u \sqrt{u} + C \\
                                &\boxed{= \frac{1}{3} (1 + x^2) \sqrt{1 + x^2} + C}
\end{align*}

\section*{5. $\int x^3 \sqrt{1 + x^2}\, dx$}

Dit is de lastigste integraal, omdat de $x^3$ buiten de wortel niet lijkt op de afgeleide van $1 + x^2$.
De truc in dit geval is om $x^3 \sqrt{1 + x^2}$ te lezen als $x \cdot x^2 \cdot \sqrt{1+x^2}$.
In dit geval is $x$ te relateren aan $\frac{1}{2}$ keer de afgeleide van $1 + x^2$, en $x^2$ is gelijk aan $(1+x^2) - 1$.

We gebruiken de substitutie
\[
    u = 1 + x^2 \quad \Rightarrow \quad du = 2x\, dx \quad \Rightarrow \quad \frac{1}{2}du = x\, dx
\]

Aangezien geldt dat $u = 1 + x^2$, kunnen we $x^2$ herschrijven als $x^2 = u - 1$.
Dit geeft ons:

\begin{align*}
    \int x^3 \sqrt{1 + x^2}\, dx    &= \int x \cdot x^2 \sqrt{1 + x^2}\, dx \\
                                    &= \int \frac{1}{2}(u - 1)\sqrt{u}\, du \\
                                    &= \int (\frac{1}{2}u\sqrt{u} - \frac{1}{2}\sqrt{u})\, du \\
                                    &= \int (\frac{1}{2}u^{3/2} - \frac{1}{2}u^{1/2})\, du \\
                                    &= \frac{1}{2}\cdot \frac{2}{5} u^{5/2} - \frac{1}{2} \cdot \frac{2}{3} u^{3/2} + C \\
                                    &= \frac{1}{5} u^2\sqrt{u} - \frac{1}{3} u\sqrt{u} + C \\
                                    &\boxed{= \frac{1}{5} (1 + x^2)^2 \sqrt{1+x^2} - \frac{1}{3} (1 + x^2)\sqrt{1+x^2} + C}
\end{align*}

\end{document}