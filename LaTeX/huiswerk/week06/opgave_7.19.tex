\question{7.19}{
    Een satelliet heeft vijf zonnepanelen.
    Ieder zonnepaneel heeft een levensduur die kan worden beschreven door een negatief-exponenti\"ele verdeling met $\lambda=0,5$.
    Dus de gemiddelde levensduur van een paneel is $\frac{1}{0,50} = 2$ jaar.
    Zodra drie zonnepanelen zijn uitgevallen, houdt de satelliet op te functioneren.   
}
\begin{enumerate}[label=(\alph*)]
    \item Hoe groot is de kans dat een willekeurig zonnepaneel na vier jaar nog functioneert?
    \answer{

    }

    \item Hoe groot is de kans dat de satelliet na vier jaar nog functioneert?
    We gaan ervan uit dat de levensduren van de panelen onderling onafhankelijk zijn.
    \answer{
        
    }
\end{enumerate}
