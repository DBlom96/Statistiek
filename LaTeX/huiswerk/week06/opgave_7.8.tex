\question{7.8}{
    Op de luchthaven van Port Elizabeth is slechts \'e\'en security-doorgang.
    Het aantal passagiers dat zich per minuut meldt bij de security kan worden beschreven door een Poissonverdeling met $\mu=3$.
}
\begin{enumerate}[label=(\alph*)]
    \item Hoe groot is de kans dat in een bepaalde minuut geen passagiers arriveren bij het security-punt?
    \answer{
        Laat $X$ de kansvariabele zijn die het aantal passagiers telt dat zich per minuut bij de security meldt.
        Gegeven is dat $X \sim \text{Poisson}(\lambda=3)$.
        De kans dat in een bepaalde minuut geen passagiers arriveren bij het security-punt is gelijk aan
        \[
            P(X=0) = e^{-\lambda}\cdot \frac{\lambda^{0}}{0!} = e^{-3}\cdot \frac{3^{0}}{0!} = e^{-3} \approx 0,0498
        \]
    }

    \item Hoe groot is de kans dat er meer dan $5$ passagiers komen in een willekeurige minuut?
    \answer{
        De kans dat er meer dan $5$ passagiers komen in een willekeurige minuut is gelijk aan
        \[
            P(X > 5) = 1 - P(X \le 5) = 1 - \poissoncdf(\lambda=3; k=5) \approx 0,0839.
        \]
    }
    
    \item Hoe groot is de kans dat er meer dan $200$ passagiers komen in een willekeurig uur?
    \answer{
        Laat nu $X$ de kansvariabele zijn die het aantal passagiers telt dat in een willekeurig uur aankomt bij de security-doorgang.
        Nu geldt dat de waarde van de Poissonparameter gelijk is aan $\lambda = 3 \cdot 60 = 180$.
        De kans op meer dan $200$ passagiers in een willekeurig uur is dan gelijk aan
        \[
            P(X > 200) = 1 - P(X \le 199) = 1 - \poissoncdf(\lambda=180; k=199) \approx 0,0749.
        \]
    }
\end{enumerate}
