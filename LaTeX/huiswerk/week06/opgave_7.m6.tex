\question{7.m6}{
    De tijd (in minuten) die verstrijkt tussen de binnenkomst van twee opeenvolgende klanten bij een postkantoor, kan worden beschouwd als een kansvariabele $T$ die een negatief exponenti\"ele verdeling heeft met parameters $\lambda=4$.
    De standaarddeviatie van de variabele $T$ is dan gelijk aan $\ldots$
}
\begin{enumerate}[label=(\alph*)]
    \item $4$ minuten.
    \item $0,25$ minuten.
    \item $2$ minuten.
    \item $0,50$ minuten.
\end{enumerate}
\answer{
    De standaarddeviatie van een exponentieel verdeelde kansvariabele $T$ met parameter $\lambda$ is gelijk aan $\sqrt{\frac{1}{\lambda^2}}=\frac{1}{\lambda}$.
    Als $\lambda=4$, is de standaarddeviatie dus gelijk aan $\frac{1}{4} = 0,25$.
    Het juiste antwoord is dus (b).
}