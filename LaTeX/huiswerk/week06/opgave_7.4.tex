\question{7.4}{
    Voor een eerstehulpafdeling van een ziekenhuis geldt dat het aantal pati\"enten met brandwonden dat per dag binnenkomt, kan worden beschreven door een Poissonverdeling met $\mu=0,35$.
    Bereken met de formule van de Poissonverdeling de kans dat het aantal van dergelijke pati\"enten per dag $\ldots$
}
\begin{enumerate}[label=(\alph*)]
    \item precies nul bedraagt.
    \answer{
        Laat $X$ de kansvariabele zijn die het aantal pati\"enten telt dat op een willekeurige dag met brandwonden binnenkomt op de eerstehulpafdeling van het ziekenhuis.
        Gegeven is dat $X \sim \text{Poisson}(\lambda=0,35)$.
        De Poissonformule geeft voor elke mogelijke uitkomst $k = 0, 1, 2, \ldots$ de kans op die uitkomst aan als
        \[
            P(X=k) = e^{-\lambda} \cdot \frac{\lambda^{k}}{k!}.
        \]
        De kans dat het aantal van dergelijke pati\"enten per dag precies nul bedraagt is dus gelijk aan
        \[
            P(X=0) = e^{-0,35} \cdot \frac{(0,35)^0}{0!} \approx 0,7047.
        \]
    }

    \item precies \'e\'en is.
    \answer{
        De kans dat het aantal van dergelijke pati\"enten per dag precies \'e\'en bedraagt is dus gelijk aan
        \[
            P(X=1) = e^{-0,35} \cdot \frac{(0,35)^1}{1!} \approx 0,2466.
        \]
    }
    
    \item meer dan \'e\'en bedraagt.
    \answer{
        De kans dat het aantal van dergelijke pati\"enten per dag meer dan \'e\'en bedraagt kan met de complementregel berekend worden:
        \begin{align*}
            P(X>1)  &= 1 - P(X \le 1) \\
                    &= 1 - P(X=0) - P(X=1) \\
                    &\approx 1 - 0,7047 - 0,2466 \\
                    &\approx 0,0487.
        \end{align*}
    }
\end{enumerate}
