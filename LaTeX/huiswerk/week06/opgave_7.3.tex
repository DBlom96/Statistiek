\question{7.3}{
    Het aantal lekke banden dat bij een garage dagelijks ter reparatie wordt aangeboden, mag worden beschouwd als een kansvariabele $K$ met een Poissonverdeling met $\mu=4$ per dag.
    Bepaal met de tabel van de Poissonverdeling voor een willekeurige dag de kans op het ter reparatie aanbieden van:
}
\begin{enumerate}[label=(\alph*)]
    \item precies \'e\'en lekke band.
    \answer{
        We hebben aangenomen dat het aantal lekke banden dat op een dag ter reparatie wordt aangeboden bij een garage een Poisson-verdeelde kansvariabele $K$ is met parameter $\mu=4$ per dag.
        De kans op precies \'e\'en lekke band is dan gelijk aan
        \[
            P(K=1) = \poissonpdf(\lambda=4; k=1) \approx 0,0733.
        \]
    }

    \item precies zes lekke banden.
    \answer{
        De kans op precies zes lekke banden is dan gelijk aan
        \[
            P(K=6) = \poissonpdf(\lambda=4; k=6) \approx 0,1042.
        \]
    }
    
    \item meer dan drie lekke banden.
    \answer{
        De kans op meer dan drie lekke banden is gelijk aan
        \[
            P(K > 3) = 1 - P(K \le 3) = 1 - \poissoncdf(\lambda=4; k=3) \approx 0,5665.
        \]
    }

    \item minder dan drie lekke banden.
    \answer{
        De kans op minder dan drie lekke banden is gelijk aan
        \[
            P(K < 3) = P(K \le 2) = \poissoncdf(\lambda=4; k=2) \approx 0,2381.
        \]
    }

    \item meer dan vier, maar minder dan tien lekke banden.
    \answer{
        De kans op meer dan vier, maar minder dan tien lekke banden is gelijk aan
        \begin{align*}
            P(4 < K < 10)   &= P(5 \le K \le 9) \\
                            &= P(K \le 9) - P(K \le 4) \\
                            &= \poissoncdf(\lambda=4; k=9) - \poissoncdf(\lambda=4; k=4) \\
                            &\approx 0,3630.
        \end{align*}
    }
\end{enumerate}
