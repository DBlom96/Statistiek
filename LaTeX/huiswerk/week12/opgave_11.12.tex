\question{11.12}{
    Een grootwinkelbedrijf verkoopt in alle filialen een bepaald afwasmiddel.
    In een week werd in zes filialen het volgende aantal flessen van dit afwasmiddel verkocht.
    \begin{center}
        \begin{tabular}{ccccccc}
            \toprule
                {\bfseries Filiaal} & {\bfseries A} & {\bfseries B} & {\bfseries C} & {\bfseries D} & {\bfseries E} & {\bfseries F} \\
            \cmidrule{1-1} \cmidrule{2-2} \cmidrule{3-3} \cmidrule{4-4} \cmidrule{5-5} \cmidrule{6-6} \cmidrule{7-7} 
                {\bfseries Aantal flessen} & $320$ & $260$ & $440$ & $400$ & $380$ & $300$ \\
            \bottomrule
        \end{tabular}
    \end{center}
    In de week nadat voorgaande gegevens werden verzameld, is een intensieve reclamecampagne gehouden voor het desbetreffende afwasmiddel.
    Volgend op deze reclamecampagne worden in tien filialen in \'e\'en week de volgende aantallen flessen verkocht:
    \begin{center}
        \begin{tabular}{ccccccccccc}
            \toprule
                {\bfseries Filiaal} & {\bfseries P} & {\bfseries Q} & {\bfseries R} & {\bfseries S} & {\bfseries T} & {\bfseries U} & {\bfseries V} &{\bfseries W} & {\bfseries X} & {\bfseries Y} \\
            \cmidrule{1-1} \cmidrule{2-2} \cmidrule{3-3} \cmidrule{4-4} \cmidrule{5-5} \cmidrule{6-6} \cmidrule{7-7} \cmidrule{8-8} \cmidrule{9-9} \cmidrule{10-10} \cmidrule{11-11}
                {\bfseries Aantal flessen} & $400$ & $360$ & $410$ & $500$ & $540$ & $380$ & $490$ & $520$ & $420$ & $480$ \\
            \bottomrule
        \end{tabular}
    \end{center}
    Toets of het gemiddelde aantal verkochte flessen voor en na de behandeling gelijk is ($\alpha=0,05$).
    (Ga ervan uit dat de waargenomen aantallen mogen worden beschouwd als trekkingen uit normale verdelingen.)
}
\answer{
    Laat $X$ en $Y$ de kansvariabelen zijn die het aantal verkochte flessen afwasmiddel telt respectievelijk v\'o\'or en na de intensieve reclamecampagne.
    We mogen ervan uitgaan dat deze aantallen trekkingen zijn uit normale verdelingen, oftewel
    \begin{align*}
        X &\sim N(\mu_X = ?; \sigma_X = ?) \\
        Y &\sim N(\mu_Y = ?; \sigma_Y = ?). 
    \end{align*}
    Merk op dat we niet weten of de varianties gelijk zijn, dus hiervoor moeten we eerst een $F$-toets uitvoeren.

    {\bfseries \boldmath$F$-toets}\\
    We gaan toetsen of de varianties van $X$ en $Y$ gelijk zijn.
    Hiervoor defini\"eren we allereerst de nulhypothese $H_0$ en alternatieve hypothese $H_1$:
    \[
        H_0: \sigma_X^2 = \sigma_Y^2 \text{ versus } H_1: \sigma_X^2 \neq \sigma_Y^2.
    \]
    We gaan opnieuw uit van een significantieniveau $\alpha = 0,05$ en de steekproefgegevens in de bovenstaande tabellen.
    We berekenen allereerst de steekproefgemiddeldes:
    \begin{align*}
        \overline{x} &= \frac{x_1+x_2+\ldots+x_n}{n} = \frac{320 + 260 + \ldots + 300}{6} = 350 \\
        \overline{y} &= \frac{y_1+y_2+\ldots+y_m}{m} = \frac{400 + 360 + \ldots + 480}{10} = 450 \\
    \end{align*}
    Met behulp van de steekproefgemiddeldes kunnen we nu de steekproefvarianties bepalen.
    \begin{align*}
        s_X^2 &= \frac{(x_1-\overline{x})^2 + \ldots + (x_n-\overline{x})^2}{n-1} = \frac{(320-350)^2 + \ldots + (300-350)^2}{5} = 4600 \\
        s_Y^2 &= \frac{(y_1-\overline{y})^2 + \ldots + (y_m-\overline{y})^2}{m-1} = \frac{(400-450)^2 + \ldots + (480-450)^2}{9} = 4000 \\
    \end{align*}
    
    De toetsingsgrootheid van de $F$-toets is gelijk aan
    \[
        F = \frac{S_X^2}{S_Y^2}
    \]
    en volgt de $F$-verdeling met respectievelijk $\text{df1}=n-1=5$ en $\text{df2}=m-1=9$ vrijheidsgraden.
    In dit geval hebben we een toetsingsgrootheid van
    \[
        f = \frac{s_X^2}{s_Y^2} = \frac{4600}{4000} = 1,15
    \]

    {\bfseries Methode 1 (kritiek gebied):} \\
    Omdat we tweezijdig toetsen, is het kritieke gebied van de vorm $(-\infty; g_1]$ en $[g_2; \infty)$, waarbij de grenzen $g_1$ en $g_2$ bepaald kunnen worden met de $F(5,9)$-verdeling.
    \begin{align*}
        &\fcdf(\text{lower}=0; \text{upper}=g_1; \text{df1}=5; \text{df2}=9)=\alpha/2=0,025 \rightarrow g_1 \approx 0,1497\\
        &\fcdf(\text{lower}=g_2; \text{upper}=10^{99}; \text{df1}=5; \text{df2}=9)=\alpha/2=0,025 \rightarrow g_2 \approx 4,4844
    \end{align*}
    De berekende $f = 1,15$ ligt dus niet in het kritieke gebied, dus we accepteren de nulhypothese $H_0$.
    Er is op basis van deze steekproeven onvoldoende reden om aan te nemen dat de varianties niet gelijk zouden zijn aan elkaar.
    
    {\bfseries Methode 2 ($p$-waarde):} \\
    De berekende $f = 1,15$ ligt boven de $1$, dus we berekenen de $p$-waarde als de rechteroverschrijdingskans van de waarde $1,15$ en vergelijken deze met $\alpha/2 = 0,025$.
    \begin{align*}
        p &= \normalcdf(\text{lower}=1,15; \text{upper}=10^{99}; \text{df1}=5; \text{df2}=9) \approx 0,4020.
    \end{align*}
    De $p$-waarde is groter dan het significantieniveau $\alpha$, dus we accepteren de nulhypothese $H_0$.
    Er is op basis van deze steekproeven onvoldoende reden om aan te nemen dat de varianties niet gelijk zouden zijn aan elkaar.

    {\bfseries Onafhankelijke \boldmath$t$-toets (gelijke variantie)}\\
    Door het toetsresultaat van de $F$-toets mogen we uitgaande van gelijke varianties, oftewel $\sigma^2 = \sigma_X^2 = \sigma_Y^2$.
    We gaan toetsen of de gemiddeldes van $X$ en $Y$ gelijk zijn.
    Hiervoor defini\"eren we allereerst de nulhypothese $H_0$ en alternatieve hypothese $H_1$:
    \[
        H_0: \mu_X = \mu_Y \text{ versus } H_1: \mu_X = \mu_Y.
    \]
    Gegeven is het significantieniveau $\alpha = 0,05$ en de steekproefgegevens hierboven berekend:
    \begin{align*}
        \overline{x} = 350 \quad s_X^2 = 4600 \quad n = 6 \\
        \overline{y} = 450 \quad s_Y^2 = 4000 \quad m = 10 \\
    \end{align*}

    Als toetsingsgrootheid kijken we naar de verschilvariabele $V = \overline{X} - \overline{Y}$.
    Deze verschilvariabele heeft een verwachtingswaarde $\mu_V = \mu_X - \mu_Y$.
    Omdat we mogen uitgaan van gelijke varianties, kunnen we de standaardafwijking $\sigma_V$ van $V$ schatten met behulp van de ``pooled variance'':
    \begin{align*}
        s_P^2 = \frac{(n-1)s_X^2+(m-1)s_Y^2}{n-1+m-1} = \frac{6\cdot 4600 + 10 \cdot 4000}{14} \approx 4828,5714
    \end{align*}
    Een schatting van $\sigma_V$ berekenen we dan als volgt:
    \[
        s_V = \sqrt{\frac{s_P^2}{n} + \frac{s_P^2}{m}} = \sqrt{\frac{4828,5714}{6} + \frac{4828,5714}{10}}\approx 35,8834.
    \]

    Het geobserveerde verschil is $v = \overline{x} - \overline{y} = 350 - 450 = -100$.
    Onder de nulhypothese geldt $\mu_1 = \mu_2$, oftewel de verschilvariabele heeft verwachtingswaarde $\mu_V = \mu_1 - \mu_2 = 0$.

    De bijbehorende $t$-waarde berekenen we dan als volgt:
    \[
        t = \frac{v - \mu_V}{s_V} = \frac{-100 - 0}{35,8834} \approx -2,7868.
    \]

    {\bfseries Methode 1 (kritiek gebied):}
    
    Omdat we tweezijdig toetsen, is het kritieke gebied van de vorm $(-\infty; g_1]$ en $[g_2; \infty)$, waarbij de grenzen $g_1$ en $g_2$ bepaald kunnen worden met de $t$-verdeling.
    Omdat we met de pooled variance werken, is het aantal vrijheidsgraden gelijk aan $\text{df}=n+m-2=14$.
    \begin{align*}
        g_1 &= \invt(\text{opp}=\alpha/2 = 0,025; \text{df}=14) \approx-2,1448 \\
        g_2 &= \invt(\text{opp}=1-\alpha/2 = 0,975; \text{df}=14) \approx 2,1448
    \end{align*}
    De berekende $t \approx -2,7868$ ligt dus in het kritieke gebied, dus we verwerpen de nulhypothese $H_0$.
    Er is op basis van deze steekproeven voldoende reden om aan te nemen dat het aantal verkochte flessen afwasmiddel significant hoger is na de intensieve reclamecampagne.

    {\bfseries Methode 2 ($p$-waarde):}

    De berekende $t \approx -2,7868$ ligt links van het gemiddelde $\mu_V = 0$.
    Omdat we tweezijdig toetsen, berekenen we de $p$-waarde als de linkeroverschrijdingskans van de waarde $-2,7868$ en vergelijken deze met $\alpha/2 = 0,025$.
    \begin{align*}
        p &= \tcdf(\text{lower}=-10^{99}; \text{upper}=-2,7868; \text{df}=14) \approx 0,0073.
    \end{align*}
    De $p$-waarde is kleiner dan $\alpha/2$, dus we verwerpen de nulhypothese $H_0$.
    Er is op basis van deze steekproeven voldoende reden om aan te nemen dat het aantal verkochte flessen afwasmiddel significant hoger is na de intensieve reclamecampagne.
}