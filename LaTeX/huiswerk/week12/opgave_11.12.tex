\question{11.12}{
    Een grootwinkelbedrijf verkoopt in alle filialen een bepaald afwasmiddel.
    In een week werd in zes filialen het volgende aantal flessen van dit afwasmiddel verkocht.
    \begin{center}
        \begin{tabular}{ccccccc}
            \toprule
                {\bfseries Filiaal} & {\bfseries A} & {\bfseries B} & {\bfseries C} & {\bfseries D} & {\bfseries E} & {\bfseries F} \\
            \cmidrule{1-1} \cmidrule{2-2} \cmidrule{3-3} \cmidrule{4-4} \cmidrule{5-5} \cmidrule{6-6} \cmidrule{7-7} 
                {\bfseries Aantal flessen} & $320$ & $260$ & $440$ & $400$ & $380$ & $300$ \\
            \bottomrule
        \end{tabular}
    \end{center}
    In de week nadat voorgaande gegevens werden verzameld, is een intensieve reclamecampagne gehouden voor het desbetreffende afwasmiddel.
    Volgend op deze reclamecampagne worden in tien filialen in \'e\'en week de volgende aantallen flessen verkocht:
    \begin{center}
        \begin{tabular}{ccccccccccc}
            \toprule
                {\bfseries Filiaal} & {\bfseries P} & {\bfseries Q} & {\bfseries R} & {\bfseries S} & {\bfseries T} & {\bfseries U} & {\bfseries V} &{\bfseries W} & {\bfseries X} & {\bfseries Y} \\
            \cmidrule{1-1} \cmidrule{2-2} \cmidrule{3-3} \cmidrule{4-4} \cmidrule{5-5} \cmidrule{6-6} \cmidrule{7-7} \cmidrule{8-8} \cmidrule{9-9} \cmidrule{10-10} \cmidrule{11-11}
                {\bfseries Aantal flessen} & $400$ & $3600$ & $410$ & $500$ & $540$ & $380$ & $490$ & $520$ & $420$ & $480$ \\
            \bottomrule
        \end{tabular}
    \end{center}
    Toets of het gemiddelde aantal verkochte flessen voor en na de behandeling gelijk is ($\alpha=0,05$).
    (Ga ervan uit dat de waargenomen aantallen mogen worden beschouwd als trekkingen uit normale verdelingen.)
}
\answer{
    
}