\question{11.13}{
    In zes filialen van een supermarktketen is het aantal verkochte flessen van een wasmiddel vlak voor en vlak na een reclamecampagne geregistreerd.
    De resultaten waren:
    \begin{center}
        \begin{tabular}{ccccccc}
            \toprule
                {\bfseries Filiaal} & {\bfseries A} & {\bfseries B} & {\bfseries C} & {\bfseries D} & {\bfseries E} & {\bfseries F} \\
            \cmidrule{1-1} \cmidrule{2-2} \cmidrule{3-3} \cmidrule{4-4} \cmidrule{5-5} \cmidrule{6-6} \cmidrule{7-7} 
                Aantal voor & $320$ & $260$ & $440$ & $400$ & $380$ & $310$ \\
                Aantal na & $420$ & $380$ & $520$ & $490$ & $490$ & $400$ \\
            \bottomrule
        \end{tabular}
    \end{center}
    Toets of er een opvallend verschil is in de omzet per filiaal voor en na de campagne ($\alpha=0,05$).
    Voer de toets op twee manieren uit en geef aan van welke veronderstellingen men is uitgegaan bij de toetsen (vergelijk opgave 11.12).
}
\answer{
    
}