\question{11.m3}{
    Voor twee populaties mag worden verondersteld dat deze allebei een normale verdeling volgen en dezelfde onbekende variantie hebben.
    Men wil toetsen of deze populaties hetzelfde gemiddelde hebben.
    Daartoe neemt men uit beide populaties een steekproef van tien waarnemingen.
    Bij de toets is het aantal vrijheidsgraden van de toetsingsgrootheid gelijk aan $\ldots$
}
\begin{enumerate}[label=(\alph*)]
    \item $9$.
    \item $18$.
    \item $19$.
    \item $-$ (er is bij deze toetsingsgrootheid geen sprake van vrijheidsgraden).
\end{enumerate}
\answer{
    We hebben te maken met twee normaal verdeelde kansvariabelen:
    \begin{align*}
        X_1 &\sim N(\mu_1 = ?; \sigma_1 = ?) \\
        X_2 &\sim N(\mu_2 = ?; \sigma_2 = ?) 
    \end{align*} 
    Beide kansvariabelen hebben dezelfde onbekende variantie, dat wil zeggen $\sigma = \sigma_1 = \sigma_2$.
    In dat geval mogen de twee steekproefschattingen $s_1$ en $s_2$ voor de standaardafwijking (op basis van steekproeven van grootte $n = m = 10$ respectievelijk) worden gecombineerd tot \'e\'en schatter (de zogenaamde ``pooled variance'').

    De kansverdeling die de bijbehorende toetsingsgrootheid volgt is de $t$-verdeling met $df = n + m - 2 = 10 + 10 - 2 = 18$ vrijheidsgraden. 
    Het juiste antwoord is dus (b).
}
