\question{11.3}{
    De variabele $X$ is normaal verdeeld met onbekende $\mu$ en een variantie $\Var{X}=100$.
    Ook de variabele $Y$ is normaal verdeeld met onbekende verwachtingswaarde.
    Er geldt $\Var{Y}=30$.
    Van de variabele $X$ worden tien trekkingen gedaan.
    Die leveren als gemiddelde $\overline{x} = 50$.
    Er worden vijf trekkingen gedaan van de variabele $Y$ die als gemiddelde opleveren $\overline{y} = 55$.
    Toets of beide variabelen dezelfde verwachtingswaarde kunnen hebben (kies $\alpha=0,05$).
}
\answer{
    We hebben te maken met twee normaal verdeelde kansvariabelen 
    \begin{align*}
        X &\sim N(\mu_X = ?; \sigma_X = \sqrt{100}=10) \\
        Y &\sim N(\mu_Y = ?; \sigma_Y = \sqrt{30}). 
    \end{align*}
    

    Om te toetsen of de gemiddeldes van beide kansvariabelen gelijk zijn, gebruiken we de volgende nulhypothese $H_0$ en alternatieve hypothese $H_1$:
    \[
        H_0: \mu_X = \mu_Y \text{ versus } H_1: \mu_X \neq \mu_Y.
    \]
    Verder zijn het significantieniveau $\alpha = 0,05$ en de steekproefgemiddeldes $\overline{x} = 50$ en $\overline{y} = 55$ gegeven.
    Als toetsingsgrootheid kijken we naar de verschilvariabele $V = \overline{X} - \overline{Y}$.
    Hiervoor geldt dat
    \begin{align*}
        V = \overline{X} - \overline{Y} = N(\mu_V = \mu_X - \mu_Y; \sigma_V = \sqrt{\frac{\sigma_X^2}{n}+\frac{\sigma_Y^2}{m}} = \sqrt{\frac{100}{10}+\frac{30}{5}} = 4).
    \end{align*}
    
    Onder de nulhypothese geldt $\mu_X = \mu_Y$, oftewel de verschilvariabele heeft verwachtingswaarde $\mu_V = \mu_X - \mu_Y = 0$.

    {\bfseries Methode 1 (kritiek gebied):}
    
    Omdat we tweezijdig toetsen, is het kritieke gebied van de vorm $(-\infty; g_1]$ en $[g_2; \infty)$, waarbij de grenzen bepaald kunnen worden met:
    \begin{align*}
        g_1 &= \invnorm(\text{opp}=\alpha/2=0,025; \mu = \mu_V=0; \sigma = \sigma_V=4) \approx -7,8399 \\
        g_2 &= \invnorm(\text{opp}=1-\alpha/2=0,975; \mu = \mu_V=0; \sigma = \sigma_V=4) \approx 7,8399 \\
    \end{align*}
    Het geobserveerde verschil is $v = \overline{x} - \overline{y} = 50 - 55 = -5$.
    Deze waarde ligt niet in het kritieke gebied, dus we accepteren de nulhypothese $H_0$.
    Er is onvoldoende bewijs om de aanname te verwerpen dat beide variabelen dezelfde verwachtingswaarde zouden hebben.

    {\bfseries Methode 2 ($p$-waarde):} 

    Het geobserveerde verschil is $v = \overline{x} - \overline{y} = 50 - 55 = -5$ en ligt dus links van het gemiddelde $\mu_V = 0$.
    Omdat we tweezijdig toetsen, berekenen we de $p$-waarde als de linkeroverschrijdingskans van de waarde $-5$ en vergelijken deze met $\alpha/2 = 0,025$.
    \begin{align*}
        p &= \normalcdf(\text{lower}=-10^{99}; \text{upper}=-5; \mu = 0; \sigma = 4) = 0,1056
    \end{align*}
    De $p$-waarde is groter dan $\alpha/2$, dus we accepteren de nulhypothese $H_0$.
    Er is onvoldoende bewijs om de aanname te verwerpen dat beide variabelen dezelfde verwachtingswaarde zouden hebben.
}