\question{11.1}{
    Een instantie die toezicht houdt op de mate van luchtverontreiniging doet regelmatig metingen op diverse plaatsen in het land.
    De gemeten verontreiniging wordt uitgedrukt in een bepaalde index, die op alle werkdagen wordt bepaald.
    In 2018 werden enkele nieuwe richtlijnen voor de industrie afgekondigd met als doel het niveau van luchtverontreiniging te verlagen.
    Om het effect van deze maatregelen te bestuderen, werd een onderzoek gedaan.
    Hierbij werden $20$ waarden ($x_i$) die bepaald zijn in februari 2017 vergeleken met $20$ waarden ($y_i$) die zijn gemeten in februari 2018.
    Voor deze waarden werd berekend:
    \[
        \overline{x}=164,6 \text{ en } s_X = 17,2; \text{ en } \overline{y} = 143,2 \text{ en } s_Y = 15,9
    \]
    Met een verschiltoets gaan we bepalen of er een significante verbetering is opgetreden.
}
\answer{
    Laat $X$ en $Y$ de kansvariabelen zijn die het niveau van luchtverontreiniging meet op een willekeurige dag in respectievelijk februari 2017 en februari 2018.
    Van deze kansvariabelen zijn zowel de verwachtingswaarden als de standaardafwijkingen onbekend.

    Om te bepalen of er een significante verbetering is opgetreden, moeten we kijken of de gemiddelde luchtverontreiniging is gedaald in 2018.
    We voeren hiervoor een hypothesetoets uit met de volgende nulhypothese $H_0$ en de alternatieve hypothese $H_1$:
    \[
        H_0: \mu_X \le \mu_Y \text{ versus } H_1: \mu_X > \mu_Y.
    \]

    Verder wordt er uitgegaan van een significantieniveau $\alpha = 0,05$.
    Omdat de standaardafwijkingen onbekend zijn, moeten we werken met een onafhankelijke $t$-toets met behulp van de verschilvariabele $V$.
    Voor deze verschilvariabele $V$ geldt dat
    \begin{align*}
        \mu_V   &= \mu_X - \mu_Y \\
        s_V     &= \sqrt{\frac{s_X^2}{n} + \frac{s_Y^2}{m}}.
    \end{align*}

    We veronderstellen dat beide steekproeven te beschouwen zijn als trekking uit (normale) verdelingen met dezelfde variantie, dus kunnen we gebruik maken van de ``pooled variance'':
    \[
        s_P^2 = \frac{(n-1) s_X^2 + (m-1) s_Y^2}{n-1+m-1} = \frac{19 \cdot 17,2^2 + 19 \cdot 15,9^2}{38} = 274,325.
    \]

    Onder de nulhypothese $H_0$ (uitgaande van het extreme geval $\mu_X = \mu_Y$) geldt dus voor de verschilvariabele $V$ dat
    \begin{align*}
        \mu_V   &= \mu_X - \mu_Y = 0 \\
        s_V     &= \sqrt{\frac{s_P^2}{n} + \frac{s_P^2}{m}} = \sqrt{\frac{274,325}{20} + \frac{274,325}{20}} \approx 5,2376
    \end{align*}

    De $t$-waarde bijbehorende bij de uitkomst $v = \overline{x} - \overline{y} = 164,6 - 143,2 = 21,4$ is gelijk aan
    \[
        t = \frac{v - \mu_V}{s_V} = \frac{21,4 - 0}{5,2376} = 4,0858.
    \]

    Merk op dat deze $t$-waarde uit een $t$-verdeling komt met $\text{df}=n-m+2=38$ vrijheidsgraden.

    {\bfseries Methode 1 (kritiek gebied):}
    
    Omdat we rechtszijdig toetsen, is het kritieke gebied van de vorm $[g; \infty)$, waarbij de grens $g$ als volgt bepaald kan worden:
    \begin{align*}
        g &= \invt(\text{opp}=1-\alpha; \text{df} = n + m - 2) = \invt(\text{opp}=0,95; \text{df} = 38) \approx 1,6860
    \end{align*}
    De berekende waarde $t \approx 4,0858$ is groter dan deze grens $g$ is.
    Dat betekent dat $t$ in het kritieke gebied ligt, dus we verwerpen de nulhypothese $H_0$.
    Er is voldoende reden om op basis van deze steekproef aan te nemen dat er daadwerkelijk een significante daling is gerealiseerd in de luchtverontreiniging in 2018.

    {\bfseries Methode 2 ($p$-waarde):} 

    Aangezien we rechtszijdig toetsen, berekenen we de $p$-waarde als de rechteroverschrijdingskans van de waarde $t \approx 4,0858$:
    \begin{align*}
        p = \tcdf(\text{lower}=4,0858; \text{upper}=10^{99}; \text{df}=38) \approx 1,0936 \cdot 10^{-4}.
    \end{align*}
    De $p$-waarde is extreem klein (veel kleiner dan $\alpha=0,05$), dus verwerpen we de nulhypothese.
    Er is voldoende reden om op basis van deze steekproef aan te nemen dat er daadwerkelijk een significante daling is gerealiseerd in de luchtverontreiniging in 2018.
}