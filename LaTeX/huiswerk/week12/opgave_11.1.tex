\question{11.1}{
    Een instantie die toezicht houdt op de mate van luchtverontreiniging doet regelmatig metingen op diverse plaatsen in het land.
    De gemeten verontreiniging wordt uitgedrukt in een bepaalde index, die op alle werkdagen wordt bepaald.
    In 2018 werden enkele nieuwe richtlijnen voor de industrie afgekondigd met als doel het niveau van luchtverontreiniging te verlagen.
    Om het effect van deze maatregelen te bestuderen, werd een onderzoek gedaan.
    Hierbij werden $20$ waarden ($x_i$) die bepaald zijn in februari 2017 vergeleken met $20$ waarden ($y_i$) die zijn gemeten in februari 2018.
    Voor deze waarden werd berekend:
    \[
        \overline{x}=164,6 \text{ en } s_X = 17,2; \text{ en } \overline{y} = 143,2 \text{ en } s_Y = 15,9
    \]
    Met een verschiltoets gaan we bepalen of er een significante verbetering is opgetreden.
}
\answer{
    
}