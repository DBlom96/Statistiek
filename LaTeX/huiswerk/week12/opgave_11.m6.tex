\question{11.m6}{
    Voor een tweetal normale verdelingen willen we toetsen of ze dezelfde waarde van de variantie hebben.
    Voor de eerste variabele werd met een steekproef van twaalf waarnemingen de waarde $s^2 = 240$ gevonden.
    Bij de tweede variabele leverde de steekproef van tien waarnemingen de waarde $s^2 = 160$.
    De toetsingsgrootheid krijgt hier dan de waarde $\ldots$
}
\begin{enumerate}[label=(\alph*)]
    \item $80$.
    \item $1,50$.
    \item $4$.
    \item $1,25$.
\end{enumerate}
\answer{
    We hebben te maken met twee normaal verdeelde kansvariabelen 
    \begin{align*}
        X_1 &\sim N(\mu_1 = ?; \sigma_1 = ?) \\
        X_2 &\sim N(\mu_2 = ?; \sigma_2 = ?). 
    \end{align*}
    Van deze twee kansvariabelen willen we toetsen of de varianties gelijk zijn, oftewel dat geldt $\sigma^2 = \sigma_1^2 = \sigma_2^2$.
    Hiervoor moeten we een $F$-toets uitvoeren.
    
    Voor een $F$-toets werken we met de volgende nulhypothese $H_0$ en de alternatieve hypothese $H_1$.
    \[
        H_0: \sigma_1^2 = \sigma_2^2 \text{ versus } H_1: \sigma_1^2 \neq \sigma_2^2.
    \]
    Verder hebben we voor beide kansvariabelen twee steekproeven van respectievelijk grootte $n=12$ en $m=10$ getrokken om de varianties te schatten, met als resultaat $s_1^2 = 240$ en $s_2^2 = 160$.
    De toetsingsgrootheid $F$ is gelijk aan
    \[
        F = \frac{S_1^2}{S_2^2}
    \]
    en volgt onder $H_0$ de $F(n-1, m-1)$-verdeling.
    De geobserveerde toetsingsgrootheid is in dit geval gelijk aan
    \[
        f = \frac{s_1^2}{s_2^2} = \frac{240}{160} = 1,50.
    \]

    Het juiste antwoord is dus (b).
}
