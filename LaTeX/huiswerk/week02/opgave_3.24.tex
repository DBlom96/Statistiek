\question{3.24}{In een fabriek wordt de kwaliteit gecontroleerd van de uitgaande producten.
De employ\'e die de controle verrichtte, blijkt 1\% van alle goede producten af te keuren en verder keurt hij 5\% van alle slechte producten goed.
De totale productie bestaat voor 90\% uit goede producten.}
\begin{enumerate}[label=(\alph*)]
    \item Bereken de kans dat een willekeurig product goed is en wordt goedgekeurd.
    \answer{
        We zetten de relatieve frequenties even in een tabel:
        \begin{center}
            \resizebox{0.95\textwidth}{!}{
                \begin{tabular}{c|cc|c}
                    \toprule
                        & Goedgekeurd & Afgekeurd & \textbf{Totaal}\\
                    \midrule 
                        Goed product & 99\% van 90\% = 89,1\% & 1\% van 90\% = 0,9\% & 90\% \\
                        Slecht product & 5\% van 10\% = 0,5\% & 95\% van 10\% = 9,5\% & 10\% \\
                    \midrule
                        \textbf{Totaal} & 89,6\% & 10,4\% & 100\% \\
                    \bottomrule
                \end{tabular}
            }
        \end{center}
        Aflezen uit de cel voor ``goed product'' en ``goedgekeurd'' geeft een kans van $89,1\%\rightarrow 0,891$.
    }
    
    \item Hoe groot is de kans dat de controleur voor een willekeurig product de verkeerde beslissing neemt?
    \answer{
        Deze kans kunnen we opnieuw aflezen uit de tabel, namelijk de som van de cellen (``goed product'', ``afgekeurd'') en (``slecht product'', ``goedgekeurd'').
        Dit geeft een kans van $0,9\% + 0,5\% = 1,4\%\rightarrow 0,014$.
    }

    \item Hoe groot is het percentage goedgekeurde producten dat de fabriek verlaat?
    \answer{
        Deze kans kunnen we opnieuw aflezen uit de tabel, namelijk het kolomtotaal voor de kolom ``goedgekeurd''.
        Hieruit volgt dat 89,6\% van de producten wordt goedgekeurd.
    }
\end{enumerate}