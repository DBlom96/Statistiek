\question{4.m2}{Voor de populatie Nederlandse hbo-studenten is vastgesteld dat het aantal 
verschillende statistiekboeken waaruit zij wel eens hebben gestudeerd, kan worden beschreven door een kansvariabele $X$
die in de volgende tabel is weergegeven}

\begin{center}
    \begin{tabular}{cccccc}
        \toprule
            {\bfseries Aantal boeken ($k$)} & $0$ & $1$ & $2$ & $3$ & $4$ \\
        \cmidrule{1-1} \cmidrule{2-2} \cmidrule{3-3} \cmidrule{4-4} \cmidrule{5-5} \cmidrule{6-6} 
            $P(X=k)$ & $0,15$ & $0,40$ & $0,30$ & $0,10$ & $0,05$ \\
        \bottomrule
    \end{tabular}
\end{center}


De kans dat een willekeurige student heeft gestudeerd uit minstens twee statistiekboeken is daardoor gelijk aan ...
\begin{enumerate}[label=(\alph*)]
    \item $0,30$
    \item $0,85$
    \item $0,45$
    \item $0,15$
\end{enumerate}
\answer{Het juiste antwoord is (c)}