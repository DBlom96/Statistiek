\question{3.11}{Bij een exameninstituut voor rijexamens zijn drie examinatoren in dienst: Dirk, Judith en Ilse.
Van alle kandidaten doet 40\% examen bij Dirk, 40\% doet examen bij Judith en 20\% bij Ilse. Bekend is dat het slagingspercentage voor de drie examinatoren verschilt.
Dat is 40\% bij Dirk, 60\% bij Judith en 70\% bij Ilse.}
\begin{enumerate}[label=(\alph*)]
    \item Kandidaat Peter van der Meer gaat rijexamen doen bij een van de drie examinatoren. Hoe groot is de kans dat hij slaagt?
    \answer{
        Definieer de gebeurtenissen 
        \begin{itemize}
            \item $S = \{\text{Peter slaagt}\}$
            \item $J = \{\text{Peter doet examen bij Judith}\}$
            \item $I = \{\text{Peter doet examen bij Ilse}\}$
        \end{itemize}
        Uit de vraag volgt dan dat $P(D)=0,4, P(J)=0,4$ en $P(I)=0,2$, evenals de conditionele kansen $P(S|D)=0,4$, $P(S|J)=0,6$ en $P(S|I)=0,7$.
        In dat geval is de kans dat Peter slaagt gelijk aan
        \begin{align*}
            P(S)    &= P(S|D)\cdot P(D) + P(S|J) \cdot P(J) + P(S|I) \cdot P(I) \\
                    &= 0,4 \cdot 0,4 + 0,6 \cdot 0,4 + 0,7 \cdot 0,2 \\
                    &= 0,54
        \end{align*}
        Peter slaagt dus met $54\%$ kans voor zijn rijexamen.
    }
    
    \item Kandidaat Peter blijkt geslaagd te zijn. Hoe groot is de kans dat hij examen heeft afgelegd bij Ilse?
    \answer{
        In dit geval willen we de kans $P(I|S)$ bepalen. Uit de regel van Bayes volgt dan:
        \[
            P(I|S) = \frac{P(S|I)\cdot P(I)}{P(S)} = \frac{0,7 \cdot 0,2}{0,54} \approx 0,2593
        \]
        Met 25,93\% kans heeft Peter zijn examen afgelegd bij Ilse, gegeven dat hij is geslaagd.
    }
\end{enumerate}
Sommige gezakte kandidaten dienen een klacht in over het examen. 
Op basis van ervaring weten we dat 10\% van de gezakten van examinator Dirk een klacht indient.
Bij Judith is dat 6\% en bij Ilse is dat 3\%.
\begin{enumerate}[label=(\alph*)]
    \setcounter{enumi}{3}
    \item Er komt een klacht binnen over het examen zonder vermelding van de naam van de examinator.
    Hoe groot is de kans dat de klacht over Dirk gaat?
    \answer{
        De relatieve frequenties van combinaties van examinatoren en al dan niet geslaagd zijn zijn als volgt:
        \begin{center}
            \begin{tabular}{c|cc|c}
                \toprule
                    Examinator & wel geslaagd & niet geslaagd & \textbf{Totaal}\\
                \midrule 
                    Dirk & 40\% van 40\% = 16\% & 60\% van 40\% = 24\% & 40\% \\
                    Judith & 60\% van 40\% = 24\% & 40\% van 40\% = 16\% & 40\% \\
                    Ilse & 70\% van 20\% = 14\% & 30\% van 20\% = 6\% & 20\% \\
                \midrule
                    \textbf{Totaal} & 54\% & 46\% & 100\% \\
                \bottomrule
            \end{tabular}
        \end{center}
        Het percentage deelnemers dat zakt en een klacht over Dirk instuurt is dus 10\% van 24\% = 2,4\%.
        Daarnaast is het percentage deelnemers dat zake en een klacht instuurt (over alle examinatoren) gelijk aan
        \[
            0,1 \cdot 0,24 + 0,06 \cdot 0,16 + 0,03 \cdot 0,06 = 0,0354 \rightarrow 3,54\%
        \]
        De kans dat een klacht over Dirk gaat is dus gelijk aan $\frac{2,4}{3,54} \approx 0,68 \rightarrow 68\%$.
    }
\end{enumerate}