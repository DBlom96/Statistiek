\question{4.m5}{Het aantal koelkasten dat wekelijks wordt verkocht op de afdeling ``witgoed'' van een supermarkt, kan worden weergegeven door de kansvariabele $X$ waarvan de kansfunctie is weergegeven in de volgende tabel:
    \begin{center}
        \begin{tabular}{cccccc}
            \toprule
                {\bfseries Aantal uitkomsten $k$} & $0$ & $1$ & $2$ & $3$ & $4$\\
            \cmidrule{1-1} \cmidrule{2-2} \cmidrule{3-3} \cmidrule{4-4} \cmidrule{5-5} \cmidrule{6-6} 
                $P(X=k)$ & $0,25$ & $0,40$ & $0,20$ & $0,10$ & $0,05$\\
            \bottomrule
        \end{tabular}
    \end{center}
    
    De standaarddeviatie van de variabele $X$ bedraagt dus $\ldots$
}
\begin{enumerate}[label=(\alph*)]
    \item $1,3$ koelkasten
    \item $1,1$ koelkasten
    \item $0,93$ koelkasten
    \item $1,21$ koelkasten
\end{enumerate}
\answer{Het juiste antwoord is (b)}