\question{4.9}{Een huiseigenaar in Amsterdam besluit drie kamers te huur aan te bieden via Airbnb.
Het dagelijks aantal verhuurde kamers op een willekeurige dag kan worden weergegeven door de kansvariabele $X$, zoals weergegeven in de tabel:}

\begin{center}
    \begin{tabular}{cc}
        \toprule
            {\bfseries Aantal boekingen} & {\bfseries $P(X=k)$}\\
        \cmidrule{1-1} \cmidrule{2-2}
            $0$ & $0,50$ \\
            $1$ & $0,25$ \\
            $2$ & $0,15$ \\
            $3$ & $0,10$ \\
        \cmidrule{1-1} \cmidrule{2-2}
            {\bfseries Totaal} & $1,00$\\
        \bottomrule
    \end{tabular}
\end{center}

    \begin{enumerate}[label=(\alph*)]
        \item Hoe groot is de kans dat op een willekeurige dag minstens twee kamers zijn verhuurd.
        \answer{
            We berekenen deze kans als volgt:
            \begin{align*}
                P(X\ge 2)   &= P(X=2) + P(X=3) \\
                            &= 0,15 + 0,10 \\
                            &= 0,25
            \end{align*}
        }

        \item De aantallen boekingen op twee opeenvolgende dagen kan men beschouwen als onderling onafhankelijke kansvariabelen.
        Maak een kanstabel voor de variabele $X_{\text{som}}$ waarmee het totaal aantal boeking per twee dagen wordt aangegeven.
        Hoe groot is de kans dat in twee dagen meer dan 3 boekingen plaatsvinden?
        \answer{
            De kanstabel voor de somvariabele $X_{\text{som}}$ ziet er als volgt uit:
            \begin{center}
                \begin{tabular}{cc}
                    \toprule
                        {\bfseries Aantal boekingen} & {\bfseries $P(X_{\text{som}}=k)$}\\
                    \cmidrule{1-1} \cmidrule{2-2}
                        $0$ & $0,25$ \\
                        $1$ & $0,25$ \\
                        $2$ & $0,2125$ \\
                        $3$ & $0,175$ \\
                        $4$ & $0,0725$ \\
                        $5$ & $0,03$ \\
                        $6$ & $0,01$ \\
                    \cmidrule{1-1} \cmidrule{2-2}
                        {\bfseries Totaal} & $1,00$\\
                    \bottomrule
                \end{tabular}
            \end{center}
            De kans dat er in twee dagen meer dan 3 boekingen plaatsvinden is dus gelijk aangeeft
            \begin{align*}
                P(X_{\text{som}} > 3)   &= P(X_{\text{som}}= 4) + P(X_{\text{som}}= 5) + P(X_{\text{som}}= 6) \\
                                        &= 0,0725 + 0,03 + 0,01 \\
                                        &= 0,1125
            \end{align*}
        }
\end{enumerate}