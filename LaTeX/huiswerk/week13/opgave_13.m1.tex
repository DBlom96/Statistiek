\question{13.m1}{
    Bij regressieanalyse wordt de grootheid die aangeeft hoe de variabele $Y$ (gemiddeld) reageert op veranderingen in de variabele $X$ aangeduid met $\ldots$
    
    \begin{enumerate}[label=(\alph*)]
        \item de constante term.
        \item de storingsterm.
        \item de richtingsco\"effici\"ent van de regressielijn.
        \item de correlatieco\"effici\"ent.
    \end{enumerate}
}
\answer{
    Bij een regressieanalyse op de onafhankelijke (verklarende) variabele $X$ en de afhankelijke (te verklaren) variabele $Y$ proberen we een lineair verband te vinden.
    De regressielijn is van de vorm $Y = \alpha + \beta X$, waarbij $\alpha$ de constante term is en $\beta$ de richtingsco\"effici\"ent van de regressielijn.
    Deze richtingsco\"effici\"ent geeft aan hoe de variabele $Y$ gemiddeld reageert op een verandering van $X$.

    Het juiste antwoord is dus (c).
}
