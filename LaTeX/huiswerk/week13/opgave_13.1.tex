\question{13.1}{
    Een fabrikant van synthetische vezels onderzoekt of het krimpen van de vezels samenhangt met de temperatuur waarbij ze worden gewassen.
    Er wordt achtmaal een proef verricht waarbij de vezels gedurende $30$ minuten aan een bepaalde temperatuur worden blootgesteld.
    De geconstateerde krimp werd (in procenten van de oorspronkelijke lengte) als volgt vastgesteld:
    \begin{center}
        \begin{tabular}{ccccccccc}
            \toprule
                {\bfseries Temperatuur \boldmath$\degree C$} & $60$ & $70$ & $80$ & $90$ & $100$ & $75$ & $85$ & $100$ \\
            \midrule
                {\bfseries Krimp (\%)} & $1,2$ & $1,9$ & $2,8$ & $3,8$ & $4,2$ & $2,6$ & $3,2$ & $4,5$ \\
            \bottomrule
        \end{tabular}
    \end{center}
    Het verband tussen temperatuur en krimp willen we beschrijven door een regressielijn te berekenen op basis van de waargenomen uitkomsten.

}
\begin{enumerate}[label=(\alph*)]
    \item Welke sommaties moeten we berekenen om de regressieco\"effici\"enten te kunnen bepalen?
    \answer{

    }

    \item Bereken de sommaties in een rekenschema.
    \answer{

    }

    \item Bereken de co\"effici\"enten $a$ en $b$ van de lineaire vergelijking.
    \answer{
        
    }
\end{enumerate}