\question{13.4}{
    De Stichting voor Veilig Verkeer doet onderzoek naar de invloed van alcohol op het bloedalcoholgehalte (promillage).
    In een onderzoek onder tien mannen heeft men het promillage gemeten, twee uur na het begin van het drinken van vijf glazen bier.
    Tevens heeft men van deze tien personen het lichaamsgewicht gemeten.
    De gegevens staan in de tabel.
    \begin{center}
        \begin{tabular}{cc}
            \toprule
                {\bfseries Promillage} & {\bfseries Gewicht (in kg)} \\
            \cmidrule{1-1} \cmidrule{2-2}
                $1,06$ & $61$ \\
                $0,77$ & $82$ \\
                $0,72$ & $86$ \\
                $0,95$ & $70$ \\
                $0,65$ & $96$ \\
                $0,83$ & $80$ \\
                $0,99$ & $67$ \\
                $0,73$ & $90$ \\
                $0,84$ & $75$ \\
                $0,96$ & $73$ \\
            \bottomrule
        \end{tabular}
    \end{center}
    Een onderzoeker heeft het idee dat het gewicht van een persoon invloed heeft op het bloedalcoholgehalte en onderzoekt dit met regressie.
}
\begin{enumerate}[label=(\alph*)]
    \item Welke van de variabelen is de te verklaren variabele?
    \answer{

    }

    \item Stel de vergelijking van de regressielijn op.
    \answer{

    }

    \item Bereken de correlatieco\"effici\"ent.
    \answer{
        
    }

    \item Geef het voorspelde promillage (twee uur na het begin van het drinken van vijf glazen bier) voor een man van $85$ kg.
    \answer{

    }

    \item Uit onderzoek is gebleken: het promillage ligt, twee uur na het begin van het drinken van zes glazen bier (bij mannen) $20\%$ hoger dan na het drinken van vijf glazen.
    Stel de vergelijking van de regressielijn voor dit geval op.
    \answer{
        
    }
\end{enumerate}