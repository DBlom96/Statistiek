\question{13.11}{
    Voor een groep studenten wordt een tweetal testscores vastgesteld.
    Een voor de $A$-vakken en een voor de $B$-vakken.
    De resultaten waren als volgt:
    \begin{center}
        \begin{tabular}{cccccccc}
            \toprule
                {\bfseries Student} & $A$ & $B$ & $C$ & $D$ & $E$ & $F$ & $G$ & $H$ \\
            \cmidrule{1-1} \cmidrule{2-2} \cmidrule{3-3} \cmidrule{4-4} \cmidrule{5-5} \cmidrule{6-6} \cmidrule{7-7}
                {\bfseries $A$-score} & $40$ & $60$ & $50$ & $45$ & $74$ & $53$ & $45$ & $42$ \\
                {\bfseries $B$-score} & $88$ & $35$ & $45$ & $80$ & $35$ & $60$ & $70$ & $85$ \\
            \bottomrule
        \end{tabular}
    \end{center}
    Bereken hiervoor de rangcorrelatieco\"effici\"ent van Spearman.
}
\answer{
    
}