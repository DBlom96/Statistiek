\question{13.11}{
    Voor een groep studenten wordt een tweetal testscores vastgesteld.
    Een voor de $A$-vakken en een voor de $B$-vakken.
    De resultaten waren als volgt:
    \begin{center}
        \begin{tabular}{ccccccccc}
            \toprule
                {\bfseries Student}   & $A$ & $B$ & $C$ & $D$ & $E$ & $F$ & $G$ & $H$ \\
            \cmidrule{1-1} \cmidrule{2-2} \cmidrule{3-3} \cmidrule{4-4} \cmidrule{5-5} \cmidrule{6-6} \cmidrule{7-7} \cmidrule{8-8} \cmidrule{9-9}
                {\bfseries $A$-score} & $40$ & $60$ & $50$ & $45$ & $74$ & $53$ & $45$ & $42$ \\
                {\bfseries $B$-score} & $88$ & $35$ & $45$ & $80$ & $35$ & $60$ & $70$ & $85$ \\
            \bottomrule
        \end{tabular}
    \end{center}
    Bereken hiervoor de rangcorrelatieco\"effici\"ent van Spearman.
}
\answer{
    De eerste stap bij het berekenen van Spearman's correlatieco\"effici\"ent is het bepalen van de rankings van de uitkomsten voor de $A$- en $B$-scores:
        \begin{center}
            \renewcommand{\arraystretch}{1.25}
            \begin{tabular}{ccccccccc}
                \toprule
                    {\bfseries Rangnummers $A$-scores} & $1$ & $7$ & $5$ & $3,5$ & $8$ & $6$ & $3,5$ & $2$ \\
                    {\bfseries Rangnummers $B$-scores} & $8$ & $1,5$ & $3$ & $6$ & $1,5$ & $4$ & $5$ & $7$ \\
                \midrule
                    {\bfseries Verschillen $d_i$} & $-7$ & $5,5$ & $2$ & $-2,5$ & $6,5$ & $2$ & $-1,5$ & $-5$ \\
                    {\bfseries Kwadratische verschillen $d_i^2$} & $49$ & $30,25$ & $4$ & $6,25$ & $42,25$ & $4$ & $2,25$ & $25$ \\
                \bottomrule
            \end{tabular}
        \end{center}

        De som van de kwadratische rangnummerverschillen is gelijk aan $\sum_i d_i^2 = 163$.
        Aangezien de steekproefgrootte gelijk is aan $n = 8$, is de rangcorrelatieco\"effici\"ent van Spearman gelijk aan
        \begin{align*}
            r_s &= 1 - \frac{ 6 \cdot \sum_i d_i^2 }{ n^3 - n }  \\
                &= 1 - \frac{ 6 \cdot 163 }{ 8^3 - 8 }  \\
                &\approx -0,9405.
        \end{align*}

        Omdat de rangcorrelatieco\"effici\"ent $r_s$ dichtbij $-1$ ligt, geldt dat de rangnummers een nagenoeg tegengestelde samenhang vertonen.

        {
            \itshape Side note: merk op dat we in deze opgave meerdere keren dezelfde $A$-scores hebben waargenomen (in dit bijvoorbeeld 45).
            In dat geval zijn deze respectievelijk op de derde en vierde plaats in de ranking (van laag naar hoog).
            Omdat je niet eenduidig kunt zeggen welke van de twee derde is en welke vierde, wordt uitgegaan van het gemiddelde van de rankings, in dit geval dus $3,5$.
        }
}