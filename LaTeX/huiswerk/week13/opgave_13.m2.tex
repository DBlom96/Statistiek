\question{13.m2}{
    Als bij een regressieanalyse een correlatieco\"effici\"ent wordt gevonden ter grootte van $-0,50$, dan $\ldots$
    \begin{enumerate}[label=(\alph*)]
        \item moet er een rekenfout zijn gemaakt.
        \item is de variabele $Y$ ongeschikt als verklarende variabele.
        \item zullen bij hogere $X$-waarden doorgaans lagere $Y$-waarden worden aangetroffen.
        \item kan de storingsterm geen positieve variantie hebben.
    \end{enumerate}
}
\answer{
    Een negatieve correlatieco\"effici\"ent duidt op het feit dat als de verklarende variabele $X$ stijgt, dan is er een negatieve trend voor de te verklaren variabele $Y$.
    Dat houdt dus in dat bij hogere $X$-waarden doorgaans lagere $Y$-waarden worden aangetroffen.
    
    Het juiste antwoord is dus (c).
}
