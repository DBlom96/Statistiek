\question{13.m6}{
    Bij een regressieanalyse wordt het verband onderzocht tussen de omzet $Y$ ($\times 10 000$ euro) van een bedrijf en de reclame-uitgaven $X$ ($\times 1 000$ euro) in de voorafgaande periode.
    De leverde als regressievergelijking op:
    \[
        Y = 220 + 32 X.
    \]
    Stel dat in een nieuwe periode $3 000$ euro wordt uitgegeven aan reclame, dan is de voorspelde omzet op basis van de regressievergelijking gelijk aan
    \begin{enumerate}[label=(\alph*)]
        \item $756 000$ euro.
        \item $2 296 000$ euro.
        \item $96 220$ euro.
        \item $3 160 000$ euro.
    \end{enumerate}
}
\answer{
    Stel dat in een nieuwe periode $3 000$ euro wordt uitgegeven, dat betekent dus dat $X = 3$.
    Invullen in de regressielijn geeft ons $Y = 220 + 32 \cdot 3 = 316$.
    Aangezien $Y$ gemeten wordt in tienduizenden euro's, betekent dit dat de voorspelde omzet gelijk is aan $316 \cdot 10 000 = 3 160 000$ euro.
    Het juiste antwoord is dus (d).
}
