\question{13.5}{
    De weerstand van een blokje van een bepaalde metaalsoort hangt af van de temperatuur.
    Bij verschillende temperaturen ($X$) werd de weerstand ($Y$) gemeten.
    De resultaten zijn weergegeven in de volgende tabel:
    \begin{center}
        \begin{tabular}{ccccccc}
            \toprule
                {\bfseries Temperatuur $\calc C$} & $-100$ & $-50$ & $0$ & $50$ & $100$ & $150$ & $200$ \\
            \cmidrule{1-1} \cmidrule{2-2} \cmidrule{3-3} \cmidrule{4-4} \cmidrule{5-5} \cmidrule{6-6} \cmidrule{7-7}
                {\bfseries Weerstand ($\Omega$)} & $13,2$ & $17,5$ & $21,3$ & $25,4$ & $28,9$ & $32,8$ & $36,8$ \\
            \bottomrule
        \end{tabular}
    \end{center}
   }
\begin{enumerate}[label=(\alph*)]
    \item Ga ervan uit dat het verband tussen temperatuur en weerstand lineair is.
    Bepaal de regressielijn met de temperatuur als verklarende variabele.
    \answer{

    }

    \item Welke weerstand kan men verwachten bij $400\calc C$?
    \answer{

    }

    \item Bij experimenten als het hier genoemde wordt de temperatuur vaak uitgedrukt in Kelvin ($K$).
    Hoe luidt de vergelijking van de regressielijn als de temperatuur in $K$ wordt uitgedrukt?
    (Het nulpunt van de Kelvin-temperatuurschaal ligt bij $-273\calc C$.
    Een stijging met $1\calc C$ is gelijk aan een stijging met $1 K$.)
    \answer{
    
    }
\end{enumerate}