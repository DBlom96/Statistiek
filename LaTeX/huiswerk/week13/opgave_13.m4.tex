\question{13.m4}{
    Als bij een enkelvoudige lineaire regressieanalyse blijkt dat geldt $\alpha=0$ voor de berekende regressielijn, dan $\ldots$
    \begin{enumerate}[label=(\alph*)]
        \item geldt altijd $\alpha=0$ voor het regressiemodel.
        \item is de correlatie tussen $X$ en $Y$ gelijk aan nul.
        \item gaat de berekende regressielijn door de oorsprong.
        \item loopt de berekende lijn horizontaal.
    \end{enumerate}
}
\answer{
    Een regressielijn is altijd van de vorm $Y = \alpha + \beta X$.
    Als dan blijkt dat $\alpha=0$, houden we een vorm $Y = \beta X$ over.
    Zodra we $X = 0$ invullen, krijgen we ook $Y = \beta \cdot 0 = 0$.
    Dit betekent dus dat de regressielijn door de oorsprong, oftewel het punt $(0,0)$ gaat.
    
    Het juiste antwoord is dus (c).
}
