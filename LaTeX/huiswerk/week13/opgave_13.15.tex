\question{13.15}{
    Er is een onderzoek gehouden naar het verband tussen motorinhoud en maximumsnelheid van personenauto's.

    \(  \)
    Er werden tien auto's getest.
    De resultaten waren als volgt:
    \begin{center}
        \begin{tabular}{ccc}
            \toprule
                {\bfseries Auto nr.} & {\bfseries Cilinderinhoud} & {\bfseries Onderhoudskosten (\boldmath$Y$) in de laatste $12$ maanden} \\            \cmidrule{1-1} \cmidrule{2-2} \cmidrule{3-3} \cmidrule{4-4} \cmidrule{5-5} \cmidrule{6-6} \cmidrule{7-7}
            \cmidrule{1-1} \cmidrule{2-2} \cmidrule{3-3}
                $I$ & $1,2$ & $140$ \\
                $II$ & $0,8$ & $110$ \\
                $III$ & $0,8$ & $100$ \\
                $IV$ & $2,0$ & $180$ \\
                $V$ & $1,4$ & $150$ \\
                $VI$ & $1,0$ & $100$ \\
                $VII$ & $1,6$ & $160$ \\
                $VIII$ & $1,8$ & $190$ \\
                $IX$ & $1,3$ & $140$ \\
                $X$ & $1,1$ & $130$ \\
            \bottomrule
        \end{tabular}
    \end{center}
}
\begin{enumerate}[label=(\alph*)]
    \item Teken de gegevens in een spreidingsdiagram.
    \answer{

    }

    \item Bepaal het verband tussen de variabelen met behulp van lineaire regressie.
    \answer{

    }

    \item Voorspel aan de hand van en de gevonden lijn de maximumsnelheid van een auto met een cilinderinhoud van $1,5$ liter.
    \answer{
    
    }

    \item Geef een schatting van de variantie van de storingsterm.
    \answer{

    }

    \item Geef een $95\%$-voorspellingsinterval voor de maximumsnelheid van een auto waarvan is gegeven dat deze een cilinderinhoud van $1,45$ liter heeft.
    \answer{

    }
\end{enumerate}