% === Week 2 ===
\chapter*{Week 2: discrete kansvariabelen}

\section*{Hoofdstuk 4}

\question{4.m1}{We besluiten driemaal een muntstuk op te gooien en we tellen het aantal malen dat de uitkomst ``kop'' verschijnt. Dit aantal malen ``kop'' ...}
\begin{enumerate}[label=(\alph*)]
    \item is geen kansvariabele, omdat de uitkomsten ``kop'' en ``munt'' geen getallen zijn.
    \item is een continue kansvariabele, omdat dit spelletje eindeloos vaak kan worden herhaald.
    \item laat altijd een waarde tussen 1 en 3 zien.
    \item is een discrete variabele, omdat slechts eindig veel uitkomsten hiervoor mogelijk zijn.
\end{enumerate}
\answer{Het juiste antwoord is (d)}

\question{4.m2}{Voor de populatie Nederlandse hbo-studenten is vastgesteld dat het aantal 
verschillende statistiekboeken waaruit zij wel eens hebben gestudeerd, kan worden beschreven door een kansvariabele $X$
die in de volgende tabel is weergegeven}

\begin{tabular}{cccccc}
    \toprule
        {\bfseries Aantal boeken ($k$)} & $0$ & $1$ & $2$ & $3$ & $4$ \\
    \cmidrule{1-1} \cmidrule{2-2} \cmidrule{3-3} \cmidrule{4-4} \cmidrule{5-5} \cmidrule{6-6} 
        $\mathbf{P(X=k)}$ & $0,15$ & $0,40$ & $0,30$ & $0,10$ & $0,05$ \\
    \bottomrule
\end{tabular}

De kans dat een willekeurige student heeft gestudeerd uit minstens twee statistiekboeken is daardoor gelijk aan ...
\begin{enumerate}[label=(\alph*)]
    \item $0,30$
    \item $0,85$
    \item $0,45$
    \item $0,15$
\end{enumerate}
\answer{Het juiste antwoord is (c)}

\question{4.m3}{Een kansvariabele $X$ kan $n$ verschillende waarden aannemen, namelijk $k_1, k_2, \ldots, k_n$.
Voor de kansfunctie van de variabele $X$ moet dan gelden:
\[
    \sum f(k) = \sum P(X=k) = \ldots
\]
}
\begin{enumerate}[label=(\alph*)]
    \item $0$
    \item $1$
    \item $\frac{1}{n}$
    \item $n$
\end{enumerate}
\answer{Het juiste antwoord is (b)}

\question{4.m4}{Van de kansvariabele $X$ is de verdelingsfunctie als volgt:
    \begin{tabular}{cccccccc}
        \toprule
            {\bfseries Uitkomst $k$} & $10$ & $11$ & $12$ & $13$ & $14$ & $15$ & $16$\\
        \cmidrule{1-1} \cmidrule{2-2} \cmidrule{3-3} \cmidrule{4-4} \cmidrule{5-5} \cmidrule{6-6} 
            $\mathbf{F(k)}$ & $0,16$ & $0,28$ & $0,46$ & $0,66$ & $0,84$ & $0,94$ & $1,00$\\
        \bottomrule
    \end{tabular}
    Bereken $P(X=14)$. Wat is de uitkomst?
}
\begin{enumerate}[label=(\alph*)]
    \item $0,18$
    \item $0,84$
    \item $0,10$
    \item $0,16$
\end{enumerate}
\answer{Het juiste antwoord is (a)}

\question{4.m5}{Het aantal koelkasten dat wekelijks wordt verkocht op de afdeling ``witgoed'' van een supermarkt, kan worden weergegeven door de kansvariabele $X$ waarvan de kansfunctie is weergegeven in de volgende tabel:
    \begin{tabular}{cccccc}
        \toprule
            {\bfseries Aantal uitkomsten $k$} & $0$ & $1$ & $2$ & $3$ & $4$\\
        \cmidrule{1-1} \cmidrule{2-2} \cmidrule{3-3} \cmidrule{4-4} \cmidrule{5-5} \cmidrule{6-6} 
            $\mathbf{P(X=k)}$ & $0,25$ & $0,40$ & $0,20$ & $0,10$ & $0,05$\\
        \bottomrule
    \end{tabular}
    De standaarddeviatie van de variabele $X$ bedraagt dus $\ldots$
}
\begin{enumerate}[label=(\alph*)]
    \item $1,3$ koelkasten
    \item $1,1$ koelkasten
    \item $0,93$ koelkasten
    \item $1,21$ koelkasten
\end{enumerate}
\answer{Het juiste antwoord is (b)}

