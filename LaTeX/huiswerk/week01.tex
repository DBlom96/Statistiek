% === Week 1 ===
\section*{Week 1: grondslagen statistiek en datavisualisatie}

\subsection{Hoofdstuk 1}
\question{m1}{Bij een verkeersonderzoek is een van de grootheden die wordt genoteerd het merk van de passerende autos's Dit merk is ...}
\begin{enumerate}[label=(\alph*)]
    \item een ratiovariabele
    \item een kwantitatieve variabele
    \answer{\item een nominale variabele.}
    \item geen variabele
\end{enumerate}

\question{m2}{Zie tabel. Bij de mannen is het percentage dat het eens is met de voorgestelde maatregelen gelijk aan ...}
\begin{enumerate}[label=(\alph*)]
    \item $41,7\%$
    \item $10\%$
    \item $25\%$
    \answer{\item $22,7\%$}
\end{enumerate}

\question{m3}{Zie tabel. De groep mensen die het oneens is met de maatregelen bestaat voor ... uit vrouwen.}
\begin{enumerate}[label=(\alph*)]
    \item $20\%$
    \item $56\%$
    \answer{\item $50\%$}
    \item $40\%$
\end{enumerate}

Bij de reisorganisatie $P$-tours heeft men bijgehouden hoeveel geboekte passagiers kort voor het vertrek van een busreis hun reis afzeggen.
Voor 80 busreizen leverde dit de volgende tabel:

\begin{table}[h!]
    \centering
    \caption*{Aantal passagiers per busreis dat annuleert}
    \begin{tabular}
    
        \toprule
            {\bfseries Nummer & Aantal afzeggers & Frequentie} \\
        \midrule
            1 & 0 -< 5 & 36 \\
            2 & 5 -< 8 & 24 \\
            3 & 8 -< 12 & 8 \\
            4 & 12 -< 16 & 6 \\
            5 & 16 -< 20 & 4 \\
            6 & 20 en hoger & 2 \\

        \midrule    
            Totaal & & $80$ \\
        \bottomrule
    \end{tabular}
\end{table}
\question{m4}{De klassenbreedte bij 0 -< 5 bedraagt \dots}
\begin{enumerate}[label=(\alph*)]
    \answer{\item $5$}
    \item $4$
    \item $4,5$
    \item $36$
\end{enumerate}
\question{m5}{De relatieve frequentie van de klasse 16 -< 20 bedraagt \dots}
\begin{enumerate}[label=(\alph*)]
    \answer{\item $0,05$}
    \item $4$
    \item $16$
    \item $0,20$
\end{enumerate}
\question{m6}{Het klassenmidden van de klasse 8 -< 12 bedraagt \dots}
\begin{enumerate}[label=(\alph*)]
    \item $2$
    \item $10$
    \item $10,5$
    \answer{\item $9,5$}
\end{enumerate}

\question{1.3}{Geef voor elk van de volgende gevallen aan of de genoemde verzameling} als een steekproef of als een populatie mag worden beschouwd:
\begin{enumerate}[label=(\alph*)]
    \item de commissarissen van de koning van de 12 Nederlandse provincies
    \item de 200 personen die zijn ge\"interviewd bij een straatenqu\^ete
    \item de 150 automobilisten die moesten stoppen voor een alcoholcontrole
    \item de 740 leden van een studentenvereniging
    \item de 38 klanten die tussen 11.00 en 12.00 uur een postkantoor binnenkomen
    \item de 12000 verzekerden bij een verzekeringsmaatschappij
    \item de 20 nummers die worden gedraaid in een muziekprogramma op de radio
\end{enumerate}

\answer{
    \begin{enumerate}[label=(\alph*)]
        \item Populatie
        \item Steekproef
        \item Steekproef
        \item Populatie
        \item Steekproef
        \item Populatie
        \item Steekproef
    \end{enumerate} 
}

\question{1.4}{Geef voor de volgende variabelen aan of deze een nominale, ordinale, interval- of ratioschaal heeft:

    \begin{enumerate}[label=(\alph*)]
        \item de speelduur van een dvd
        \item de kleur van tulpen
        \item de industrietak waarin werknemers een baan hebben
        \item de jaaromzet (in euro) van bedrijven
        \item het aantal sterren dat de moeilijkheidsgraad van puzzelboekjes aangeeft
        \item de hoogte boven de zeespiegel van wintersportdorpen
    \end{enumerate}
}

\answer{
    \begin{enumerate}[label=(\alph*)]
        \item Ratio
        \item Nominaal
        \item Nominaal
        \item Ratio
        \item Ordinaal
        \item Ratio
    \end{enumerate} 
}

\question{1.7}{Een groep van dertig eerstejaarsstudenten is een aantal vragen voorgelegd. Dit betrof:
    \begin{itemize}
        \item Leeftijd
        \item Woonsituatie (z=zelfstandig, o=bij ouders)
        \item Geslacht (m=man, v=vrouw)
        \item De maandelijke bestedingen aan voedsel en drank
        \item De score voor het tentamen statistiek
    \end{itemize}
    In het boek staan de resultaten in een tabel.
}

\begin{enumerate}[label=(\alph*)]
    \item Geef aan op welk type schaal de vijf variabelen worden gemeten
    \answer{Naam: nominaal, leeftijd: ratio, woonsituatie: nominaal, geslacht: nominaal, besteding: ratio, score: interval}

    \item Maak een frequentieverdeling van de leeftijden
    \answer{
        \begin{table}[h!]
            \centering
            \caption*{Aantal eerstejaarsstudenten per leeftijd}
            \begin{tabular}
            
                \toprule
                    {\bfseries Leeftijd & Frequentie} \\
                \midrule
                    18 & & 36 \\
                    19 & 5 -< 8 & 24 \\
                    20 & 8 -< 12 & 8 \\
                    21 & 12 -< 16 & 6 \\
                    22 & 16 -< 20 & 4 \\
                    $\ge 23$ & 20 en hoger & 2 \\
        
                \midrule    
                    Totaal & & $80$ \\
                \bottomrule
            \end{tab

    }
    
    
    \item de kleur van tulpen
    \item de industrietak waarin werknemers een baan hebben
    \item de jaaromzet (in euro) van bedrijven
    \item het aantal sterren dat de moeilijkheidsgraad van puzzelboekjes aangeeft
    \item de hoogte boven de zeespiegel van wintersportdorpen
\end{enumerate}
}

\answer{
    \begin{enumerate}[label=(\alph*)]
        \item Ratio
        \item Nominaal
        \item Nominaal
        \item Ratio
        \item Ordinaal
        \item Ratio
    \end{enumerate} 
}

