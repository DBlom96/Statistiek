\question{10.12}{
    Soms wil men voor een getrokken steekproef beoordelen of die als representatief mag worden beschouwd met betrekking tot een bepaald kenmerk of een bepaalde variabele.
    Met de chikwadraattoets voor aanpassing kan worden getoetst of de waargenomen verdeling in dit opzicht voldoende gelijkenis vertoont met de populatieopbouw.
    Bij een opinieonderzoek over de toekomst van Europa worden $480$ kiesgerechtigde Nederlanders ondervraagd.
    Van alle ondervraagden is het opleidingsniveau genoteerd.
    In de volgende tabel wordt dit opleidingsniveau vergeleken met de totale Nederlandse bevolking:
    \begin{center}
        \begin{tabular}{cccccc}
            \toprule
                & {\bfseries Laag} & {\bfseries Matig} & {\bfseries Redelijk} & {\bfseries Hoog} & {\bfseries Totaal} \\
            \cmidrule{1-1} \cmidrule{2-2} \cmidrule{3-3} \cmidrule{4-4} \cmidrule{5-5}
                Steekproef & $134$ & $144$ & $129$ & $73$ & $480$ \\
                Populatie & $28\%$ & $36\%$ & $24\%$ & $12\%$ & $100\%$ \\
            \bottomrule
        \end{tabular}
    \end{center}
}
Toets met $\alpha=0,05$ of de steekproef als representatief mag worden beschouwd.
\answer{

}
