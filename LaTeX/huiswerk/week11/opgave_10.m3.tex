\question{10.m3}{
    Bij een toets voor onafhankelijkheid geeft de tabel de waargenomen frequenties weer.
    De te verwachten of \emph{expected} frequentie voor de cel waarin $80$ waarnemingen vermeld zijn, bedraagt $\ldots$
}
\begin{enumerate}[label=(\alph*)]
    \item $\frac{80}{250}$
    \item $125$
    \item $0,2$
    \item $100$
\end{enumerate}

\begin{center}
    \begin{tabular}{ccc}
        \toprule
            & {\bfseries Man} & {\bfseries Vrouw} \\
        \cmidrule{1-1} \cmidrule{2-2} \cmidrule{3-3}
            Links & $80$ & $120$ \\
            Rechts & $170$ & $130$ \\
        \bottomrule
    \end{tabular}
\end{center}

\answer{
    Bekijk nu een uitgebreidere versie van de tabel van \emph{observed} frequenties hierboven, waarbij we ook de rij- en kolomtotalen hebben toegevoegd.
    \begin{center}
        \begin{tabular}{ccc|c}
            \toprule
                & {\bfseries Man} & {\bfseries Vrouw} & {\bfseries Totaal} \\
            \cmidrule{1-1} \cmidrule{2-2} \cmidrule{3-3} \cmidrule{4-4}
                Links & $80$ & $120$ & $200$ \\
                Rechts & $170$ & $130$ & $300$ \\
            \cmidrule{1-1} \cmidrule{2-2} \cmidrule{3-3} \cmidrule{4-4}
                Totaal & $250$ & $250$ & $500$ \\
            \bottomrule
        \end{tabular}
    \end{center}

    De te verwachten of \emph{expected} frequentie $E_{ij}$ voor de cel in rij $i$ en kolom $j$ is gelijk aan
    \[
        E_{ij} = \frac{\text{rijtotaal}_{i} \cdot \text{kolomtotaal}_j}{\text{totaal}}.
    \]
    Dat wil zeggen dat de expected frequentie voor de cel linksboven, oftewel $E_{11}$, gelijk is aan
    \[
        E_{11} = \frac{\text{rijtotaal}_{1} \cdot \text{kolomtotaal}_1}{\text{totaal}} = \frac{200 \cdot 250}{500} = 100.
    \]
    Het juiste antwoord is dus (d).
}
