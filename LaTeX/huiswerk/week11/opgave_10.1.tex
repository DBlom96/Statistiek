\question{10.1}{
    Bij een onderzoek naar de rookgewoonten van Nederlanders van $18$ jaar en ouder werden door loting $200$ proefpersonen gekozen die vervolgens werden ingedeeld naar leeftijd en naar rookgewoonte.
    De resultaten waren als volgt:
    \begin{center}
        \begin{tabular}{ccccc}
            \toprule
                \multicolumn{5}{c}{Leeftijd} \\
            \cmidrule{1-1} \cmidrule{2-2} \cmidrule{3-3} \cmidrule{4-4} \cmidrule{5-5}
                & {\bfseries $18-<30$} & {\bfseries $30-<45$} & {\bfseries 45 en ouder} & {\bfseries Totaal} \\
            \cmidrule{1-1} \cmidrule{2-2} \cmidrule{3-3} \cmidrule{4-4} \cmidrule{5-5}
                Roker & $25$ & $35$ & $20$ & $80$ \\
                Niet-roker & $55$ & $25$ & $40$ & $120$ \\
            \cmidrule{1-1} \cmidrule{2-2} \cmidrule{3-3} \cmidrule{4-4} \cmidrule{5-5}
                Totaal & $80$ & $60$ & $60$ & $200$ \\
            \bottomrule
        \end{tabular}
    \end{center}
    We gaan met behulp van de chikwadraattoets onderzoeken of de indelingen naar leeftijd en rookgewoonte al dan niet afhankelijk van elkaar zijn.
    We toetsen met $\alpha=0,01$.
    De nulhypothese luidt: $H_0:$ onafhankelijkheid.
}
\begin{enumerate}[label=(\alph*)]
    \item Bereken de \emph{expected}-tabel.
    \answer{
    
    }

    \item Bereken de toetsingsgrootheid $\chi^2$.
    \answer{
        
    }

    \item Hoeveel vrijheidsgraden heeft de chikwadraatverdeling die gebruikt moet worden?
    \answer{

    }

    \item Geef het kritieke gebied aan van de grootheid $\chi^2$.
    \answer{

    }

    \item Wat is uw eindconclusie?
    \answer{
        
    }
\end{enumerate}