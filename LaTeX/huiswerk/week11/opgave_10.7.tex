\question{10.7}{
    Bij een onderzoek naar de gevolgen van roken is gemeten hoe het gesteld is met de bloeddruk van $50$-jarige mannen.
    In het onderzoek waren $200$ mannen betrokken, waarvan $40$ te kwalificeren zijn als stevige roker, $30$ als gelegenheidsrokers en $130$ als niet-rokers.
    Hun bloeddruk werd een van de volgende drie kwalificaties gegeven, namelijk normaal, licht verhoogd of ernstig verhoogd.
    Dat leverde de volgende tabel:
    \begin{center}
        \begin{tabular}{cccc}
            \toprule
                {\bfseries Bloeddruk} & {\bfseries Stevige roker} & {\bfseries Matige roker} & {\bfseries Niet-roker} \\
            \cmidrule{1-1} \cmidrule{2-2} \cmidrule{3-3} \cmidrule{4-4}
                Normaal & $6$ & $8$ & $86$ \\
                Licht verhoogd & $10$ & $8$ & $22$ \\
                Ernstig verhoogd & $24$ & $14$ & $22$ \\
            \bottomrule
        \end{tabular}
    \end{center}
}
Toets of rookgedrag en bloeddrukniveau significant samenhangen (kies $\alpha=0,01$).
\answer{
    
}