\question{10.7}{
    Bij een onderzoek naar de gevolgen van roken is gemeten hoe het gesteld is met de bloeddruk van $50$-jarige mannen.
    In het onderzoek waren $200$ mannen betrokken, waarvan $40$ te kwalificeren zijn als stevige roker, $30$ als gelegenheidsrokers en $130$ als niet-rokers.
    Hun bloeddruk werd een van de volgende drie kwalificaties gegeven, namelijk normaal, licht verhoogd of ernstig verhoogd.
    Dat leverde de volgende tabel:
    \begin{center}
        \renewcommand{\arraystretch}{1.25}
        \begin{tabular}{cccc}
            \toprule
                {\bfseries Bloeddruk} & {\bfseries Stevige roker} & {\bfseries Matige roker} & {\bfseries Niet-roker} \\
            \cmidrule{1-1} \cmidrule{2-2} \cmidrule{3-3} \cmidrule{4-4}
                Normaal & $6$ & $8$ & $86$ \\
                Licht verhoogd & $10$ & $8$ & $22$ \\
                Ernstig verhoogd & $24$ & $14$ & $22$ \\
            \bottomrule
        \end{tabular}
    \end{center}
}
Toets of rookgedrag en bloeddrukniveau significant samenhangen (kies $\alpha=0,01$).
\answer{
    Omdat we weer te maken hebben met twee \emph{nominale} variabelen, namelijk ``rookgedrag'' en ``bloeddrukniveau'', kunnen we een chikwadraattoets uitvoeren om de onafhankelijkheid tussen beide variabelen te toetsen.

    Bij een chikwadraattoets voor onafhankelijkheid beginnen we met het defini\"eren van de nulhypothese $H_0$ en de alternatieve hypothese $H_1$.
    \begin{align*}
        H_0: &\qquad \text{``rookgedrag'' en ``bloeddrukniveau'' zijn onafhankelijk van elkaar.} \\
        H_1: &\qquad \text{``rookgedrag'' en ``bloeddrukniveau'' zijn afhankelijk van elkaar.}
    \end{align*}
    Het significantieniveau $\alpha=0,01$ is gegeven in de opdracht, net als de geobserveerde data.
    Om de toetsingsgrootheid $X^2$ te kunnen bepalen, moeten we eerst de \emph{expected}-tabel berekenen.
    Hiervoor starten we vanuit een lege tabel waarin alleen de totalen (van de rijen en kolommen respectievelijk) gegeven zijn:
    
    \begin{center}
        \renewcommand{\arraystretch}{1.25}
        \begin{tabular}{ccccc}
            \toprule
                {\bfseries Bloeddruk} & {\bfseries Stevige roker} & {\bfseries Matige roker} & {\bfseries Niet-roker} & {\bfseries Totaal} \\
            \cmidrule{1-1} \cmidrule{2-2} \cmidrule{3-3} \cmidrule{4-4} \cmidrule{5-5}
                Normaal & & & & $100$ \\
                Licht verhoogd & & & & $40$ \\
                Ernstig verhoogd & & & & $60$ \\
            \cmidrule{1-1} \cmidrule{2-2} \cmidrule{3-3} \cmidrule{4-4} \cmidrule{5-5}
                Totaal & $40$ & $30$ & $130$ & $200$ \\
            \bottomrule
        \end{tabular}
    \end{center}

    Voor iedere cel in de tabel bepalen we de expected frequentie met de formule:

    \[
        E_{ij} = \frac{\text{rijtotaal}_{i} \cdot \text{kolomtotaal}_{j}}{\text{totaal}}
    \]

    Dit geeft de volgende \emph{expected}-tabel:

    \begin{center}
        \renewcommand{\arraystretch}{1.5}
        \begin{tabular}{ccccc}
            \toprule
                {\bfseries Bloeddruk} & {\bfseries Stevige roker} & {\bfseries Matige roker} & {\bfseries Niet-roker} & {\bfseries Totaal} \\
            \cmidrule{1-1} \cmidrule{2-2} \cmidrule{3-3} \cmidrule{4-4} \cmidrule{5-5}
                Normaal & $\frac{40\cdot 100}{200} = 20$ & $\frac{30\cdot 100}{200} = 15$ & $\frac{130\cdot 100}{200} = 65$ & $100$ \\
                Licht verhoogd & $\frac{40\cdot 40}{200} = 8$ & $\frac{30\cdot 40}{200} = 6$ & $\frac{130\cdot 40}{200} = 26$ & $40$ \\
                Ernstig verhoogd & $\frac{40\cdot 60}{200} = 12$ & $\frac{30\cdot 60}{200} = 9$ & $\frac{130\cdot 60}{200} = 39$ & $60$ \\
            \cmidrule{1-1} \cmidrule{2-2} \cmidrule{3-3} \cmidrule{4-4} \cmidrule{5-5}
                Totaal & $40$ & $30$ & $130$ & $200$ \\
            \bottomrule
        \end{tabular}
    \end{center}

    De toetsingsgrootheid $X^2$ bepalen we aan de hand van de volgende formule:

    \begin{align*}
        X^2  &= \sum_{i,j} \frac{(O_{ij} - E_{ij})^2}{E_{ij}},
    \end{align*}

    waarbij $i = 1,2,3$ de index van de rij is, en $j = 1,2,3$ de index van de kolom is.
    Dit geeft in dit specifieke geval een geobserveerde toetsingsgrootheid 
    
    \begin{align*}
        \chi^2  &= \sum_{i,j} \frac{(O_{ij} - E_{ij})^2}{E_{ij}} \\
                &=\frac{(O_{11} - E_{11})^2}{E_{11}} + \frac{(O_{12} - E_{12})^2}{E_{12}} + \ldots + \frac{(O_{33} - E_{33})^2}{E_{33}} \\
                &= \frac{(6 - 20)^2}{20} + \frac{(8 - 15)^2}{15} + \ldots + \frac{(22-39)^2}{39}\\
                &\approx 43,8214
    \end{align*}

    Onder de nulhypothese volgt de toetsingsgrootheid $X$ een chikwadraatverdeling met
    \[
        \text{df} = (\#\text{rijen}-1) \cdot (\#\text{kolommen}-1) = (3-1)\cdot(3-1) = 4
    \]
    vrijheidsgraden.
    De $p$-waarde (rechteroverschrijdingskans) die hoort bij deze geobserveerde toetsingsgrootheid $\chi^2$ is gelijk aan
    \begin{align*}
        p   &= P(X^2 \ge \chi^2) \\
            &= \chi^2\text{cdf}(\text{lower}=\chi^2\approx43,8214; \text{upper}=10^{10}; \text{df}=4) \\
            &\approx 6,9879 \cdot 10^{-9}
    \end{align*}

    Aangezien de $p$-waarde extreem klein is, betekent dit dat de geobserveerde toetsingsgrootheid $\chi^2$ extreem hoge waarde heeft onder de aanname van onafhankelijkheid.
    Omdat de $p$-waarde veel kleiner is dan het significantieniveau $\alpha=0,01$, wordt $H_0$  verworpen.
    Er is voldoende bewijs om aan te nemen dat de twee nominale variabelen ``rookgedrag'' en ``bloeddrukniveau'' afhankelijk zijn van elkaar.
    
    \vspace{1em}

    {
        \itshape \textbf{Side note:} de toetsuitslag van het verwerpen van $H_0$ laat niet zien hoe deze afhankelijkheid eruit ziet. Als je de ruwe data bekijkt, zie je echter dat onder de niet-rokers een groot deel gewoon een normale bloeddruk heeft, terwijl voor stevige rokers de meeste juist een ernstig verhoogde bloeddruk hebben.
        Een analyse van de data is dus nodig om te bekijken wat de daadwerkelijke samenhang is tussen de variabelen.
    }
}