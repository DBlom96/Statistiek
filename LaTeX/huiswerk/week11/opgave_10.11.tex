\question{10.11}{
    In een ziekenhuis worden dagelijks vijf orthopedische operaties uitgevoerd.
    Bekend is dat deze in $20\%$ van de gevallen leiden tot complicaties waardoor de pati\"ent enige tijd moet verblijven op de afdeling intensive care.
    Voor een periode van $100$ dagen leidde dit tot de volgende aantallen verwijzingen naar de afdeling intensive care:
    \begin{center}
        \begin{tabular}{cc}
            \toprule
                {\bfseries Aantal per dag ($k$)} & {\bfseries Aantal dagen met $k$ verwijzingen} \\
            \cmidrule{1-1} \cmidrule{2-2}
                $0$ & $15$ \\
                $1$ & $25$ \\
                $2$ & $36$ \\
                $3$ & $12$ \\
                $4$ & $10$ \\
                $5$ & $2$ \\
            \cmidrule{1-1} \cmidrule{2-2}
                Totaal & $100$ dagen \\
            \bottomrule
        \end{tabular}
    \end{center}
}
Toets met $\alpha=0,05$ of de waargenomen verdeling overeenstemt met een binomiale verdeling met $p=0,20$.
\answer{
    Omdat we willen kijken of de gegeven data overeenkomen met een specifieke discrete kansverdeling, in dit geval de binomiale verdeling met $p=0,20$, kunnen we een chikwadraattoets voor aanpassing (goodness-of-fit test) uitvoeren.
    Laat $X$ een kansvariabele zijn die het aantal verwijzingen naar de afdeling intensive care telt op een willekeurige dag.
    Verder is gegeven dat er per dag $5$ orthopedische operaties zijn, die in $20\%$ van de gevallen leidt tot complicaties.
    Bij een chikwadraattoets voor aanpassing beginnen we met het defini\"eren van de nulhypothese $H_0$ en de alternatieve hypothese $H_1$.

    \begin{align*}
        H_0: &\qquad \text{$X$ is binomiaal verdeeld met parameters $n=5$ en $p=0,20$.} \\
        H_1: &\qquad \text{$X$ is NIET binomiaal verdeeld met parameters $n=5$ en $p=0,20$.}
    \end{align*}

    Het significantieniveau $\alpha=0,05$ is gegeven in de opdracht, net als de geobserveerde data.
    Om de toetsingsgrootheid $X^2$ te kunnen bepalen, moeten we eerst de \emph{expected}-tabel berekenen.
    Omdat onder de nulhypothese $H_0$ geldt dat $X \sim \text{Binomiaal}(n=5; p=0,20)$, kunnen we deze verwachte aantallen uitrekenen. 
    Omdat we $100$ afzonderlijke dagen bekijken, moet dit ook worden meegenomen in de verwachte frequenties.   
    \begin{center}
        \begin{tabular}{ccc}
            \toprule
                {\bfseries Aantal per dag ($k$)} & {\bfseries Observed} & {\bfseries Expected} \\
            \cmidrule{1-1} \cmidrule{2-2} \cmidrule{3-3}
                $0$ & $15$ & $100 \cdot \binompdf(n=5; p=0,20; k=0) = 32,768$ \\
                $1$ & $25$ & $100 \cdot \binompdf(n=5; p=0,20; k=1) = 40,96$ \\
                $2$ & $36$ & $100 \cdot \binompdf(n=5; p=0,20; k=2) = 20,48$ \\
                $3$ & $12$ & $100 \cdot \binompdf(n=5; p=0,20; k=3) = 5,12$ \\
                $4$ & $10$ & $100 \cdot \binompdf(n=5; p=0,20; k=4) = 0,64$ \\
                $5$ & $2$  & $100 - 32,768 - 40,96 - \ldots - 0,64 = 0,032$ \\
            \cmidrule{1-1} \cmidrule{2-2} \cmidrule{3-3}
                Totaal & $100$ dagen & $100$ dagen \\
            \bottomrule
        \end{tabular}
    \end{center}
    Merk op dat we de chikwadraatverdeling enkel konden gebruiken zodra de verwachte frequenties allemaal minstens $5$ zijn.
    Om dit te bereiken, moeten we categorie\"en samennemen, namelijk $3$ tot en met $5$:
    \begin{center}
        \begin{tabular}{ccc}
            \toprule
                {\bfseries Aantal per dag ($k$)} & {\bfseries Observed} & {\bfseries Expected} \\
            \cmidrule{1-1} \cmidrule{2-2} \cmidrule{3-3}
                $0$ & $15$ & $100 \cdot \binompdf(n=5; p=0,20; k=0) = 32,768$ \\
                $1$ & $25$ & $100 \cdot \binompdf(n=5; p=0,20; k=1) = 40,96$ \\
                $2$ & $36$ & $100 \cdot \binompdf(n=5; p=0,20; k=2) = 20,48$ \\
                $\ge 3$ & $24$ & $100 - 32,768 - 40,96 - 20,48 = 5,792$ \\
            \cmidrule{1-1} \cmidrule{2-2} \cmidrule{3-3}
                Totaal & $100$ dagen & $100$ dagen \\
            \bottomrule
        \end{tabular}
    \end{center}

    De toetsingsgrootheid $X^2$ bepalen we aan de hand van de volgende formule:

    \begin{align*}
        X^2  &= \sum_{i} \frac{(O_{i} - E_{i})^2}{E_{i}},
    \end{align*}

    waarbij $i = 1,2,3,4$ de index van de rij is.
    Dit geeft in dit specifieke geval een geobserveerde toetsingsgrootheid 
    
    \begin{align*}
        \chi^2  &= \sum_{i} \frac{(O_{i} - E_{i})^2}{E_{i}} \\
                &=\frac{(O_{1} - E_{1})^2}{E_{1}} + \frac{(O_{2} - E_{2})^2}{E_{2}} + \frac{(O_{3} - E_{3})^2}{E_{3}} + \frac{(O_{4} - E_{4})^2}{E_{4}} \\
                &= \frac{(15 - 32,768)^2}{32,768} + \frac{(25-40,96)^2}{40,96} + \frac{(36-20,48)^2}{20,48} + \frac{(24-5,792)^2}{5,792} \\
                &\approx 84,8540
    \end{align*}

    Onder de nulhypothese volgt de toetsingsgrootheid $X$ een chikwadraatverdeling met
    \[
        \text{df} = (\#\text{rijen}-1) = 4 - 1 = 3
    \]
    vrijheidsgraden.
    De $p$-waarde (rechteroverschrijdingskans) die hoort bij deze geobserveerde toetsingsgrootheid $\chi^2$ is gelijk aan
    \begin{align*}
        p   &= P(X^2 \ge \chi^2) \\
            &= \chi^2\text{cdf}(\text{lower}=\chi^2\approx 84,8540; \text{upper}=10^{10}; \text{df}=3) \\
            &\approx 2,7893 \cdot 10^{-18}
    \end{align*}

    Aangezien de $p$-waarde extreem klein is, betekent dit dat de geobserveerde toetsingsgrootheid $\chi^2$ extreem hoge waarde heeft onder de aanname van een binomiale verdeling met $n=5$ en $p=0,20$.
    Omdat de $p$-waarde veel kleiner is dan het significantieniveau $\alpha=0,01$, wordt $H_0$  verworpen.
    Er is voldoende bewijs om aan te nemen dat het aantal verwijzingen naar de afdeling intensive care niet binomiaal verdeeld is met parameters $n=5$ en $p=0,20$.
        
    \vspace{1em}

    {
        \itshape \textbf{Side note:} 
        eigenlijk moet je nog doordenken over wat de toetsuitslag betekent.
        Merk op dat volgens de metingen het totaal aantal doorverwijzingen in $100$ dagen gelijk is aan
        \[
            0\cdot 15 + 1 \cdot 25 + 2 \cdot 36 + 3 \cdot 12 + 4 \cdot 10 + 5 \cdot 2 = 183
        \]
        Aangezien er elke dag $5$ operaties plaatsvinden, is het totaal aantal operaties gelijk aan $500$.
        Dit betekent dat de fractie doorverwezen pati\"enten gelijk is aan $\frac{183}{500} = 0,366$.
        Deze fractie is veel groter dan $p=0,20$, dus misschien is het niet zo realistisch om te verwachten dat de nulhypothese gaat kloppen.
        Een analyse van de ruwe data is dus nodig om te bekijken wat de daadwerkelijke samenhang is tussen de variabelen.
    }
}
