\question{2.9}{Een groep van 10 studenten behaalde de volgende scores bij een examen:
    \[
        45, 60, 82, 32, 25, 75, 65, 66, 70, 80
    \]
}
\begin{enumerate}[label=(\alph*)]
    \item Bereken het rekenkundig gemiddelde, de mediaan en de standaarddeviatie voor deze gegevens.
    \answer{
        Het rekenkundig gemiddelde wordt berekend met:
        \[
        \samplemean = \frac{\sum x_i}{n} = \frac{45 + 60 + \ldots + 80}{10} = \frac{600}{10} = 60
        \]

        De mediaan wordt gevonden door eerst de scores te sorteren:
        \[
        25, 32, 45, 60, 65, 66, 70, 75, 80, 82
        \]
        Omdat er een even aantal scores is, bekijken we het gemiddelde van de middelste twee waarden:
        \[
        \text{Mediaan} = \frac{x_{5} + x_{6}}{2} = \frac{65 + 66}{2} = 65.5
        \]

        Omdat we te maken hebben met een steekproef, wordt de variantie berekend met:
        \begin{align*}
            \samplevar  &= \frac{1}{n-1}\cdot \sum (x_i - \samplemean)^2 \\
                        &= \frac{1}{10-1} \cdot \left( (45-60)^2 + (60-60)^2 + \ldots + (80-60)^2 \right) \\
                        &= \frac{1}{9} \cdot 3504 \\
                        &= \frac{3504}{9}
        \end{align*}
        De standaarddeviatie vinden we door de wortel uit de variantie te nemen:
        \[
            \samplestd = \sqrt{\samplevar} = \sqrt{\frac{3750}{9}} \approx 19.7315
        \]
    }

    \item Er wordt besloten alle studenten 10 punten extra te geven. Wat is het gevolg van deze maatregel voor de bij vraag (a) berekende maatstaven.
    \answer{
        \textbf{Rekenkundig gemiddelde:} als alle scores met 10 worden verhoogd, dan wordt het rekenkundig gemiddelde ook met 10 verhoogd.

        \textbf{Mediaan:} als alle scores met 10 worden verhoogd, dan geldt dit specifiek ook voor de middelste twee scores (merk op dat de volgorde gelijk blijft!).
        De mediaan is het gemiddelde van de twee middelste scores, dus de mediaan wordt ook verhoogd met 10

        \textbf{Standaarddeviatie}: als alle scores met 10 worden verhoogd, dan blijft de spreiding van de scores gelijk, alleen met 10 naar rechts verschoven.
        De standaarddeviatie blijft in dat geval dus hetzelfde.
    }

    \item In plaats van 10 punten voor iedereen erbij (vraag b) wordt besloten de kandidaten allemaal een 10\% hogere score te geven dan zij oorspronkelijk gehaald hebben. Wat is de invloed van deze maatregel op de onder a berekende grootheden?
    \answer{
        \textbf{Rekenkundig gemiddelde:} als alle scores met 10\% worden verhoogd, dan wordt het rekenkundig gemiddelde ook met 10\% verhoogd.

        \textbf{Mediaan:} als alle scores met 10\% worden verhoogd, dan geldt dit specifiek ook voor de middelste twee scores (merk op dat de volgorde gelijk blijft!).
        De mediaan is het gemiddelde van de twee middelste scores, dus de mediaan wordt ook verhoogd met 10\%

        \textbf{Standaarddeviatie}: als alle scores met 10\% worden verhoogd, dan wordt de spreiding van de scores groter.
        Merk op dat voor elke term $(x_i - \samplemean)^2$ in de formule van de variantie in deze nieuwe situatie moet worden vervangen door 
        \[
            (1.1\cdot x_i - 1.1 \cdot \samplemean)^2 = \left( 1.1 \cdot (x_i-\samplemean)^2\right) = 1.1^2 \cdot (x_i-\samplemean)^2 = 1.21 \cdot (x_i-\samplemean)^2
        \]
        De variantie stijgt dus met een factor $1.21$, en de standaarddeviatie met een factor $\sqrt{1.21}=1.1$ (10\%).
    }
\end{enumerate}      