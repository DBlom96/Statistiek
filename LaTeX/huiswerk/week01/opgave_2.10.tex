\question{2.10}{In een gemeente is bijgehouden hoe de prijsontwikkeling is van woningen.
Hierbij wordt een onderscheid gemaakt naar het type woning.
De volgende resultaten zijn verzameld over het afgelopen jaar.
Prijsverandering geeft het verschil aan tussen huidige verkoopprijzen en de prijzen van een jaar geleden.
}

\begin{center}
    \begin{tabular}{cccc}
        \toprule
            \textbf{Type woning} & \textbf{Aantal verkocht} & \textbf{Aantal aanwezig} & \textbf{Prijsverandering} \\
        \midrule
            Villa & $14$ & $280$ & $+5\%$ \\
            Rijtjeshuis & $130$ & $1.600$ & $+2,4\%$ \\
            Appartement & $48$ & $450$ & $+1,6\%$ \\
            Twee onder een kap & $26$ & $500$ & $+7,2\%$ \\
            Serviceflat & $28$ & $140$ & $-2,5\%$ \\
        \bottomrule
    \end{tabular}
\end{center}

\begin{enumerate}[label=(\alph*)]
    \item Bereken voor deze gemeente de gemiddelde prijsverandering van de verkochte huizen.
    \answer{
        De gemiddelde prijsverandering van de verkochte huizen wordt berekend als een gewogen gemiddelde:
        \begin{align*}
            \samplemean &= \frac{14 \cdot 5 + 130 \cdot 2,4 + 48 \cdot 1,6 + 26 \cdot 7,2 + 28 \cdot (-2,5)}{14+130+48+26+28} \\
                        &= \frac{576}{246} \\
                        &\approx 2,3415
        \end{align*}
    }

    \item Bereken op basis van de gegevens de gemiddelde waardeverandering van het totale woningbestand op basis van weging met het aantal aanwezige huizen.
    \answer{
        De gemiddelde waardeverandering van het totale woningbestand wordt berekend als een gewogen gemiddelde:
        \begin{align*}
            \samplemean &= \frac{280 \cdot 5 + 1600 \cdot 2,4 + 450 \cdot 1,6 + 500 \cdot 7,2 + 140 \cdot (-2,5)}{280+1600+450+500+140} \\
                        &= \frac{9210}{2970} \\
                        &\approx 3,1010
        \end{align*}
    }
\end{enumerate}