\question{8.m4}{
    Bij een schattingsprobleem moet men de $t$-verdeling gebruiken als voor de te onderzoeken kansvariabele geldt dat er een steekproef is genomen van:
}
\begin{enumerate}[label=(\alph*)]
    \item een normale verdeling met onbekende $\mu$ en onbekende variantie.
    \item een binomiale verdeling met onbekende succeskans $\pi$.
    \item een normale verdeling met gegeven standaarddeviatie en onbekende verwachtingswaarde.
    \item een gegeven normale verdeling met minder dan $30$ vrijheidsgraden.
\end{enumerate}
\answer{
    De $t$-verdeling moet gebruikt worden in het geval dat de te onderzoeken kansvariabele normaal verdeeld is met onbekende verwachtingswaarde $\mu$ en onbekende standaarddeviatie $\sigma$.
    Het juiste antwoord is dus (a).
}