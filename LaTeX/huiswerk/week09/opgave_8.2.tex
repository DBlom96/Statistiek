\question{8.2}{
    Van een bepaalde oplossing worden tien even grote monsters genomen.
    Van elk monster bepaalt men het aantal milligram eiwit, om hiermee het eiwitgehalte ($\mu$) van de gehele oplossing te kunnen schatten.
    De waargenomen uitkomsten zijn als volgt:
    \begin{center}
        \begin{tabular}{cccccccccc}
            \toprule
                $36,5$ & $45,2$ & $40,5$ & $42,0$ & $37,7$ & $39,4$ & $41,6$ & $38,8$ & $39,0$ & $43,3$ \\
            \bottomrule
        \end{tabular}
    \end{center}
}
\begin{enumerate}[label=(\alph*)]
    \item Een zuivere schatter voor $\mu$ is het steekproefgemiddelde $\overline{x}$.
    Bereken $\overline{x}$.
    \answer{

    }

    \item Hier kennen we $\sigma$ niet, dus voordat er een schattingsinterval kan worden berekend voor $\mu$, moet er een schatting van de variantie gemaakt worden.
    Bereken deze variantie.
    \answer{

    }

    \item Bereken voor $\mu$ een betrouwbaarheidsinterval dat een betrouwbaarheid heeft van $95\%$.
    \answer{
        
    }
\end{enumerate}
