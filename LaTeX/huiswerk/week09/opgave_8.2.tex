\question{8.2}{
    Van een bepaalde oplossing worden tien even grote monsters genomen.
    Van elk monster bepaalt men het aantal milligram eiwit, om hiermee het eiwitgehalte ($\mu$) van de gehele oplossing te kunnen schatten.
    De waargenomen uitkomsten zijn als volgt:
    \begin{center}
        \begin{tabular}{cccccccccc}
            \toprule
                $36,5$ & $45,2$ & $40,5$ & $42,0$ & $37,7$ & $39,4$ & $41,6$ & $38,8$ & $39,0$ & $43,3$ \\
            \bottomrule
        \end{tabular}
    \end{center}
}
\begin{enumerate}[label=(\alph*)]
    \item Een zuivere schatter voor $\mu$ is het steekproefgemiddelde $\overline{x}$.
    Bereken $\overline{x}$.
    \answer{
        Het steekproefgemiddelde $\overline{x}$ berekenen we als volgt:
        \begin{align*}
            \overline{x} = \frac{x_1+x_2+\ldots+x_{10}}{10} = \frac{36,5+45,2+\ldots+43,3}{10} = 40,4
        \end{align*}
        Gemiddeld zit er $40,4$ mg eiwit in een monster.
    }

    \item Hier kennen we $\sigma$ niet, dus voordat er een schattingsinterval kan worden berekend voor $\mu$, moet er een schatting van de variantie gemaakt worden.
    Bereken deze variantie.
    \answer{
        De steekproefvariantie $s^2$ berekenen we als volgt:
        \begin{align*}
            s^2 &= \frac{(x_1-\overline{x})^2+(x_2-\overline{x})^2+\ldots+(x_{10}-\overline{x})^2}{10-1} \\
                &= \frac{(36,5-40,4)^2+(45,2-40,4)^2+\ldots+(43,3-40,4)^2}{10-1} \\
                &\approx 7,0533
        \end{align*}
        Gemiddeld zit er $40,4$ mg eiwit in een monster.
    }

    \item Bereken voor $\mu$ een betrouwbaarheidsinterval dat een betrouwbaarheid heeft van $95\%$.
    \answer{
        
    }
\end{enumerate}
