\question{8.2}{
    Van een bepaalde oplossing worden tien even grote monsters genomen.
    Van elk monster bepaalt men het aantal milligram eiwit, om hiermee het eiwitgehalte ($\mu$) van de gehele oplossing te kunnen schatten.
    De waargenomen uitkomsten zijn als volgt:
    \begin{center}
        \begin{tabular}{cccccccccc}
            \toprule
                $36,5$ & $45,2$ & $40,5$ & $42,0$ & $37,7$ & $39,4$ & $41,6$ & $38,8$ & $39,0$ & $43,3$ \\
            \bottomrule
        \end{tabular}
    \end{center}
}
\begin{enumerate}[label=(\alph*)]
    \item Een zuivere schatter voor $\mu$ is het steekproefgemiddelde $\overline{x}$.
    Bereken $\overline{x}$.
    \answer{
        Het steekproefgemiddelde $\overline{x}$ berekenen we als volgt:
        \begin{align*}
            \overline{x} = \frac{x_1+x_2+\ldots+x_{10}}{10} = \frac{36,5+45,2+\ldots+43,3}{10} = 40,4
        \end{align*}
        Gemiddeld zit er $40,4$ mg eiwit in een monster.
    }

    \item Hier kennen we $\sigma$ niet, dus voordat er een schattingsinterval kan worden berekend voor $\mu$, moet er een schatting van de variantie gemaakt worden.
    Bereken deze variantie.
    \answer{
        De steekproefvariantie $s^2$ berekenen we als volgt:
        \begin{align*}
            s^2 &= \frac{(x_1-\overline{x})^2+(x_2-\overline{x})^2+\ldots+(x_{10}-\overline{x})^2}{10-1} \\
                &= \frac{(36,5-40,4)^2+(45,2-40,4)^2+\ldots+(43,3-40,4)^2}{10-1} \\
                &\approx 7,0533
        \end{align*}
        Gemiddeld zit er $40,4$ mg eiwit in een monster.
    }

    \item Bereken voor $\mu$ een betrouwbaarheidsinterval dat een betrouwbaarheid heeft van $95\%$.
    \answer{
        Noteer met $X$ de kansvariabele die het aantal milligram eiwit in een willekeurig monster meet.
        Gegeven is dat $X \sim N(\mu=?; \sigma=?)$.
        Volgens de centrale limietstelling geldt dat het gemiddelde aantal milligram eiwit $\overline{X}$ in $10$ willekeurige monsters
        normaal verdeeld is met verwachtingswaarde $\mu$ en een standaarddeviatie $\frac{\sigma}{\sqrt{n}} = \frac{\sigma}{\sqrt{10}}$.
        
        We hebben een geobserveerde steekproef van $10$ zakken genomen met steekproefgemiddelde $\overline{x} = 40,4$ milligram en steekproefstandaardafwijking $s = \sqrt{s^2} = \sqrt{7,0533} \approx 2,6558$ milligram. 
        Aangezien we een betrouwbaarheid van $95\%$ willen, is $\alpha = 0,05$.
        Verder, omdat de standaardafwijking onbekend is en de steekproefomvang $n < 30$, moeten we de $t$-verdeling met $df=n-1$ vrijheidsgraden gebruiken.
        In het bijzonder gebruiken we
        \[
            t = \invt(opp=1-\alpha/2; df=n-1) = \invt(opp=0,975; df=9) \approx 2,2622.
        \]
        Het $95\%$-betrouwbaarheidsinterval voor $\mu$ is dan gelijk aan 
        \begin{align*}
            &[\overline{x} - t \cdot \frac{s}{\sqrt{n}}; \overline{x} + t \cdot \frac{s}{\sqrt{n}}] \\
            &=[40,4 - 2,2622 \cdot \frac{2,6558}{\sqrt{10}}; 20142 + 2,2622 \cdot \frac{2,6558}{\sqrt{10}}] \\
            &=[38,50; 42,30]
        \end{align*}
        Met $95\%$ betrouwbaarheid ligt de gemiddelde hoeveelheid eiwit in een monster tussen $38,50$ en $42,30$ milligram.
    }
\end{enumerate}
