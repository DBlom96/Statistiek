\question{8.19}{
    Een technicus van een bedrijf dat oefenapparaten ontwikkelt voor fitnesstrainingen heeft een nieuw ontwerp gemaakt van een bepaalde machine.
    Bedoeling is dat dit apparaat spierversterkend werkt voor personen met rugklachten.
    Helaas was maar \'e\'en prototype van het apparaat beschikbaar om het te kunnen testen.
    Hierdoor konden slechts de gegevens van twaalf pati\"enten worden verzameld.
    Deze twaalf pati\"enten gingen oefenen met dit apparaat en het aantal dagen werd opgetekend dat nodig was om volledig het gewenste resultaat te behalen.
    Van de twaalf pati\"enten werden de volgende aantallen dagen waargenomen:
    \[
        15, 18, 41, 27, 23, 16, 29, 30, 32, 22, 25, 22.
    \]
}
\begin{enumerate}[label=(\alph*)]
    \item Schat de standaarddeviatie.
    \answer{
        We berekenen een schatting $s$ voor de standaarddeviatie door allereerst het steekproefgemiddelde te bepalen:
        \[
            \overline{x} = \frac{x_1 + x_2 + \ldots + x_n}{n} = \frac{x_1 + x_2 + \ldots + x_{12}}{12} = \frac{15+18+\ldots+22}{12} = 25.
        \]

        Nu kunnen we de steekproefvariantie bepalen:
        \begin{align*}
            s^2 &= \frac{(x_1-\overline{x})^2 + \ldots + (x_n-\overline{x})^2}{n-1} \\
                &= \frac{(x_1-\overline{x})^2 + \ldots + (x_{12}-\overline{x})^2}{12-1} \\      
                &= \frac{(15-25)^2 + \ldots + (22-25)^2}{11} \\
                &= 54.7272
        \end{align*}
        De puntschatting $s$ voor de standaarddeviatie vinden we door de wortel hiervan te nemen:
        \[
            s = \sqrt{54,7272} \approx 7,3978.
        \]
    }

    \item Bereken een $95\%$-betrouwbaarheidsinterval voor $\mu$: de gemiddelde hersteltijd voor de populatie.
    \answer{
        Gegeven is een steekproef van $n=12$ pati\"enten, waarvoor het aantal dagen tot het gewenste resutaat het steekproefgemiddelde $\overline{x} = 25$ dagen en de steekproefstandaarddeviatie $s = 7,3978$ dagen is bepaald.
        Omdat de steekproefomvang $n = 12 < 30$ is en de standaarddeviatie $\sigma$ onbekend is, moeten we de $t$-verdeling gebruiken.
        Omdat het gewenste betrouwbaarheidsniveau $95\%$ is, geldt dat $\alpha=0,05$.

        De bijbehorende $t$-waarde is gelijk aan 
        \[
            t = \invt(opp=1-\alpha/2; df=n-1) = \invt(opp=0,975; df=11) \approx 2,2010.
        \]
        Het $95\%$-betrouwbaarheidsinterval is gelijk aan
        \begin{align*}
            &[\overline{x} - t \cdot \frac{s}{\sqrt{n}}; \overline{x} + t \cdot \frac{s}{\sqrt{n}}] \\
            &=[25 - 2,2010 \cdot \frac{7,3978}{\sqrt{12}}; 25 + 2,2010 \cdot \frac{7,3978}{\sqrt{12}}] \\
            &=[20,2997; 29,7003] \\
        \end{align*}
        Met $95\%$ betrouwbaarheid zit het gemiddelde aantal dagen tot het gewenste resultaat tussen $20$ en $30$ dagen.
    }

    \item In een vervolgstudie werden $90$ personen getest.
    Er werd een standaarddeviatie gevonden van $6,5$ dagen en een gemiddelde hersteltijd van $27$ dagen.
    Bereken opnieuw een $95\%$-betrouwbaarheidsinterval voor $\mu$.
    \answer{
        Gegeven is nu een steekproef van $n=90$ pati\"enten, waarvoor het aantal dagen tot het gewenste resutaat het steekproefgemiddelde $\overline{x} = 27$ dagen en de steekproefstandaarddeviatie $s = 6,5$ dagen is bepaald.
        Omdat nu geldt dat de steekproefomvang $n = 90 > 30$ is, mogen we wel de normale verdeling gebruiken en $\sigma$ schatten met $s$.
        Omdat het gewenste betrouwbaarheidsniveau $95\%$ is, geldt dat $\alpha=0,05$.

        De bijbehorende $z$-waarde is gelijk aan 
        \[
            z = z_{\alpha/2} = \invnorm(opp=1-\alpha/2) = \invnorm(0,975) \approx 1,9600.
        \]
        Het $95\%$-betrouwbaarheidsinterval is gelijk aan
        \begin{align*}
            &[\overline{x} - z \cdot \frac{s}{\sqrt{n}}; \overline{x} + z \cdot \frac{s}{\sqrt{n}}] \\
            &=[27 - 1,9600 \cdot \frac{6,5}{\sqrt{90}}; 27 + 1,9600 \cdot \frac{6,5}{\sqrt{90}}] \\
            &=[25,6571; 28,3429] \\
        \end{align*}
        Met $95\%$ betrouwbaarheid zit het gemiddelde aantal dagen tot het gewenste resultaat tussen $25$ en $28$ dagen.
        Merk op dat dit betrouwbaarheidsinterval al een stuk kleiner is, wat het effect van een grotere steekproefomvang onderschrijft!
    }

    \item Men wenst het bij vraag \textbf{c} gevraagde interval zodanig te bepalen dat het resulterende interval voor $\mu$ niet breder is dan twee dagen.
    Hoe groot moet de steekproefomvang worden gekozen om dat te bereiken, als de standaarddeviatie van vraag \textbf{a} als vertrekpunt mag worden gekozen?
    \answer{
        Als het resulterende interval voor $\mu$ niet breder is dan twee dagen, dan is ook de foutmarge (afwijking van het gemiddelde) $a$ niet groter dan 1 dag ($a = 1$).
        Verder mogen we aannemen dat $\sigma = 7,3978$.
        In formulevorm ziet dit er als volgt uit:
        \[
            z_{\alpha/2} \cdot \frac{\sigma}{\sqrt{n}} \le a = 1,
        \]
        oftewel
        \[
            n \ge \left(\frac{z_{\alpha/2} \cdot \sigma}{a} \right)^2 = \left(\frac{1,9600 \cdot 7,3978}{1} \right)^2 \approx 210,2410
        \]
        Afgerond naar boven naar het eerstvolgende gehele getal geeft een minimale steekproefomvang van $n=211$ personen om een interval te krijgen voor $\mu$ van maximaal $2$ dagen breed.
    }
\end{enumerate}