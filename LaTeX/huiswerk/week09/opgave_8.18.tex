\question{8.18}{
    De Rijksdienst voor het Wegverkeer doet onderzoek naar het percentage auto's dat geregistreerd staat zonder dat er een geldige WA-verzekering is afgesloten.    
}
\begin{enumerate}[label=(\alph*)]
    \item Voor een steekproef van $1000$ kentekens werd nauwkeurig nagetrokken of op de auto een WA-verzekering was afgesloten.
    Er bleken $80$ auto's niet verzekerd te zijn.
    Geef een $95\%$-betrouwbaarheidsinterval voor de fractie onverzekerde auto's in de gehele populatie.
    \answer{

    }

    \item Stel dat de Rijksdienst het percentage onverzekerde auto's wil schatten met een marge van plus of min $1\%$.
    Hoeveel auto's moeten dan voor een steekproef worden opgenomen?
    Geef aan welke veronderstellingen u hanteert.
    \answer{
        
    }

    \item Voor een steekproef van $25$ onverzekerde auto's werd de ouderdom bepaald.
    Dat leverde een gemiddelde van $8,4$ jaar en een standaarddeviatie van $1,7$ jaar.
    Geef een $95\%$-betrouwbaarheidsinterval voor de gemiddelde leeftijd van alle onverzekerde auto's.
    \answer{
        
    }
\end{enumerate}