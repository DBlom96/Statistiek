\question{8.m6}{
    Men wil een $99\%$-betrouwbaarheidsinterval voor $\mu$ berekenen voor een normale verdeling met een onbekende $\sigma$.
    Er worden acht waarnemingen gedaan.
    Het interval wordt dan berekend met $\ldots$
}
\begin{enumerate}[label=(\alph*)]
    \item $z=2,58$
    \item $t=3,499$
    \item $t=2,998$
    \item $t=2,306$
\end{enumerate}
\answer{
    Gegeven is dat we te maken hebben met een kansvariabele $X \sim N(\mu=?; \sigma=?)$ en een steekproefomvang $n=8$.
    Omdat de gewenste betrouwbaarheid gelijk is aan $99\%$, maken we gebruik van $\alpha=0,01$ en de $t$-verdeling met $df=n-1=7$ vrijheidsgraden.
    \[
        t = \invt(opp=1-\alpha/2; df=n-1)=\invt(opp=0,995; df=7) \approx 3,4995.
    \]
    Het juiste antwoord is dus (b).
}