\question{8.14}{
    Een handelsmaatschappij onderzoekt hoe groot de fractie orders is die binnen de afgesproken levertijd worden uitgevoerd.
    Van $200$ orders bleken er $180$ binnen de levertijd uitgevoerd te zijn.
}
\begin{enumerate}[label=(\alph*)]
    \item Geef een $95\%$-betrouwbaarheidsinterval voor de fractie orders die op tijd zijn uitgevoerd.
    \answer{
        Laat $X$ de kansvariabele zijn die het aantal orders telt die op tijd zijn uitgevoerd.
        Omdat iedere order onafhankelijk van elkaar wordt uitgevoerd, hebben we te maken met een binomiale verdeling.
        Op basis van de steekproef geldt dat $X \sim \text{Binomiaal}(n=200; p=?)$.
        
        We willen een $95\%$-betrouwbaarheidsinterval voor $p$ bepalen.
        Dit doen we met behulp van de Clopper-Pearson methode.
        Omdat de gewenste betrouwbaarheid $95\%$ is, geldt dat $\alpha = 0,05$.

        De Clopper-Pearson methode werkt als volgt:
        \begin{enumerate}
            \item Bepaal de succeskans $p_1$ waarvoor geldt dat de linkeroverschrijdingskans van de uitkomst $k=180$ gelijk is aan $\alpha/2$, oftewel $P(X \le 180) = \alpha/2 = 0,025$.
            Voer hiervoor in het functiescherm van de grafische rekenmachine:
            \begin{align*}
                y_1 &= \binomcdf(n=200; p_1=X; k=180) \\
                y_2 &= 0,025
            \end{align*}
            De solver optie geeft een waarde van $p_1 \approx 0,9378$.
            \item Bepaal de succeskans $p_2$ waarvoor geldt dat de rechteroverschrijdingskans van de uitkomst $k=180$ gelijk is aan $\alpha/2$, oftewel $P(X \ge 180) = 1 - P(X \le 179) = \alpha/2 = 0,025$.
            Voer hiervoor in het functiescherm van de grafische rekenmachine:
            \begin{align*}
                y_1 &= 1 - \binomcdf(n=200; p_1=X; k=179) \\
                y_2 &= 0,025
            \end{align*}
            De solver optie geeft een waarde van $p_2 \approx 0,8498$.
        \end{enumerate}
        We vinden het Clopper-Pearson interval door de twee gevonden waarden als grenzen te nemen.
        Het $95\%$-betrouwbaarheidsinterval voor de fractie orders die op tijd wordt uitgevoerd is dus gelijk aan $[0,8498; 0,9378]$.
    }

    \item Hoeveel orders moet worden onderzocht om een interval voor $p$ te krijgen dat hoogstens $0,02$ breed is? (We gaan ervan uit dat $p$ ongeveer $0,9$ is).
    \answer{
        Deze vraag is buiten de scope van het vak.
    }
\end{enumerate}