\question{8.17}{
    De handvatten van koffers moeten van zodanige kwaliteit zijn dat ze niet breken bij een forse belasting van de koffers.
    Ter controle van de kwaliteit van de handvatten werden er proeven genomen waarbij het gewicht dat aan een handvat hing net zolang werd opgevoerd tot het handvat brak.
    Er werden negen handvatten getest waarbij de maximale belasting werd vastgesteld.
    De resultaten staan in de volgende tabel:

    \begin{center}
        \begin{tabular}{cccccccccc}
            \toprule
                {\bfseries Handvat} & {\bfseries A} & {\bfseries B} & {\bfseries C} & {\bfseries D} & {\bfseries E} & {\bfseries F} & {\bfseries G} & {\bfseries H} & {\bfseries I} \\
            \cmidrule{1-1} \cmidrule{2-2} \cmidrule{3-3} \cmidrule{4-4} \cmidrule{5-5} \cmidrule{6-6} \cmidrule{7-7} \cmidrule{8-8} \cmidrule{9-9} \cmidrule{10-10}
                Breuk bij een gewicht (in kg) van & $84$ & $87$ & $81$ & $85$ & $90$ & $93$ & $86$ & $88$ & $80$ \\ 
            \bottomrule
        \end{tabular}
    \end{center}
}
\begin{enumerate}[label=(\alph*)]
    \item Geef een schatting van de variantie.
    \answer{
        We berekenen een schatting $s^2$ voor de variantie door allereerst het steekproefgemiddelde te bepalen:
        \[
            \overline{x} = \frac{x_1 + x_2 + \ldots + x_n}{n} = \frac{x_1 + x_2 + \ldots + x_9}{9} = \frac{84+87+\ldots+90}{9} = 86.
        \]

        Nu kunnen we de steekproefvariantie bepalen:
        \begin{align*}
            s^2 &= \frac{(x_1-\overline{x})^2 + \ldots + (x_n-\overline{x})^2}{n-1} \\
                &= \frac{(x_1-\overline{x})^2 + \ldots + (x_9-\overline{x})^2}{9-1} \\      
                &= \frac{(84-86)^2 + \ldots + (80-86)^2}{8} \\
                &= 17,0
        \end{align*}
        De puntschatting voor de variantie $\sigma^2$ is dus gelijk aan $s^2=17$.
    }

    \item Geef een betrouwbaarheidsinterval voor $\mu_x$: de verwachtingswaarde van de maximumbelasting van een handvat.
    Kies een $99\%$-betrouwbaarheidsinterval.
    \answer{
        Omdat we hier te maken hebben met een onbekende standaardafwijking $\sigma$ en een kleine steekproefomvang $n=9 < 30$, moeten we de $t$-verdeling gebruiken.
        Gegeven is een betrouwbaarheidsniveau van $99\%$, oftewel $\alpha = 0,01$.
        De $t$-waarde die hierbij hoort is gelijk aan
        \[
            t = \invt(opp=1-\alpha/2; df=n-1) = \invt(opp=0,995; df=8) \approx 3,3554.
        \]
        Het $99\%$-betrouwbaarheidsinterval voor $\mu_x$ is in dit geval gelijk aan
        \begin{align*}
            &[\overline{x} - t \cdot \frac{s}{\sqrt{n}}; \overline{x} + t \cdot \frac{s}{\sqrt{n}}] \\
            &=[86 - 3,3554 \cdot \frac{17}{\sqrt{9}}; 86 + 3,3554 \cdot \frac{17}{\sqrt{9}}] \\
            &=[66,9861; 105,0139] \\
        \end{align*}
        Met $99\%$ betrouwbaarheid zit de gemiddelde maximumbelasting van een handvat tussen ongeveer $67$ en $105$ kg.
    }
\end{enumerate}