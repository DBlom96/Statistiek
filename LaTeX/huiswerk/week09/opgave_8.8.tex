\question{8.8}{
    In een fabriek maakt men badkamertegels.
    Bekend is dat het productieproces $20\%$ ondeugdelijke tegels oplevert.
    Na wat wijzigingen in het productieproces, werd bijgehouden hoe groot het aantal afgekeurde tegels is in de nieuwe situatie.
    Men vond de volgende aantallen:
    \begin{center}
        \begin{tabular}{ccc}
            \toprule
                {\bfseries Dag} & {\bfseries Productieomvang} & {\bfseries Aantal afgekeurd} \\
            \cmidrule{1-1} \cmidrule{2-2} \cmidrule{3-3}
                $1$ & $1200$ & $120$ \\
                $2$ & $1800$ & $210$ \\
                $3$ & $1000$ & $80$ \\
                $4$ & $3000$ & $450$ \\
                $5$ & $1500$ & $120$ \\
                $6$ & $1700$ & $170$ \\
                $7$ & $2000$ & $220$ \\
                $8$ & $1400$ & $80$ \\
                $9$ & $1800$ & $200$ \\
                $10$ & $1500$ & $130$ \\
            \bottomrule
        \end{tabular}
    \end{center}
}
\begin{enumerate}[label=(\alph*)]
    \item Stel dat alleen de gegevens van dag $1$ bekend zouden zijn.
    Wat is op basis daarvan een $95\%$-betrouwbaarheidsinterval voor $\pi$: de populatiefractie van afgekeurde tegels?
    \answer{

    }

    \item Bereken voor iedere dag de fractie afgekeurde tegels.
    Bereken op basis van deze dagelijkse steekproeffracties een interval van $\pi_{D}$: de gemiddelde dagelijkse fractie afgekeurde tegels.
    \answer{

    }

    \item Bereken de totalen over de tien dagen van productieomvang en aantal afgekeurde tegels.
    Bereken op basis hiervan een interval voor $\pi$.
    Wat is de betekenis van deze $\pi$?
    \answer{
        
    }
\end{enumerate}
