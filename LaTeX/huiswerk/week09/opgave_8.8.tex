\question{8.8}{
    In een fabriek maakt men badkamertegels.
    Bekend is dat het productieproces $20\%$ ondeugdelijke tegels oplevert.
    Na wat wijzigingen in het productieproces, werd bijgehouden hoe groot het aantal afgekeurde tegels is in de nieuwe situatie.
    Men vond de volgende aantallen:
    \begin{center}
        \begin{tabular}{ccc}
            \toprule
                {\bfseries Dag} & {\bfseries Productieomvang} & {\bfseries Aantal afgekeurd} \\
            \cmidrule{1-1} \cmidrule{2-2} \cmidrule{3-3}
                $1$ & $1200$ & $120$ \\
                $2$ & $1800$ & $210$ \\
                $3$ & $1000$ & $80$ \\
                $4$ & $3000$ & $450$ \\
                $5$ & $1500$ & $120$ \\
                $6$ & $1700$ & $170$ \\
                $7$ & $2000$ & $220$ \\
                $8$ & $1400$ & $80$ \\
                $9$ & $1800$ & $200$ \\
                $10$ & $1500$ & $130$ \\
            \bottomrule
        \end{tabular}
    \end{center}
}
\begin{enumerate}[label=(\alph*)]
    \item Stel dat alleen de gegevens van dag $1$ bekend zouden zijn.
    Wat is op basis daarvan een $95\%$-betrouwbaarheidsinterval voor $\pi$: de populatiefractie van afgekeurde tegels?
    \answer{
        Laat $X$ de kansvariabele zijn die het aantal afgekeurde tegels telt op een bepaalde dag.
        Omdat iedere tegel onafhankelijk van elkaar ondeugdelijk is of niet, hebben we te maken met een binomiale verdeling.
        Op dag $1$ is er een productieomvang  van $1200$, waarvan er $120$ zijn afgekeurd.
        Op basis hiervan geldt dat $X \sim \text{Binomiaal}(n=1200; p=?)$.
        
        We willen een $95\%$-betrouwbaarheidsinterval voor $p$ bepalen.
        Dit doen we met behulp van de Clopper-Pearson methode.
        Omdat de gewenste betrouwbaarheid $95\%$ is, geldt dat $\alpha = 0,05$.

        De Clopper-Pearson methode werkt als volgt:
        \begin{enumerate}
            \item Bepaal de succeskans $p_1$ waarvoor geldt dat de linkeroverschrijdingskans van de uitkomst $k=120$ gelijk is aan $\alpha/2$, oftewel $P(X \le 120) = \alpha/2 = 0,025$.
            Voer hiervoor in het functiescherm van de grafische rekenmachine:
            \begin{align*}
                y_1 &= \binomcdf(n=1200; p_1=X; k=120) \\
                y_2 &= 0,025
            \end{align*}
            De solver optie geeft een waarde van $p_1 \approx 0,1184$.
            \item Bepaal de succeskans $p_2$ waarvoor geldt dat de rechteroverschrijdingskans van de uitkomst $k=120$ gelijk is aan $\alpha/2$, oftewel $P(X \ge 120) = 1 - P(X \le 119) = \alpha/2 = 0,005$.
            Voer hiervoor in het functiescherm van de grafische rekenmachine:
            \begin{align*}
                y_1 &= 1 - \binomcdf(n=1200; p_1=X; k=119) \\
                y_2 &= 0,025
            \end{align*}
            De solver optie geeft een waarde van $p_2 \approx 0,0836$.
        \end{enumerate}
        We vinden het Clopper-Pearson interval door de twee gevonden waarden als grenzen te nemen.
        Het $95\%$-betrouwbaarheidsinterval voor de fractie ondeugdelijke tegels is dus gelijk aan $[0,0836; 0,1184]$.        
    }

    \item Bereken voor iedere dag de fractie afgekeurde tegels.
    Bereken op basis van deze dagelijkse steekproeffracties een interval van $\pi_{D}$: de gemiddelde dagelijkse fractie afgekeurde tegels.
    \answer{
        Deze vraag is niet van toepassing
    }

    \item Bereken de totalen over de tien dagen van productieomvang en aantal afgekeurde tegels.
    Bereken op basis hiervan een interval voor $\pi$.
    Wat is de betekenis van deze $\pi$?
    \answer{
        Laat $X$ nu de kansvariabele zijn die het aantal afgekeurde tegels telt over alle tien de dagen.
        Omdat iedere tegel onafhankelijk van elkaar ondeugdelijk is of niet, hebben we weer te maken met een binomiale verdeling.
        De totale productieomvang is  $16900$ geproduceerde tegels, waarvan $1780$ afgekeurd.
        Op basis hiervan geldt dat $X \sim \text{Binomiaal}(n=16900; p=?)$.
        
        We willen een $95\%$-betrouwbaarheidsinterval voor $p$ bepalen.
        Dit doen we met behulp van de Clopper-Pearson methode.
        Omdat de gewenste betrouwbaarheid $95\%$ is, geldt dat $\alpha = 0,05$.

        De Clopper-Pearson methode werkt als volgt:
        \begin{enumerate}
            \item Bepaal de succeskans $p_1$ waarvoor geldt dat de linkeroverschrijdingskans van de uitkomst $k=1780$ gelijk is aan $\alpha/2$, oftewel $P(X \le 1780) = \alpha/2 = 0,025$.
            Voer hiervoor in het functiescherm van de grafische rekenmachine:
            \begin{align*}
                y_1 &= \binomcdf(n=16900; p_1=X; k=1780) \\
                y_2 &= 0,025
            \end{align*}
            De solver optie geeft een waarde van $p_1 \approx 0,1101$.
            \item Bepaal de succeskans $p_2$ waarvoor geldt dat de rechteroverschrijdingskans van de uitkomst $k=1780$ gelijk is aan $\alpha/2$, oftewel $P(X \ge 1780) = 1 - P(X \le 1779) = \alpha/2 = 0,005$.
            Voer hiervoor in het functiescherm van de grafische rekenmachine:
            \begin{align*}
                y_1 &= 1 - \binomcdf(n=16900; p_1=X; k=1779) \\
                y_2 &= 0,025
            \end{align*}
            De solver optie geeft een waarde van $p_2 \approx 0,1007$.
        \end{enumerate}
        We vinden het Clopper-Pearson interval door de twee gevonden waarden als grenzen te nemen.
        Het $95\%$-betrouwbaarheidsinterval voor de fractie afgekeurde tegels is dus gelijk aan $[0,1007; 0,1101]$.

        Merk op dat dit interval al een stuk smaller is dan het interval dat we bij \textbf{a} hebben berekend.
        Dit komt doordat we nu een veel grotere steekproefomvang hebben waarop het interval is gebaseerd.
    }
\end{enumerate}
