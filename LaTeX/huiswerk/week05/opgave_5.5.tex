\question{5.5}{Een garagebedrijf bestudeert de tijdsduur waarin de jaarlijkse servicebeurt van een auto kan worden uitgevoerd.
Deze kan worden beschouwd als een normaal verdeelde variabele $X$ met $\mu=120$ minuten en $\sigma=20$ minuten.}

\begin{enumerate}[label=(\alph*)]
    \item Hoe groot is de kans dat een servicebeurt meer dan 150 minuten vergt?
    \answer{

    }

    \item Een klant komt zijn auto brengen voor service.
    De werkzaamheden gaan onmiddellijk beginnen.
    De garagehouder zegt tegen de klant: ``Komt u over $X$ minuten maar terug.''.
    Hoe groot moet $X$ worden gekozen als we eisen dat de klant minder dan $5\%$ kans wil hebben dat hij nog moet wachten als hij zich na $X$ minuten weer meldt bij de garage?
    En hoe zit dat bij $1\%$ kans?
    \answer{

    }
\end{enumerate}
