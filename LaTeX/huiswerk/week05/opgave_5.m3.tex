\question{5.m3}{Voor de populatie van mannelijke economiestudenten is bekend dat het lichaamsgewicht kan worden beschreven door middel van een normaal verdeelde kansvariabele $X$ met $\mu=75$ kilogram en een standaarddeviatie van $6$ kilogram.
Hieruit volgt dat het percentage studenten van deze populatie dat een gewicht van meer dan $85$ kilogram heeft ongeveer bedraagt:}
\begin{enumerate}[label=(\alph*)]
    \item $95,2$
    \item $66$
    \item $34$
    \item $4,8$
\end{enumerate}
\answer{Om dit percentage te berekenen, berekenen we eerst de $z$-score behorende bij $x = 85$:
\[
    z = \frac{x-\mu}{\sigma} = \frac{85-75}{6} \approx 1,6667
\]

maken we gebruik van de functie \text{normalcdf} op de grafische rekenmachine.
In het bijzonder geldt namelijk dat de kans op een gewicht van meer dan $85$ kilogram gelijk is aan:
\begin{align*}
    P(X > 85)   &= P(Z > 1,6667) \\
                &= \text{normalcdf}(a=1,6667; b=10^{10}; \mu=0; \sigma=1) \\
                &\approx 0,0478.
\end{align*}
Met ongeveer $4,78\%$ kans heeft een economiestudent een gewicht van meer dan $85$ kilogram. Het juiste antwoord is dus (d).
}