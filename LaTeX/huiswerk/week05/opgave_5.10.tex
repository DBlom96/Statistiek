\question{5.10}{In een kliniek wordt van een groep $50$-plussers bloed afgenomen voor de bepaling van het cholesterolgehalte.
Het is een bekend gegeven dat de hoeveelheid cholesterol per buisje bloed kan worden weergegeven door een normaal verdeelde variabele $X$ met $\mu=5,20$ mg en $\sigma=0,9$ mg.
Het $98$-ste percentiel van een verdeling is het punt op de $x$-as waarboven zich $2\%$ van de waarnemingen bevindt, dus waaronder $98\%$ van de waarnemingen ligt.
}

\begin{enumerate}[label=(\alph*)]
    \item Welke $x$-waarde geeft hier het $98$-ste percentiel aan?
    \answer{

    }

    \item Hoeveel standaarddeviaties ligt dat punt verwijderd van het gemiddelde van $5,20$ mg.
    \answer{

    }

    \item In een gemeente wordt een bevolkingsonderzoek gehouden naar het cholesterolgehalte.
    Hierbij waren $8 500$ personen betrokken.
    Mensen met een cholesterolwaarde boven het $98$-ste percentiel worden voor nader onderzoek opgeroepen.
    Hoeveel personen krijgt naar verwachting een oproep?
    \answer{

    }

    \item Voor een willekeurig persoon uit de groep $50$-plussers bepalen we het cholesterolgehalte.
    Noem deze uitkomst $y$.
    Wat is het twee-sigmagebied voor $Y$? en het drie-sigmagebied?
    \answer{

    }
\end{enumerate}