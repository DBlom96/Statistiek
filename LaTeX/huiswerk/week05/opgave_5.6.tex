\question{5.6}{De gewichten van appels uit een grote partij blijken normaal verdeeld te zijn met $\mu=100$ gram en $\sigma=20$ gram.
We willen deze appels in vijf gewichtsklassen verdelen die allemaal evenveel appels bevatten.}

\begin{enumerate}[label=(\alph*)]
    \item Wat is de klassengrens van de $20\%$ appels met het geringste gewicht?
    \answer{
        Laat $X$ de normaal verdeelde kansvariabele zijn die het gewicht van een appel uit een grote partij meet.
        Er geldt dus $X \sim N(\mu=100; \sigma=20)$.
        Je wilt de appels opdelen in vijf gewichtsklassen die elk evenveel appels bevatten, d.w.z. dat de kans dat een appel in een bepaalde gewichtsklasse zit moet gelijk zijn voor elke gewichtsklasse. 
        Er zijn vijf klassen dus die kans is telkens $\frac{1}{5}=0,2$.
        De klasse met de $20\%$ lichtste appels loopt van $-\infty$ tot $g_1$ voor een bepaald gewicht $g_1$ en er geldt $P(X<g_1)=0,2$.
        Dat betekent dus dat
        \[
            g_1 = \invnorm(opp=0,2; \mu=100; \sigma=20) \approx 83,1676 \text{ g}.
        \]
    }

    \item Bepaal ook de andere klassengrenzen?
    \answer{
        Om de overige klassengrenzen te bepalen, gebruiken we soortgelijke berekeningen.
        Voor de tweede klassegrens $g_2$ geldt 
        \[
            P(g_1 < X < g_2) = 0,2
        \]
        Omdat voor $g_1$ geldt dat $P(X < g_1) = 0,2$, geldt nu dat:
        \[
            P(X < g_2) = P(X < g_1) + P(g_1 < X < g_2) = 0,2 + 0,2 = 0,4.
        \]
        We kunnen de tweede klassegrens dus bepalen als volgt:
        \[
            g_2 = \invnorm(opp=0,4; \mu=100; \sigma=20) \approx 94,9331 \text{ g}.
        \]
        Op een soortgelijke manier vinden we $g_3$ en $g_4$:
        \begin{align*}
            g_3 &= \invnorm(opp=0,6; \mu=100; \sigma=20) \approx 105,0669 \text{ g}. \\
            g_4 &= \invnorm(opp=0,8; \mu=100; \sigma=20) \approx 116,8324 \text{ g}.
        \end{align*}

        \textbf{LET OP:} de eerste klasse loopt van $-\infty$ tot $g_1$.
        Aangezien appels geen negatief gewicht kunnen hebben lijkt dit raar.
        Dit is echter een gevolg van de aanname dat we met een normale verdeling werken, die alle re\"ele getallen als uitkomst kan hebben.
        De kans op een appel met een negatief gewicht is echter bijzonder klein.
        De $z$-score is gelijk aan $z = \frac{0-100}{20} = -5$, dus
        \begin{align*}
            P(X < 0)    &= P(Z < -5) \\
                        &=\text{normalcdf}(a=-10^10; b=-5; \mu=0;\sigma=1) \\
                        &= 2,8711 \cdot 10^{-7}
        \end{align*}
    }
\end{enumerate}
