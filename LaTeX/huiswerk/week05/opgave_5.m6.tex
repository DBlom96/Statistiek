\question{5.m5}{Voor een speerwerper geldt dat het resultaat van een worp $X$ kan worden beschreven door een normale verdeling met $\mu=61$ meter en $\sigma=3$ meter.
De speerwerper doet drie pogingen, waarvoor mag worden verondersteld dat de worpresultaten kunnen worden beschouwd als onderling onafhankelijke kansvariabelen.
De kans dat het gemiddelde resultaat van de drie worpen minder is dan 59 meter, is ongeveer gelijk aan ...}
\begin{enumerate}[label=(\alph*)]
    \item $0,35$.
    \item $0,255$.
    \item $0,125$.
    \item $0,745$.
\end{enumerate}
\answer{Aangezien de originele verdeling de normale verdeling is, volgt uit de centrale limietstelling dat de kansverdeling van het gemiddelde $\overline{X}$ van deze drie worpen van de speerwerper gelijk is aan de normale verdeling met verwachtingswaarde $\mu = 61$ meter en standaarddeviatie $\frac{\sigma}{\sqrt{3}} = \sqrt{3}$ meter.
De kans dat het gemiddelde resultaat van de drie worpen minder is dan 59 meter is gelijk aan:
\[
    P(\overline{X} \le 59) = \text{normalcdf}(a=-10^{10}; b=59; \mu=61; \sigma=\sqrt{3}) \approx 0,1241.
\]
Het juiste antwoord is dus (c).
In de grafische rekenmachine kun je in het functiescherm het volgende invullen:
\begin{align*}
    y_1 &= \text{normalcdf}(-10^{10}; 15,5; x; 9) \\
    y_2 &= 0,3085
\end{align*}
De optie intersect (2nd-TRACE-5) geeft in dat geval een snijpunt voor $x = 20,0010$.
Dit betekent dat $\mu$ een waarde van ongeveer 20 heeft.
Het juiste antwoord is dus (c).
}