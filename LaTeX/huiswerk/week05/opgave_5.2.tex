\question{5.2}{Een vulmachine vult pakken rijst waarop staat dat daarin $1 000$ gram rijst zit.
Omdat er altijd wat variatie zit in de afgeleverde hoeveelheden, staat de machine ingesteld op een gemiddeld vulgewicht van $1 010$ gram.
We mogen ervan uitgaan dat de afgeleverde hoeveelheden beschreven worden door een normale verdeling met $\mu = 1 010$ gram.
De standaarddeviatie is echter onbekend.}

\begin{enumerate}[label=(\alph*)]
    \item Bij nauwkeurig nawegen van een groot aantal pakken bleek dat in $20\%$ van de gevallen een pak minder dan $1 000$ gram rijst bevat.
    Bereken op basis van deze informatie de standaarddeviatie van de vulgewichten.
    \answer{
        
    }

    \item Hoe groot is de kans dat een willekeurig pak rijst meer weegt dan $1 020$ gram?
    \answer{

    }

    \item Stel we kiezen willekeurig $16$ pakken rijst.
    Hoe groot is de kans dat deze gemiddeld minder dan $1 000$ gram rijst bevatten?
    \answer{
    
    }
\end{enumerate}
