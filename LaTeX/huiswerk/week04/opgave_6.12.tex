\question{6.12}{
    In de binnenstad van Utrecht worden door de politie op een gegeven dag 800 parkeerbonnen uitgeschreven, elk met een boetebedrag van 40 euro per parkeerbon.
    Het is een bekend gegeven dat 75\% van de ontvangers van zo'n bekeuring het bedrag overmaakt.
    25\% doet dat niet.
}

\begin{enumerate}[label=(\alph*)]
    \item Bereken voor de dag met 800 parkeerbonnen de verwachtingswaarde en de standaarddeviatie van het aantal directe betalers.
    Stel dat voor deze specifieke dag op de uiterste betaaldatum 520 bekeurden hun boete hebben voldaan, is dat dan uitzonderlijk gegeven de betaalkans van 75\%?
    \answer{

    }

    \item Wat is het verwachte bedrag dat binnenkomt van een dag met 800 bekeurden en wat is de standaarddeviatie?
    \answer{
        
    }

    \item De gemeente besluit extra invorderingsmaatregelen te nemen voor de niet-betalers van de boete.
    Zo'n actie kost 5 euro per niet-betaalde parkeerbon.
    De boet is nu verhoogd naar 52 euro.
    Bekend is dat van deze groep 60\% daarna betaalt.
    Maak een kosten-batenanalyse van de extra maatregel.
    \answer{
        
    }
\end{enumerate}