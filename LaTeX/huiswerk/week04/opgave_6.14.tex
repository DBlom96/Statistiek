\question{6.14}{
    Bij een keukenbedrijf weet men dat van alle klanten die een catalogus aanvragen, 15\% daadwerkelijk een bestelling zal plaatsen in de maand volgend op de toezending van de catalogus.
    Daarna, dus m\'e\'er dan een maand later, blijken er nooit bestellingen te worden gedaan.
}

\begin{enumerate}[label=(\alph*)]
    \item In een bepaalde maand worden 120 catalogi verzonden.
    Bereken de kans dat meer dan twintig klanten een bestelling gaan plaatsen.
    \answer{
        Laat $X$ de kansvariabele zijn die het aantal klanten telt dat daadwerkelijk een bestelling zal plaatsen.
        Er geldt dat $X \sim \text{Binomiaal}(n; p)$, met parameters $n=120$ (aantal klanten die een catalogus aangevraagd hebben) en $p=0,15$ (kans dat een klant die een catalogus aanvraagt daadwerkelijk een bestelling plaatst)
        We willen de kans $P(X > 20)$ bepalen (``meer dan" betekent ``>"):
        \begin{align*}
            P(X > 20) = P(X \ge 21) &= 1 - P(X \le 20) \\
                                    &= 1 - \text{binomcdf}(n=120; p=0,15; k=20) \\
                                    &\approx 0,2557
        \end{align*}
        
        Met 25,57\% kans zullen meer dan twintig klanten een bestelling gaan plaatsen.
    }

    \item In een bepaald jaar worden 1 200 catalogi aangevraagd.
    Hoe groot is het verwachte aantal bestellingen? Bereken een voorspellingsinterval (symmetrisch) waarvoor geldt dat hierin met 95\% kans het aantal waar te nemen bestellingen ligt.
    \answer{
        De kansvariabele $X\sim\text{Binomiaal}(n; p)$ telt het aantal klanten dat daadwerkelijk een bestelling zal plaatsen, met nu
        $n=1200$ (aantal klanten die een catalogus aangevraagd hebben) en $p=0,15$ (kans dat een klant die een catalogus aanvraagt daadwerkelijk een bestelling plaatst).
        We vinden nu dat
        \[
            E[X] = n \cdot p = 1200 \cdot 0,15 = 180,
        \]
        en
        \[
            \sigma(X) = \sqrt{n\cdot p \cdot (1-p)}\approx 12,3693.
        \]
        Het 95\%-voorspellingsinterval worden dan gegeven door
        \begin{align*}
            E[X] - 2\cdot \sigma(X) &=180 - 2 \cdot 12,3693 \approx 155,2614 \\
            E[X] + 2\cdot \sigma(X) &=180 + 2 \cdot 12,3693 \approx 204,7386
        \end{align*} 
        
        Met 95\% kans zal het aantal waar te nemen bestellingen tussen 155 en 205 liggen (merk op: naar buiten afronden om 95\% zekerheid te behouden).
    }

    \item In een bepaalde periode werden 800 catalogi verzonden.
    Hoe groot is de kans dat de fractie bestellers minder is dan 10\% voor deze groep aanvragers?
    \answer{
        De kansvariabele $X \sim \text{Binomiaal}(n; p)$ telt het aantal klanten dat daadwerkelijk een bestelling zal plaatsen, met nu
        $n=800$ (aantal klanten die een catalogus aangevraagd hebben) en $p=0,15$ (kans dat een klant die een catalogus aanvraagt daadwerkelijk een bestelling plaatst).
        Laat $F$ de fractie van het totaal aantal klanten zijn dat daadwerkelijk een bestelling zal plaatsen.
        We willen de kans $P(F < 0,1)$ bepalen (``minder dan"\ betekent ``<"):
        \begin{align*}
            P(F < 0,1)  &= P(X < 0,1 \cdot n) \\
                        &= P(X < 80) \\
                        &= P(X \le 79) \\
                        &= \text{binomcdf}(n=800; p=0,15; k=79) \\
                        &\approx 1,2317 \cdot 10^{-5}
        \end{align*}
        Het is zeer onwaarschijnlijk dat minder dan 10\% van de groep aanvragers een bestelling plaatst.
    }

\end{enumerate}