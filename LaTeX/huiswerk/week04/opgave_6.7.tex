\question{6.7}{
    De Bloedbank is een organisatie waar burgers bloed kunnen laten aftappen. 
    Dit donorbloed kan worden gebruikt bij pati\"enten die dit nodig hebben.
    In Nederland heeft 44\% van de bevolking bloedgroep $A$.
    Deze groep is onderverdeeld in 36\% met bloedgroep $A+$ en 8\% met bloedgroep $A-$.
    Er komen op een bepaalde dag 30 donoren om bloed te geven.
}

\begin{enumerate}[label=(\alph*)]
    \item Hoe groot is de kans dat hiervan precies 8 donoren bloedgroep $A$ hebben?
    \answer{
        Het aantal bloeddonoren met bloedgroep $A$ uit een groep van 30 donoren is een binomiaal verdeelde kansvariabele $X$ met parameters $n = 30$ (aantal donoren) en $p = 0,44$ (succeskans / kans op bloedgroep $A$).
        De kans dat precies 8 van de 30 donoren bloedgroep $A$ hebben is gelijk aan
        \[
            P(X=8) = \text{binompdf}(n=30; p=0,44; k=8) \approx 0,0237
        \]
    }

    \item Hoe groot is de kans dat \emph{binnen} de groep van 8 mensen met bloedgroep $A$ er twee zijn met bloedgroep $A-$?
    \answer{
        Het aantal donoren met bloedgroep $A-$ binnen een groep van 8 donoren met bloedgroep $A$ is een binomiaal verdeelde kansvariabele $Y$ met parameters $n=8$ en $p = \frac{36}{36+8}=\frac{36}{44}$.
        De kans dat precies twee donoren binnen deze groep bloedgroep $A-$ heeft is gelijk aan
        \[
            P(Y=2) = \text{binompdf}(n = 8; p = \frac{36}{44}; k=2) \approx 0,0007
        \]
    }
\end{enumerate}