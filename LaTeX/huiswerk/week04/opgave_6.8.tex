\question{6.8}{
    De kans dat een willekeurig gekozen student slaagt bij een bepaald tentamen is $0,6$.
    Hoe groot is de kans dat van 150 willekeurig gekozen studenten er meer dan 100 voor het tentamen slagen?
}

\answer{
    Het aantal studenten dat slaagt bij een bepaald tentamen is een binomiaal verdeelde kansvariabele $X$ met parameters $n = 150$ (aantal willekeurig gekozen studenten) en $p=0,6$ (de succeskans, kans op slagen).
    De kans dat meer dan 100 studenten voor het tentamen slagen is gelijk aan 
    \[
        P(X>100) = P(X \ge 101) = 1 - P(X \le 100) = 1 - \text{binomcdf}(n=150; p=0,6; k=100) \approx 0,0389.
    \]
}