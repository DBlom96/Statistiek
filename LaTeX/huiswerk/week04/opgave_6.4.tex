\question{6.4}{Een multiplechoice-examen bestaat uit tien vragen die elk drie antwoordmogelijkheden kennen, waarvan er precies \'e\'en correct is.
Een kandidaat die volstrekt niet weet wat hij moet antwoorden, kruist naar willekeur bij elk van de tien vragen een antwoord aan.
}

\begin{enumerate}[label=(\alph*)]
    \item Bereken de kans op respectievelijk nul, \'e\'en of twee antwoorden goed.
    \answer{
        Laat $X$ de kansvariabele zijn die het aantal juiste antwoorden telt.
        Dan geldt dat $X$ binomiaal verdeeld is met parameters $n=10$ (aantal ``kansexperimenten'', oftewel aantal examenvragen) en $p=\frac{1}{3}$ (succeskans / kans op het juiste antwoord)
        De kansen op respectievelijk nul, \'e\'en of twee antwoorden goed zijn gelijk aan
        \begin{align*}
            P(X=0) = \text{binompdf}(n=10; p = \frac{1}{3}; k=0) &\approx 0,0174 \\
            P(X=1) = \text{binompdf}(n=10; p = \frac{1}{3}; k=1) &\approx 0,0870 \\
            P(X=2) = \text{binompdf}(n=10; p = \frac{1}{3}; k=2) &\approx 0,1955 \\
        \end{align*}
    }

    \item Bereken de kans dat hij minstens zes antwoorden goed heeft.
    \answer{
        De kans dat hij minstens zes antwoorden goed heeft is gelijk aan
        \[
            P(X \ge 6) = 1 - P(X \le 5) = 1 - \text{binomcdf}(n=10; p=\frac{1}{3}; k=5) \approx 0,0762
        \]
    }

    \item Bereken de verwachtingswaarde van het aantal goede antwoorden.
    \answer{
        De verwachtingswaarde van een binomiaal verdeelde kansvariabele $X \sim \text{Bin}(n; p)$ is gelijk aan $E[X]=n\cdot p$.
        Aangezien in dit geval geldt dat $n=10$ en $p=\frac{1}{3}$, is de verwachtingswaarde van het aantal correcte antwoorden gelijk aan $E[X]=10\cdot \frac{1}{3} = \frac{10}{3}$.
    }
\end{enumerate}
