\question{6.6}{
    Bij een aangeboren afwijking blijkt bij 25\% van de baby's een bepaalde complicatie op te treden.
    In een jaar worden twintig baby's geboren met de afwijking.
    Bepaal met de tabel van de binomiale verdeling de volgende kansen:
}

\begin{enumerate}[label=(\alph*)]
    \item de kans dat twee of minder baby's de complicatie vertonen.
    \answer{
        Het aantal baby's waarbij bij een aangeboren afwijking een bepaalde complicatie optreedt is een binomiaal verdeelde kansvariabele $X$ met parameters $n = 20$ (aantal baby's met een aangeboren afwijking) en $p = 0,25$ (kans op complicatie, het is in deze context vreemd om het een ``succeskans" te noemen...).
        De kans dat twee of minder baby's de complicatie vertonen is gelijk aan
        \[
            P(X \le 2) = \text{binomcdf}(n=20; p=0,25; k=2) \approx 0,0913    
        \]
    }

    \item de kans dat precies twee baby's de complicatie vertonen.
    \answer{
        De kans dat precies twee baby's de complicatie vertonen is gelijk aan
        \[
            P(X = 2) = \text{binompdf}(n=20; p=0,25; k=2) \approx 0,0670    
        \]
    }

    \item de kans dat meer dan twee baby's de complicatie vertonen.
    \answer{
        De kans dat meer dan twee baby's de complicatie vertonen is gelijk aan
        \begin{align*}
            P(X > 2) = P(X \ge 3)   &= 1 - P(X \le 2) \\
                                    &= 1 - \text{binompdf}(n=20; p=0,25; k=2) \\
                                    &\approx 0,9087    
        \end{align*}
    }

    \item de kans dat het aantal baby's met de complicatie minstens 4 maar hoogstens 7 is.
    \answer{
        De kans dat minstens 4 maar hoogstens 7 baby's de complicatie vertonen is gelijk aan
        \begin{align*}
            P(4 \le X \le 7)    &= P(X \ge 7) - P(X \le 3) \\
                                &= \text{binomcdf}(n=20; p=0,25; k=7) \\
                                &\qquad- \text{binomcdf}(n=20; p=0,25; k=3) \\
                                &\approx 0,89819 - 0,41484 \approx 0,4834    
        \end{align*}
        
    }
\end{enumerate}