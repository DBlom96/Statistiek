\question{6.5}{Bekend is dat van de volwassen Nederlanders thans 70\% over een of meer creditcards beschikt. 
We kiezen een steekproef van vijftien personen.
Bepaal met behulp van de tabel van de binomiale verdeling de kans dat de volgende aantallen mensen in deze steekproef over een of meer creditcards beschikken:
}

\begin{enumerate}[label=(\alph*)]
    \item precies 10
    \answer{
        Het aantal mensen in een steekproef van 15 personen met een of meer creditcards is een binomiaal verdeelde kansvariabele $X$ met parameters $n = 15$ (aantal personen) en $p = 0,7$ (succeskans / kans dat we iemand hebben gekozen die een of meer creditcards heeft).
        De kans dat precies 10 van de 15 mensen een of meer creditcards heeft is gelijk aan
        \[
            P(X=10) = \text{binompdf}(n=15; p=0,7; k=10) \approx 0,2016
        \]
    }

    \item precies 15
    \answer{
        De kans dat alle 15 mensen in de steekproef een of meer creditcards heeft is gelijk aan
        \[
            P(X=15) = \text{binompdf}(n=15; p=0,7; k=15) \approx 0,0047
        \]
    }

    \item meer dan 9
    \answer{
        De kans dat meer dan 9 (dus minstens 10) van de 15 mensen in de steekproef een of meer creditcards heeft is gelijk aan
        \begin{align*}
            P(X > 9) = P(X \ge 10)  &= 1 - P(X \le 9) \\
                                    &= 1 - \text{binomcdf}(n=15; p=0,7; k=9) \\
                                    &\approx 0,7216
        \end{align*}
    }

    \item minder dan 12
    \answer{
        De kans dat minder dan 12 (dus hoogstens 11) van de 15 mensen in de steekproef een of meer creditcards heeft is gelijk aan
        \[
            P(X < 12) = P(X \le 11) = \text{binomcdf}(n=15; p=0,7; k=11) \approx 0,7031
        \]
    }
\end{enumerate}