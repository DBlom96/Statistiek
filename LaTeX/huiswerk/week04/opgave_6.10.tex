\question{6.10}{
    Een viermotorig vliegtuig kan nog blijven doorvliegen als tijdens de vlucht twee van de vier motoren uitvallen.
    Voor een willekeurige motor is de kans $0,005$ dat deze tijdens de vlucht defect raakt.
}

\begin{enumerate}[label=(\alph*)]
    \item Hoe groot is de kans dat na een vlucht alle vier de motoren nog functioneren? En drie van de vier?
    \answer{
        Het aantal motoren dat na een vlucht nog functioneren is een binomiaal verdeelde kansvariabele met parameters $n = 4$ (aantal motoren) en $p=1-0,005=0,995$ (succeskans, kans dat een motor nog functioneert).
        De kans dat alle vier de motoren nog functioneren is gelijk aan
        \[
            P(X=4) = \text{binompdf}(n=4; p=0,995; k=4) \approx 0,9801.
        \]

    }

    \item Bereken de kans dat het vliegtuig neerstort?
    \answer{
        Een vliegtuig stort neer als meer dan twee van de vier motoren uitvallen, oftewel dat maximaal \'e\'en motor nog functioneert.
        De kans dat hoogstens \'e\'en motor nog functioneert na een vlucht is gelijk aan
        \[
            P(X \le 1) = \text{binomcdf}(n=4; p=0,995; k=1) \approx 4,9813 * 10^{-7}
        \]
    }

    \item Van de motoren zijn er twee bevestigd aan de linkervleugel en twee aan de rechtervleugel.
    Als drie of vier motoren uitvallen, stort het vliegtuig neer.
    Als echter twee motoren uitvallen die zich aan dezelfde kant van het vliegtuig bevinden, dan stort het vliegtuig ook neer.
    Bereken opnieuw de kans dat vliegtuig neerstort.
    Geef commentaar op het verschil met het antwoord op vraag b.
    \answer{
        We kunnen in dit geval de situatie het best opsplitsen in twee binomiaal verdeelde kansvariabelen $X_{L}$ en $X_{R}$ die het aantal nog functionerende motoren telt aan respectievelijk de linker- en rechtervleugel.
        Deze binomiaal verdeelde kansvariabelen hebben parameters $n = 2$ en $p = 0,995$.
        Een vliegtuig stort alleen neer als \'e\'en van beide kansvariabelen een waarde van 0 aanneemt.
        Dit gebeurt met kans
        \begin{align*}
            P(X_{L} = 0 \text{ of } X_{R}=0)    &= 1 - P(X_{L} \ge 1 \text{ en } X_{R} \ge 1) \\
                                                &= 1 - P(X_{L} \ge 1)\cdot P(X_{R} \ge 2) \\
                                                &= 1 - \text{binomcdf}
        \end{align*}
    }
\end{enumerate}