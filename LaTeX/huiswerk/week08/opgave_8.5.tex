\question{8.5}{
    Een bedrijf vervaardigt plastic draagtasjes die winkeliers kunnen verkopen aan klanten om hun inkopen in te doen.
    Deze dienen een behoorlijk groot gewicht te kunnen dragen.
    We willen weten hoe sterk die tasjes zijn.
    Bij proeven worden de tasjes steeds zwaarder belast totdat zij stukgaan.
    Bekend is dat de breeksterkte van een tasje, uitgedrukt in kilogram belasting, beschreven kan worden door een normale verdeling met $\mu=11,5$ en $\sigma=1,3$ kg.
}
\begin{enumerate}[label=(\alph*)]
    \item Hoeveel procent van de tasjes breekt reeds bij een belasting van hoogstens $10$ kg?
    \answer{
        Laat $X$ de kansvariabele zijn die de breeksterkte van een tasje (in kg) meet.
        Gegeven is dat $X \sim N(\mu = 11,5; \sigma = 1,3)$.
        De kans dat een tasje breekt bij een belasting van hoogstens $10$ kg is gelijk aan
        \begin{align*}
            P(X \le 10) = \normalcdf(a=-10^{99}; b=10; \mu=11,5; \sigma=1,3) \approx 0,1243.
        \end{align*}
        Dat betekent dus dat $12,43\%$ van de tasjes al breekt bij een belasting van hoogstens $10$ kg.
    }

    \item De fabrikant gaat een nieuw soort plastic gebruiken bij de productie van de tasjes.
    Er wordt een steekproef genomen van $25$ tasjes.
    Deze leveren een gemiddelde breeksterkte op van $12,8$ kg.
    Ga ervan uit dat de standaarddeviatie nog hetzelfde is.
    Bepaal een $95\%$-betrouwbaarheidsinterval voor $\mu$: de gemiddelde breeksterkte.
    \answer{
        Laat nu $X \sim N(\mu=?; \sigma=1,3)$ de kansvariabele zijn die de breeksterkte (in kg) meet van een willekeurige tasje van het nieuw soort plastic.
        Volgens de centrale limietstelling is het steekproefgemiddelde $\overline{X}$ van $25$ tasjes dan ook normaal verdeeld met gemiddelde $\mu$ en standaardafwijking $\frac{\sigma}{\sqrt{n}} = \frac{1,3}{\sqrt{25}}$.
        Omdat de gewenste betrouwbaarheid gelijk is aan $95\%$, hebben we te maken met $z_{\alpha/2} = z_{0,025} = \invnorm(opp=0,975) \approx 1,9600$.
        Verder is gegeven dat $n=25$ en $\overline{x}=12,8$ kg.
        Het $95\%$ betrouwbaarheidsinterval is dus gelijk aan
        \begin{align*}
            &[\overline{x} - z_{\alpha/2} \cdot \frac{\sigma}{\sqrt{n}}; \overline{x} + z_{\alpha/2} \cdot \frac{\sigma}{\sqrt{n}}] \\
            &=[12,8 - 1,9600 \cdot \frac{1,3}{\sqrt{25}}; 12,8 + 1,9600 \cdot \frac{1,3}{\sqrt{25}}] \\
            &=[12,29; 13,31]
        \end{align*}
        De gemiddelde breeksterkte van tasjes van het nieuw soort plastic zal met $95\%$ betrouwbaarheid tussen $12,29$ en $13,31$ kg liggen.
    }

    \item Hoeveel procent van de tasjes breekt bij een belasting van minder dan $10$ kg als de nieuwe waarde van $\mu$ $12,8$ is?
    En hoeveel procent is dit als $\mu$ gelijk is aan de linkergrens $L$ respectievelijk de rechtergrens $R$ van het interval dat berekend is bij vraag \textbf{b}?
    \answer{
        In het geval dat de nieuwe waarde van $\mu=12,8$, kunnen we weer de kans berekenen met de normalcdf functie:
        \begin{align*}
            P(X \le 10) = \normalcdf(a=-10^{99}; b=10; \mu=12,8; \sigma=1,3) \approx 0,0156.
        \end{align*}
        Als $\mu=12,8$ kg, dan breekt $1,56\%$ van de tasjes van het nieuw soort plastic.

        In het geval dat de nieuwe waarde van $\mu=L=12,29$, kunnen we weer de kans berekenen met de normalcdf functie:
        \begin{align*}
            P(X \le 10) = \normalcdf(a=-10^{99}; b=10; \mu=12,29; \sigma=1,3) \approx 0,0391.
        \end{align*}
        Als $\mu=12,29$ kg, dan breekt $3,91\%$ van de tasjes van het nieuw soort plastic.

        In het geval dat de nieuwe waarde van $\mu=R=13,31$, kunnen we weer de kans berekenen met de normalcdf functie:
        \begin{align*}
            P(X \le 10) = \normalcdf(a=-10^{99}; b=10; \mu=12,29; \sigma=1,3) \approx 0,0054.
        \end{align*}
        Als $\mu=13,31$ kg, dan breekt $0,54\%$ van de tasjes van het nieuw soort plastic.
    }
\end{enumerate}
