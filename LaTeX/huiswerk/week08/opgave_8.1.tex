\question{8.1}{
    Een machine vult zakken met meel.
    Het gewicht van het meel in een willekeurige zak is te beschouwen als een trekking uit een normale verdeling met onbekende $\mu$ en een standaarddeviatie van $100$ gram.
    Om het schattingsinterval te berekenen voor $\mu$ weegt men de inhoud van $25$ zakken meel.
    Het gemiddelde gewicht van het meel per zak bedroeg $20,142$ kg.   
}
\begin{enumerate}[label=(\alph*)]
    \item Bereken een schattingsinterval voor $\mu$ (de verwachtingswaarde van het gewicht van het meel) dat een betrouwbaarheid heeft van $95\%$.
    \answer{
        Noteer met $X$ de kansvariabele die het gewicht (in gram) van een willekeurige zak meel meet.
        Gegeven is dat $X \sim N(\mu=?; \sigma=100)$.
        Volgens de centrale limietstelling geldt dat het gemiddelde gewicht $\overline{X}$ van $25$ zakken meel normaal verdeeld is met verwachtingswaarde $\mu$ en een standaarddeviatie $\frac{\sigma}{\sqrt{n}} = \frac{100}{\sqrt{25}} = 20$ gram.
        We hebben een geobserveerde steekproef van $25$ zakken genomen met uitkomst $\overline{x} = 20142$ gram. 
        Aangezien we een betrouwbaarheid van $95\%$ willen, is $\alpha = 0,05$ en $z_{\alpha/2} = \invnorm(1-\alpha/2) \approx 1,9600$.
        Het $95\%$-betrouwbaarheidsinterval voor $\mu$ is dan gelijk aan 
        \begin{align*}
            &[\overline{x} - z_{\alpha/2} \cdot \frac{\sigma}{\sqrt{n}}; \overline{x} + z_{\alpha/2} \cdot \frac{\sigma}{\sqrt{n}}] \\
            &=[20142 - 1,9600 \cdot \frac{100}{\sqrt{25}}; 20142 + 1,9600 \cdot \frac{100}{\sqrt{25}}] \\
            &=[20102,8; 20181,2]
        \end{align*}
        Met $95\%$ betrouwbaarheid ligt het gemiddelde gewicht van een zak meel tussen de $20102,8$ en $20181,2$ gram.
    }

    \item Bereken een schattingsinterval voor $\mu$ dat een betrouwbaarheid heeft van $99\%$.
    \answer{
        Aangezien we in dit geval een betrouwbaarheid van $99\%$ willen, is $\alpha = 0,01$ en $z_{\alpha/2} = \invnorm(1-\alpha/2) \approx 2,5758$.
        Het $99\%$-betrouwbaarheidsinterval voor $\mu$ is dan gelijk aan 
        \begin{align*}
            &[\overline{x} - z_{\alpha/2} \cdot \frac{\sigma}{\sqrt{n}}; \overline{x} + z_{\alpha/2} \cdot \frac{\sigma}{\sqrt{n}}] \\
            &=[20142 - 2,5758 \cdot \frac{100}{\sqrt{25}}; 20142 + 2,5758 \cdot \frac{100}{\sqrt{25}}] \\
            &=[20090,5; 20193,5]
        \end{align*}
        Met $99\%$ betrouwbaarheid ligt het gemiddelde gewicht van een zak meel tussen de $20090,5$ en $20193,5$ gram.
    }

    \item Hoe groot moet men de steekproefomvang $n$ kiezen om tot een $95\%$-betrouwbaarheidsinterval voor $\mu$ te komen dat een breedte heeft van $20$ eenheden (gram)?
    \answer{
        De breedte van het betrouwbaarheidsinterval
        \[
            [\overline{x} - z_{\alpha/2} \cdot \frac{\sigma}{\sqrt{n}}; \overline{x} + z_{\alpha/2} \cdot \frac{\sigma}{\sqrt{n}}]
        \]
        is gelijk aan $2z_{\alpha/2} \cdot \frac{\sigma}{\sqrt{n}} = 20$.
        Aangezien we een betrouwbaarheid van $95\%$ willen en een breedte van $20$ gram, krijgen we
        \begin{align*}
            2z_{\alpha/2} \cdot \frac{\sigma}{\sqrt{n}} = 20 \rightarrow n &\ge (\frac{2z_{\alpha/2}\cdot \sigma}{20})^2 \\
                                                                            &\approx 384,16
        \end{align*}
        Om de garantie van een breedte van maximaal $20$ gram te waarborgen, moeten we naar boven afronden, dus de steekproefomvang moet minimaal $n=385$ zijn.
    }
\end{enumerate}
