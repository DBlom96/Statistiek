\question{8.3}{
    Een meetapparaat in een laboratorium wordt gebruikt voor het bepalen van de hoeveelheid van een stof die aanwezig is in een oplossing.
    Van het apparaat is bekend dat dit voor de gemeten hoeveelheid een standaarddeviatie heeft van $1,2$ mg per bepaling.
    Er worden zes monsters genomen.
    Deze leverden voor de hoeveelheid stof de volgende resultaten (in milligram): $14,7$; $13,2$; $15,8$; $16,1$; $14,3$ en $15,0$.
    Bereken met deze gegevens een $95\%$-betrouwbaarheidsinterval voor $\mu$: de gemiddelde hoeveelheid van de stof per monster.
    Ga ervan uit dat de meetresultaten een normale verdeling volgen.
}

\answer{

}
