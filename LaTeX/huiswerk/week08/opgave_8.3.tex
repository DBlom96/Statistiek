\question{8.3}{
    Een meetapparaat in een laboratorium wordt gebruikt voor het bepalen van de hoeveelheid van een stof die aanwezig is in een oplossing.
    Van het apparaat is bekend dat dit voor de gemeten hoeveelheid een standaarddeviatie heeft van $1,2$ mg per bepaling.
    Er worden zes monsters genomen.
    Deze leverden voor de hoeveelheid stof de volgende resultaten (in milligram): $14,7$; $13,2$; $15,8$; $16,1$; $14,3$ en $15,0$.
    Bereken met deze gegevens een $95\%$-betrouwbaarheidsinterval voor $\mu$: de gemiddelde hoeveelheid van de stof per monster.
    Ga ervan uit dat de meetresultaten een normale verdeling volgen.
}

\answer{
    Gegeven is de standaardafwijking $\sigma = 1,2$ mg en de gewenste betrouwbaarheid van $95\%$, dus $\alpha=0,05$.
    Het steekproefgemiddelde van een steekproef van zes monsters is volgens de centrale limietstelling weer normaal verdeeld met gemiddelde $\mu$ en standaarddeviatie $\frac{\sigma}{\sqrt{n}} = \frac{1,2}{\sqrt{6}}$.
    Om het betrouwbaarheidsinterval te berekenen, bepalen we eerst het steekproefgemiddelde:
    \begin{align*}
        \overline{x} = \frac{x_1+x_2+\ldots+x_6}{6} = \frac{14,7+13,2+\ldots+15,0}{6} = 14,85
    \end{align*} 

    Omdat $\alpha = 0,05$, werken we verder met $z_{\alpha/2} = z_{0,025} = \invnorm(0,975) \approx 1,9600$.
    Dit geeft een betrouwbaarheidsinterval van
    \begin{align*}
        &[\overline{x} - z_{\alpha/2} \cdot \frac{\sigma}{\sqrt{n}}; \overline{x} + z_{\alpha/2} \cdot \frac{\sigma}{\sqrt{n}}] \\
        &=[14,85 - 1,9600 \cdot \frac{1,2}{\sqrt{6}}; 14,85 + 1,9600 \cdot \frac{1,2}{\sqrt{6}}] \\
        &=[13,89; 15,81]
    \end{align*}
    Met $95\%$ ligt de gemiddelde hoeveelheid stof per monster tussen $13,89$ en $15,81$ mg.
}
