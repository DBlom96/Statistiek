\question{8.10}{
    Bij een snelweg is door een verdekt opgestelde controlpost de snelheid bepaald voor een grote steekproef van de passerende auto's.
    Uit de aantekeningen van de onderzoeker blijkt dat de berekende standaarddeviatie van de snelheden $8,2$ km/uur bedraagt.
    Verder bleek dat men als $95\%$-betrouwbaarheidsinterval voor de gemiddelde snelheid als resultaat opgeschreven had dat $\mu$ tussen $74,2$ en $76,8$ zou moeten liggen.
    Helaas was niet meer te vinden hoe groot de steekproefomvang is geweest.
}
\begin{enumerate}[label=(\alph*)]
    \item Bereken op basis van de verstrekte gegevens de omvang van de steekproef.
    (Ga er hierbij van uit dat de nog onbekende steekproefomvang groot genoeg is om toepassing van de normale verdeling te rechtvaardigen.)
    \answer{

    }

    \item Uit het interval blijkt dat $\mu$ op z'n minst $74,2$ en op z'n hoogst $76,8$ zou kunnen zijn.
    Op het betreffende traject geldt een maximumsnelheid van $80$ km/uur.
    Bereken op basis van beide uiterste waarden van $\mu$ het percentage auto's dat te hard rijdt.
    De snelheden vormen een normale verdeling.
    Veronderstel dat de politie alleen bekeuringen geeft indien de snelheid minstens $85,0$ km/uur is.
    Hoeveel procent van de automobilisten krijgt dan een bekeuring?
    Formuleer dit als een interval.
    \answer{

    }
\end{enumerate}
