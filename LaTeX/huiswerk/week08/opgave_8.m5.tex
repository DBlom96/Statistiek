\question{8.m5}{Van een normale verdeling met onbekende $\mu$ en $\sigma$ worden zes waarnemingen gedaan.
Deze waren: $8, 12, 7, 9, 11, 13$.
De puntschatting voor $\sigma$ is dus $\ldots$}
\begin{enumerate}[label=(\alph*)]
    \item $5,6$
    \item $4,67$
    \item $2,37$
    \item $1,87$
\end{enumerate}
\answer{
    We berekenen een puntschatting $s$ voor een onbekende standaardafwijking $\sigma$ van een normale verdeling door de steekproefvariantie $s^2$ te bepalen en daar de wortel uit te nemen.
    Allereerst berekenen we het steekproefgemiddelde
    \[
        \overline{x} = \frac{x_1 + x_2 + \ldots + x_6}{6} = \frac{8+12+7+9+11+13}{6} = 10.
    \]

    Nu kunnen we de steekproefvariantie bepalen:
    \begin{align*}
        s^2 &= \frac{(x_1-\overline{x})^2 + \ldots + (x_n-\overline{x})^2}{n-1} \\
            &= \frac{(8-10)^2 + \ldots + (13-10)^2}{5} \\
            &= \frac{(x_1-\overline{x})^2 + \ldots + (x_1-\overline{6})^2}{6-1} \\
            &= 5,6
    \end{align*}
    De puntschatting $s$ voor de standaardafwijking $\sigma$ vinden we door de wortel hiervan te nemen:
    \[
        s = \sqrt{s^2} = \sqrt{5,6} \approx 2,3664.
    \]
    Het juiste antwoord is dus antwoord (c).
}