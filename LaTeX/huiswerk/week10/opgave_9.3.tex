\question{9.3}{
    Een variabele $X$ is normaal verdeeld met standaarddeviatie $10$.
    We toetsen $H_0: \mu=50$ tegen $H_1: \mu > 50$.
}
\begin{enumerate}[label=(\alph*)]
    \item Bereken het kritieke gebied als er \'e\'en waarneming wordt gedaan bij een kans op een fout van de eerste soort $\alpha=0,05$ en ook bij $\alpha=0,01$.
    \answer{
        Gegeven is dat we te maken hebben met een kansvariabele $X \sim N(\mu=?; \sigma=10)$.
        We willen toetsen of er geldt dat $\mu = 50$.
        Stel verder eerst dat $\alpha = 0,05$.

        Als de nulhypothese $H_0$ waar is, dan geldt dat $X \sim N(\mu=50; \sigma=10)$.
        Merk op dat de alternatieve hypothese waarschijnlijker is als de waarneming die wordt gedaan leidt tot een extreem hoge waarde.
        Het kritieke gebied is dus van de vorm $(g; \infty)$, waarbij $g$ de grenswaarde is op $1 - \alpha$ 

    }

    \item We doen $100$ waarnemingen van de betrokken variabele.
    Bereken het kritieke gebied bij $\alpha=0,10$ en $\alpha=0,001$.
    \answer{
        
    }
\end{enumerate}