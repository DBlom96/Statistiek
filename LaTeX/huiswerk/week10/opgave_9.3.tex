\question{9.3}{
    Een variabele $X$ is normaal verdeeld met standaarddeviatie $10$.
    We toetsen $H_0: \mu=50$ tegen $H_1: \mu > 50$.
}
\begin{enumerate}[label=(\alph*)]
    \item Bereken het kritieke gebied als er \'e\'en waarneming wordt gedaan bij een kans op een fout van de eerste soort $\alpha=0,05$ en ook bij $\alpha=0,01$.
    \answer{
        Gegeven is dat we te maken hebben met een kansvariabele $X \sim N(\mu=?; \sigma=10)$.
        We willen toetsen of er geldt dat $\mu = 50$.
        \begin{itemize}
        \item Stel verder eerst dat $\alpha = 0,05$.
        Omdat we rechtszijdig toetsen, berekenen we de $z$-waarde
        \[
            z_{\alpha} = \invnorm(opp=1-\alpha) = \invnorm(opp=0,95) \approx 1,6449
        \]

        De toetsingsgrootheid is in dit geval het (theoretische) steekproefgemiddelde $\overline{X}$.
        Uit de centrale limietstelling volgt dat $\overline{X} \sim N(\mu; \frac{\sigma}{\sqrt{n}})$.
        Onder de aanname dat $H_0$ waar is, geldt dat het gemiddelde $\mu = 50$.
        Omdat verder de standaardafwijking $\sigma(X) = 10$ bekend is en we een enkele waarneming doen, oftewel $n=1$,
        volgt dat $\sigma(\overline{X}) = \frac{\sigma}{\sqrt{n}} = \frac{10}{\sqrt{1}}=10$, oftewel $\overline{X} \sim N(\mu=50; \sigma=10)$.
        Merk op dat hoe hoger de waarneming is, hoe waarschijnlijker dat de alternatieve hypothese waar is.
        Het kritieke gebied is dus van de vorm $[g; \infty)$, waarbij $g$ kan worden berekend als volgt:
        \begin{align*}
            g   &= \mu + z_{\alpha} \cdot \frac{\sigma}{\sqrt{n}} \\
                &= 50 + 1,6449 \cdot \frac{10}{\sqrt{1}} \\
                &\approx 66,4485
        \end{align*}

        Het kritieke gebied is dus gelijk aan $[66,4485; \infty)$.

        \item Stel nu dat $\alpha = 0,01$.
        Omdat we rechtszijdig toetsen, berekenen we de $z$-waarde
        \[
            z_{\alpha} = \invnorm(opp=1-\alpha) = \invnorm(opp=0,99) \approx 2,3263
        \]

        Het kritieke gebied is opnieuw van de vorm $[g; \infty)$, waarbij $g$ kan worden berekend als volgt:
        \begin{align*}
            g   &= \mu + z_{\alpha} \cdot \frac{\sigma}{\sqrt{n}} \\
                &= 50 + 2,3263 \cdot \frac{10}{\sqrt{1}} \\
                &\approx 73,2635
        \end{align*}

        Het kritieke gebied is dus gelijk aan $[73,2635; \infty)$.
        \end{itemize}
    }

    \item We doen $100$ waarnemingen van de betrokken variabele.
    Bereken het kritieke gebied bij $\alpha=0,10$ en $\alpha=0,001$.
    \answer{
        \begin{itemize}
            \item Stel eerst dat $\alpha = 0,10$.
            Omdat we rechtszijdig toetsen, berekenen we de $z$-waarde
            \[
                z_{\alpha} = \invnorm(opp=1-\alpha) = \invnorm(opp=0,9) \approx 1,2816
            \]

            De toetsingsgrootheid is in dit geval het (theoretische) steekproefgemiddelde $\overline{X}$.
            Uit de centrale limietstelling volgt dat $\overline{X} \sim N(\mu; \frac{\sigma}{\sqrt{n}})$.
            Onder de aanname dat $H_0$ waar is, geldt dat het gemiddelde $\mu = 50$.
            Omdat verder de standaardafwijking $\sigma(X) = 10$ bekend is en we nu $100$ waarnemingen, oftewel $n=100$, 
            volgt dat $\sigma(\overline{X}) = \frac{\sigma}{\sqrt{n}} = \frac{10}{\sqrt{100}}=1$, oftewel $\overline{X} \sim N(\mu=50; \sigma=1)$.
            Merk op dat hoe hoger het steekproefgemiddelde is, hoe waarschijnlijker dat de alternatieve hypothese waar is.
            Het kritieke gebied is dus van de vorm $[g; \infty)$, waarbij $g$ kan worden berekend als volgt:
            \begin{align*}
                g   &= \mu + z_{\alpha} \cdot \frac{\sigma}{\sqrt{n}} \\
                    &= 50 + 1,2816 \cdot \frac{10}{\sqrt{100}} \\
                    &\approx 51,2816
            \end{align*}

            Het kritieke gebied is dus gelijk aan $[51,2816; \infty)$.
            
            \item Stel nu dat $\alpha = 0,001$.
            Omdat we rechtszijdig toetsen, berekenen we de $z$-waarde
            \[
                z_{\alpha} = \invnorm(opp=1-\alpha) = \invnorm(opp=0,999) \approx 3,0902
            \]

            Het kritieke gebied is opnieuw van de vorm $[g; \infty)$, waarbij $g$ kan worden berekend als volgt:
            \begin{align*}
                g   &= \mu + z_{\alpha} \cdot \frac{\sigma}{\sqrt{n}} \\
                    &= 50 + 3,0902 \cdot \frac{10}{\sqrt{100}} \\
                    &\approx 53,0902
            \end{align*}

            Het kritieke gebied is dus gelijk aan $[53,0902; \infty)$.
        \end{itemize} 
    }   
\end{enumerate}