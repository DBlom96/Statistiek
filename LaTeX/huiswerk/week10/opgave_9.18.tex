\question{9.18}{
    Een bedrijf voert een administratie van het aantal klachten van afnemers.
    In acht weken werden de volgende aantallen klachten per week geregistreerd:

    \begin{center}
        \begin{tabular}{ccccccccc}
            \toprule
                {\bfseries Week} & {\bfseries 1} & {\bfseries 2} & {\bfseries 3} & {\bfseries 4} & {\bfseries 5} & {\bfseries 6} & {\bfseries 7} & {\bfseries 8} \\
            \cmidrule{1-1} \cmidrule{2-2} \cmidrule{3-3} \cmidrule{4-4} \cmidrule{5-5} \cmidrule{6-6} \cmidrule{7-7} \cmidrule{8-8} \cmidrule{9-9}
                Aantal klachten & $30$ & $45$ & $46$ & $50$ & $52$ & $34$ & $35$ & $28$ \\ 
            \bottomrule
        \end{tabular}
    \end{center}

    We gaan ervan uit dat het aantal klachten per week kan worden beschouwd als een variabele met een normale verdeling.
    In het verleden bleek het aantal klachten gemiddeld $51$ per week te zijn.
    Toets of dit gemiddelde nog houdbaar is in het licht van verzamelde uitkomsten (kies $\alpha=0,05$).
}
\answer{
    In deze hypothesetoets geldt dat de nulhypothese $H_0: \mu = 51$ (gemiddeld $51$ klachten per week) getest wordt tegen de alternatieve hypothese $H_1: \mu \neq 51$ (gemiddeld iets anders dan $51$ klachten per week).
    In acht weken wordt het aantal klachten geteld, oftewel $n=8$
    Het geobserveerde steekproefgemiddelde is gelijk aan 
    \begin{align*}
        \overline{x} = \frac{x_1+x_2+\ldots+x_n}{n} = \frac{30+45+\ldots+28}{8} = 40.
    \end{align*}
    Verder is de standaarddeviatie $\sigma$ niet gegeven, dus moeten we die schatten aan de hand van de gegevens.
    De schatting $s^2$ van de steekproefvariantie vinden we als volgt
    \begin{align*}
     s^2    &= \frac{(x_n-\overline{x})^2 + \ldots + (x_n-\overline{x})^2}{n-1} \\
            &= \frac{(x_1-\overline{x})^2 + \ldots + (x_8-\overline{x})^2}{8-1} \\
            &= \frac{(30-40)^2 + \ldots + (28-40)^2}{8-1} \\
            &\approx 9,3350
    \end{align*}
    Hierdoor volgt een schatting van de standaarddeviatie $s = \sqrt{s^2} = \sqrt{9,3350} \approx 3,0553$.
    De steekproefomvang $n = 8 < 30$ en de standaardafwijking $\sigma$ onbekend is, dus we moeten een $t$-waarde bepalen.
    Omdat we tweezijdig toetsen, berekenen we die als volgt
    \[
        t = \invt(opp=1-\alpha/2; df=n-1) = \invt(opp=0,975; df=7) \approx 2,3646
    \]

    De toetsingsgrootheid is in dit geval het (theoretische) steekproefgemiddelde $\overline{X}$, die dus een $t$-verdeling volgt.
    Onder de aanname dat $H_0$ waar is, geldt dat het gemiddelde $\mu = 51$.
    Merk op dat hoe verder het steekproefgemiddelde afligt van $51$ (zowel ver daaronder als ver daarboven), hoe waarschijnlijker dat de alternatieve hypothese waar is.

    Het kritieke gebied is dus van de vorm $(-\infty, g_1]$ en $[g_2, \infty)$, waarbij $g_1$ en $g_2$ kunnen worden berekend als volgt:
        \begin{align*}
            g_1 &= \mu - t \cdot \frac{s}{\sqrt{n}} \\
                &= 51 - 2,3646 \cdot \frac{3,0553}{\sqrt{8}}\\
                &\approx 48,4457 \\
            g_2 &= \mu + t \cdot \frac{s}{\sqrt{n}} \\
                &= 51 + 2,3646 \cdot \frac{3,0553}{\sqrt{8}} \\
                &\approx 53,5543 \\    
        \end{align*}

    Het kritieke gebied is dus gelijk aan $(-\infty; 48,4457]$ en $[53,5543, \infty)$.

    Het geobserveerde steekproefgemiddelde $\overline{x}=40$ ligt in het kritieke gebied, dus verwerpen we de nulhypothese $H_0$.
    Er is voldoende reden om aan te nemen dat het gemiddelde aantal klachten per week significant anders is dan $51$.
}