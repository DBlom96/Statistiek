\question{9.7}{
    Capsules die zijn gevuld met een bepaald medicijn, moeten $5$ mg werkzaam bestanddeel bevatten.
    Het is bekend dat door onnauwkeurigheden met de machine die de capsules vult de hoeveelheid werkzaam bestanddeel te beschouwen is als een normaal verdeelde kansvariabele $X$ met verwachtingswaarde $5,0$ mg en standaarddeviatie $0,15$ mg.
    Ge\"eist wordt dat de hoeveelheid werkzaam bestanddeel per capsule tussen $4,6$ en $5,4$ mg ligt.
}
\begin{enumerate}[label=(\alph*)]
    \item Hoeveel procent van de capsules heeft een inhoud buiten de gestelde normen als de vulmachine correct is ingesteld?
    \answer{
    
    }

    \item De instelling van de machine kan tijdens het gebruik veranderen.
    Daarom wordt er regelmatig een aantal capsules gecontroleerd in het laboratorium.
    Een steekproef van $25$ capsules levert een gemiddeld gehalte van het werkzame bestanddeel van $4,70$ mg.
    Toets of hieruit mag worden geconcludeerd dat de instelling van de machine is gewijzigd.
    Toets hierbij tweezijdig, kies $\alpha=0,01$ en ga ervan uit dat de standaarddeviatie niet is veranderd.
    \answer{
        
    }

    \item Als de capsules worden gevuld met gemiddelde $4,70$ mg werkzaam bestanddeel ($\sigma$ is nog steeds gelijk aan $0,15$ mg), hoeveel procent van de capsules voldoet dan niet meer aan de norm?
    \answer{
        
    }
\end{enumerate}