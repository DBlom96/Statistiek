\question{9.m4}{Als bij een toetsingsprocedure wordt gewerkt met de zogeheten overschrijdingskansen $\mu$, dan $\ldots$}
\begin{enumerate}[label=(\alph*)]
    \item dient de nulhypothese te worden verworpen als $\mu$ groter is dan $\alpha$.
    \item dient de nulhypothese te worden verworpen als bij tweezijdige toetsing de $p$-waarde kleiner is dan $\frac{1}{2}\alpha$.
    \item is $1-p$ het onderscheidingsvermogen.
    \item mag $p$ niet groter zijn van $0,05$.
\end{enumerate}
\answer{Het juiste antwoord is (b).}