% !TEX TS-program = xelatex
% !TEX encoding = UTF-8 Unicode

\documentclass[11pt,a4paper,dutch]{article} % document type and language

\usepackage[centering,noheadfoot,margin=0.7in]{geometry}
\usepackage[dutch]{babel}
\usepackage{palatino}  
\usepackage{booktabs}
\usepackage{array}
\usepackage{multirow}
\usepackage{float}
\usepackage{bm}
\setlength{\parindent}{0cm} 

\usepackage{amsmath, amsfonts, amssymb, amsthm}

\title{Formuleblad Statistiek deel 1 (2024-2025)}
\author{}
\date{}

\begin{document}
\maketitle

Gegeven is een steekproef met $n$ uitkomsten $x_1, x_2, \ldots, x_n$.

\textbf{Steekproefgemiddelde:}
\[
    \overline{x} = \frac{\sum\limits_{i} x_i}{n} = \frac{x_1 + x_2 + \ldots + x_n}{n}
\]

\textbf{Steekproefvariantie:}
\begin{align}
    s^2 &= \frac{\sum\limits_{i} (x_i-\overline{x})^2}{n} = \frac{(x_1-\overline{x})^2 + (x_2-\overline{x})^2 + \ldots + (x_n - \overline{x})^2}{n} \hfill \tag{optie 1} \\%= \frac{(x_1-\overline{x})^2 + (x_2-\overline{x})^2 + \ldots + (x_n - \overline{x})^2}{n - 1}
    s^2 &= \frac{\left(\sum\limits_{i} x_i^2\right) - n \cdot \overline{x}^2}{n} = \frac{(x_1^2+x_2^2+\ldots+x_n^2) - n \cdot \overline{x}^2}{n} \hfill \tag{optie 2}
\end{align}
\vspace{.5cm}

\textbf{Rekenregels kansrekening:}
\begin{align*}
    P(A \text{ of } B)      &= P(A) + P(B) - P(A \text{ en } B) \tag{optelregel}\\
    P(B)                    &= 1 - P(\text{niet } B) \tag{complementregel}\\
    P(A \mid B)         &= \frac{P(A \text{ en } B)}{P(B)} \tag{conditionele kansen}\\
\end{align*}

\textbf{Discrete en continue kansverdelingen:}
\begin{table}[H]
    \centering
    \begin{tabular}{m{3.8cm}|m{6cm}|m{5.7cm}}
        \toprule
                                                            & \textbf{Discrete kansvariabelen}  & \textbf{Continue kansvariabelen} \\ 
        \midrule
            \textbf{Uitkomstenruimte:}                      & Eindig / aftelbaar oneindig   & Overaftelbaar oneindig \\
        \midrule
            \textbf{Toepassingen:}                          & Tellen / categoriseren        & Meten \\
        \midrule
            \textbf{Kansbegrip:}                            & Kansfunctie $p(k) = P(X=k)$   & Kansdichtheidsfunctie $f(x)$ \\
        \midrule
            \textbf{CDF:}                                   & $F(k) = P(X \le k) = \sum_{\ell:\ell\le k} p(\ell)$ & $F(x) = P(X \le x) = \int_{-\infty}^{x} f(y)\ dy$ \\
        \midrule
            \textbf{Verwachtingswaarde:}                    & $E[X] = \sum_{k} k \cdot P(X=k)$ & $E[X] = \int x \cdot f(x)\ dx$\\
        \midrule
            \textbf{Variantie:}                             & $\text{Var}(X) = \sum_{k} (k - E[X])^2 \cdot P(X=k)$   & $\text{Var}(X) = \int (x - E[X])^2 \cdot f(x)\ dx$ \\    
        \midrule
            \textbf{Standaardafwijking:}                    & $\sigma(X) = \sqrt{\text{Var}(X)}$   & $\sigma(X) = \sqrt{\text{Var}(X)}$ \\
        \bottomrule
    \end{tabular}
% \caption{Discrete and Continuous Probability Distributions with Mean and Variance}
% \label{tab:prob_dist}
\end{table}

\newpage
\textbf{Verwachtingswaarde en variantie van veelgebruikte kansverdelingen:}
\begin{table}[H]
    \centering
    \begin{tabular}{c|p{4.5cm}|c|c|c}
        \toprule
            \textbf{Verdeling} & {\bfseries Kans(dichtheids)functie} & \textbf{CDF} & $E(X)$ & $\textrm{Var}(X)$ \\ 
        \midrule
            \multicolumn{5}{c}{\textbf{Discreet}} \\ 
        \midrule
            Uniform($a, b$) & $p(k) = \frac{1}{b-a+1}$ \newline $(k=a,a+1,\ldots,b)$ & $F(k) = \begin{cases} 
            0 & x < a \\
            \frac{k - a + 1}{b-a+1} & a \leq k < b \\
            1 & k \geq b
            \end{cases}$ & $\frac{a+b}{2}$ & $\frac{(b-a+1)^2-1}{12}$ \\ 
        \midrule
        %     Bernoulli($p$) & $p(k) = \begin{cases} 1-p, & \text{$k=0$} \\ p, & \text{$k=1$} \end{cases}$ & $F(k) = \begin{cases} 
        %     0 & k < 0 \\
        %     1-p & k=0 \\
        %     1 & k \geq 1
        %     \end{cases}$ & $p$ & $p(1-p)$ \\ 
        % \midrule
            Binomiaal($n,p$) & $p(k) = \binom{n}{k}p^k(1-p)^{n-k}$ & $F(k) = \sum\limits_{i=0}^k \binom{n}{i}p^i(1-p)^{n-i}$ & $np$ & $np(1-p)$ \\ 
        \midrule
            % Geo($p$) & $P(X=k) = (1-p)^{k-1}\cdot p$ & $F(k) = \sum_{i=1}^k(1-p)^{k-1}\cdot p$ & $\frac{1}{p}$ & $\frac{1-p}{p^2}$ \\ \midrule
            Poisson($\lambda$) & $p(k) = e^{-\lambda}\cdot \frac{\lambda^k}{k!}$ & $F(k) = \sum\limits_{i=0}^k e^{-\lambda} \cdot \frac{\lambda^i}{i!}$ & $\lambda$ & $\lambda$ \\ 
        \midrule
            \multicolumn{5}{c}{\textbf{Continuous}} \\ 
        \midrule
            Uniform($a, b$) & $f(x) = \begin{cases} \frac{1}{b-a}, & a \leq x \leq b \\ 0, & \text{elders.} \end{cases}$ & $F(x) = \begin{cases} 
            0, & x < a \\
            \frac{x-a}{b-a}, & a \leq x < b \\ 1, & x \geq b \end{cases}$ & $\frac{a+b}{2}$ & $\frac{(b-a)^2}{12}$ \\ 
        \midrule
        %     $N(\mu, \sigma)$ & $f(x) = \frac{1}{\sigma\cdot\sqrt{2\pi}} e^{-\frac{1}{2}\left(\frac{x-\mu}{\sigma}\right)^2}$ & $F(x)=\Phi\left(\frac{x-\mu}{\sigma}\right)$ & $\mu$ & $\sigma^2$ \\ 
        % \midrule
            Exponentieel($\lambda$) & $f(x) = \lambda e^{-\lambda x}$, $x \geq 0$ & $F(x) = \begin{cases} 1 - e^{-\lambda x}, & x \geq 0 \\ 0, & x < 0 \end{cases}$ & $\frac{1}{\lambda}$ & $\frac{1}{\lambda^2}$ \\ 
        \bottomrule
    \end{tabular}
\end{table}

\textbf{Veelgebruikte functies op de grafische rekenmachine}
\begin{table}[H]
    \centering
    \begin{tabular}{c|c|c}
        \toprule
            \textbf{Type vraag}                     & \textbf{TI-84 Plus}           & \textbf{Casio} \\
        \midrule
            \multicolumn{3}{c}{\textbf{Continue kansverdeling (willekeurig)}} \\ 
        \midrule
            $P(a \le X \le b)$                      & $\int_{a}^{b} f(x)\, dx$      & $\int_{a}^{b} f(x)\, dx$\\
        \midrule    
            \multicolumn{3}{c}{\textbf{$X \sim \text{Binomiaal}(n, p)$}} \\ 
        \midrule
            $P(X = k)$                              & binompdf$(n,p,k)$             & BinomialPD$(k, n, p)$ \\
            $P(X \le k)$                            & binomcdf$(n,p,k)$             & BinomialCD$(k, n, p)$ \\
        \midrule    
            \multicolumn{3}{c}{\textbf{$X \sim N(\mu, \sigma)$}} \\
        \midrule 
            $P(a \le X \le b)$                      & normalcdf$(a,b,\mu,\sigma)$   & NormalCD$(a, b, \sigma, \mu)$ \\
            Grenswaarde $g$ zodat $P(X \le g)=p$?   & invNorm$(p,\mu,\sigma)$       & InvNormCD$(\text{tail=left}, p, \sigma, \mu)$ \\
        \midrule 
            \multicolumn{3}{c}{\textbf{$X \sim \text{Poisson}(\lambda)$}} \\  
        \midrule
            $P(X = k)$                              & poissonpdf$(\lambda, k)$      & PoissonPD$(k, \lambda)$ \\
            $P(X \le k)$                            & poissoncdf$(\lambda, k)$      & PoissonCD$(k, \lambda)$ \\
        \bottomrule
    \end{tabular}
\end{table}
% \begin{itemize}
%     \item Continue kansvariabelen met kansdichtheidsfunctie $f$:
%     \[
%         \int_{a}^{b} f(x)\, dx = \text{fnInt}(f(x); x; a; b)
%     \]
%     \item Binomiaal verdeelde kansvariabele $X \sim \text{Binomiaal}(n; p)$:
%     \begin{itemize}
%         \item $P(X = k) = \text{binompdf}(n; p; k)$ en $P(X \le k) = \text{binomcdf}(n; p; k)$
%     \end{itemize}
%     \item Normaal verdeelde kansvariabele $X \sim N(\mu; \sigma)$:
%     \begin{itemize}
%         \item $P(a \le X \le b) = \text{normalcdf}(a; b; \mu; \sigma)$ en $P(X \le g) = p \rightarrow g = \text{invNorm}(p; \mu; \sigma)$
%     \end{itemize}
%     \item Poisson verdeelde kansvariabele $X \sim \text{Poisson}(\lambda)$:
%     \begin{itemize}
%         \item $P(X = k) = \text{poissonpdf}(\lambda; k)$ en $P(X \le k) = \text{poissoncdf}(\lambda; k)$
%     \end{itemize}
% \end{itemize}

\textbf{$z$-score:}
\[
    z = \frac{x - \mu}{\sigma}
\]

\textbf{Centrale limietstelling:}
Gegeven $n$ kansvariabelen $X_1, X_2, \ldots, X_n$ die onderling onafhankelijk zijn en dezelfde kansverdeling hebben met een verwachtingswaarde $\mu$ en standaardafwijking $\sigma$, dan geldt (bij benadering) dat
\begin{itemize}
    \item de som $\sum X = X_1 + X_2 + \ldots + X_n$ normaal verdeeld is met verwachtingswaarde $n \cdot \mu$ en standaardafwijking $\sqrt{n} \cdot \sigma$.
    \item het gemiddelde $\overline{X} = \frac{X_1 + X_2 + \ldots + X_n}{n}$ normaal verdeeld is met verwachtingswaarde $\mu$ en standaardafwijking $\frac{\sigma}{\sqrt{n}}$.
\end{itemize}

% \textbf{Covariance:}
% \[
%     Cov(X, Y) = E[(X-E[X])\cdot(Y-E[Y])] = E[X\cdot Y] - E[X]E[Y]
% \]

% \textbf{Correlation coefficient:}
% \[
%     \rho(X, Y) = \frac{Cov(X,Y)}{\sqrt{Var(X) \cdot Var(Y)}}
% \]

% \textbf{Variance of error terms:}
% \[ \hat{\sigma}^2 = \frac{1}{n-2}\sum_{i=1}^{n} (y_{i} - \hat{y}_{i})^2 = \frac{\sum_{i=1}^{n} y_{i}^2 - \hat{\beta}_{0}\cdot \sum_{i=1}^{n} y_{i} - \hat{\beta}_{1}\cdot \sum_{i=1}^{n} x_{i}y_{i}}{n-2}\]

% \textbf{Coefficients of regression line:}

% Given a sample of $n$ observations $(x_1, y_1), (x_2, y_2), \ldots, (x_n, y_n)$, the regression line $\hat{y} = \hat{\beta}_0 + \hat{\beta}_{1}\cdot x$ satisfies:
% \begin{itemize}
%     \item $\hat{\beta}_{1} = \frac{\sum_{i=1}^{n}x_iy_i - n \cdot \overline{x} \cdot \overline{y}}{\sum_{i=1}^{n} x_{i}^2 - n \cdot \overline{x}^2}$
%     \item $\hat{\beta}_{0} = \overline{y} - \hat{\beta}_{1} \cdot \overline{x}$
% \end{itemize}

% \textbf{Sample correlation coefficient:}
% \[ r(X, Y) = \frac{Cov(X, Y)}{\sqrt{s_{X}^2\cdot s_{Y}^2}} = \frac{\sum_{i=1}^{n}(x_{i}-\overline{x})\cdot(y_{i} - \overline{y})}{\sqrt{\sum_{i=1}^{n}(x_{i}-\overline{x})^2 \cdot \sum_{i=1}^{n}(y_{i}-\overline{y})^2}} = \frac{\sum_{i=1}^{n}x_{i}y_{i} - n \cdot \overline{x}\cdot \overline{y}}{\sqrt{\left(\sum_{i=1}^{n}x_{i}^2-n\cdot \overline{x}^2\right) \cdot \left(\sum_{i=1}^{n}y_{i}^2-n\cdot \overline{y}^2\right)}}\]

% \textbf{Likelihood function:}
% \[ L(x_1, x_2, \ldots, x_n; \theta) = f(x_1; \theta)\cdot f(x_2; \theta)\cdot\ldots\cdot f(x_n; \theta)\]

% \textbf{Minimum sample size with given accuracy $E$:}
% \[ n \ge \left(\frac{z_{\frac{\alpha}{2}} \cdot \sigma}{E}\right)^2\]

% \textbf{Minimum sample size for bounding type II error $\beta$}
% \[ n \ge \frac{(z_{\frac{\alpha}{2}}+z_{\beta})^2 \cdot \sigma^2}{\delta^2}\]

\end{document}
