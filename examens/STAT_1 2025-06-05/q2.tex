\begin{question}{25}{
    Tijdens een militaire missie worden Panzerhaubitzes 2000 ingezet voor langdurige operaties.
    Een kritisch onderdeel, de elevatiemotoren van de kanonnen, heeft een gemiddelde uitvalfrequentie van 1,5 storingen per maand.
    Gedurende de missie wordt gebruik gemaakt van een logistiek transport voor het leveren van reserveonderdelen (spare parts) voor de Panzerhaubitzes.
}

    \subquestion{5}{
        Wat is de kans dat er in een periode van zes maanden precies tien storingen optreden, aangenomen dat de uitvallen volgens een Poissonproces plaatsvinden?
    }
    \solution{
        Laat $X$ het aantal storingen aan de elevatiemotor van het kanon zijn in een periode van zes maanden.
        De tijdseenheid is in maanden, dus $t = 6$ en $\lambda = 1,5$, oftewel $\mu = \lambda \cdot t = 9$. \rubric{2}

        Er geldt dus dat $X \sim \text{Poisson}(\mu=9)$. \rubric{1}

        De kans dat er in die zes maanden precies tien storingen plaatsvinden is dus gelijk aan
        \[
            P(X = 10) = \poissoncdf(\mu=9; k=10) \approx 0,1186. \rubric{2}
        \]
    }

    \subquestion{4}{
        De logistieke eenheid die verantwoordelijk is voor het transport heeft een transportcapaciteit van 12 reservemotoren per zes maanden.
        Wat is de kans dat er na aankomst van het logistieke transport een tekort aan benodigde reserveonderdelen blijkt te zijn?
    }

    \solution{
        Er gaat een tekort aan reserveonderdelen zijn als geldt dat er meer storingen optreden dan het aantal reserveonderdelen dat kan worden getransporteerd? \rubric{1}
        We willen dus de kans $P(X > 12)$ bepalen, oftewel
        \begin{align*}
            P(X > 12) = P(X \ge 13) = 1 - P(X \le 12)   &= 1 - \poissoncdf(\mu=9; k=12) \\
                                                        &\approx 0,1242. \rubric{3}
        \end{align*}
    }

    % \subquestion{3}{
    %     De elevatiemotor van een Panzerhaubitze 2000 is precies \'e\'en maand geleden vervangen.
    %     Hoe groot is de kans dat deze nog een maand zonder problemen blijft functioneren?
    % }
    % \solution{
    %     Laat $T$ de tijd meten vanaf de vervanging van de motor tot de eerste storing.
    %     Aangezien storingen plaatsvinden volgens een Poissonproces, geldt dat
    %     \[
    %         P(T > 2 \mid T > 1) = \frac{P(T > 2 \text{ en } T > 1)}{P(T > 1)} = \frac{P(T>2)}{P(T>1)} = \frac{e^{-1.5\cdot 2}}{e^{-1,5\cdot 1}} = \frac{1}{e\sqrt{e}} \approx 0,2231. \rubric{3}
    %     \]
    %     \textbf{Alternatief}: aangezien $T$ een exponentieel verdeelde kansvariabele is, hebben we te maken met geheugenloosheid.\rubric{1}
    %     De kans dat deze nog een maand zonder problemen blijft functioneren is
    %     \[
    %         P(T > 1) = e^{-\lambda\cdot 1} = e^{-1,5\cdot 1} = e^{-1,5} \approx 0,2231. \rubric{2} 
    %     \]
    % }   

    \subquestion{7}{
        De eenheid overweegt twee opties om de kans op een tekort aan reserveonderdelen te minimaliseren:
        \begin{enumerate}
            \item het doorvoeren van technologische upgrades die de uitvalfrequentie met $10\%$ verminderen.
            \item een verhoging van de transportcapaciteit naar 13 reservemotoren per zes maanden.
        \end{enumerate}
        Welke maatregel is het meest effectief in het verlagen van de kans op een tekort aan reserveonderdelen?
        Beargumenteer je antwoord aan de hand van berekeningen.
    }
    \solution{
        Zodra optie 1 wordt doorgevoerd, wordt de uitvalfrequentie met $10\%$ vermindert, oftewel de gemiddelde uitvalfrequentie per maand daalt van $1,5$ naar $0,9 \cdot 1,5 = 1,35$ storingen.\rubric{1}
        Per zes maanden geeft dit $\mu = \lambda \cdot t = 1,35 \cdot 6 = 8,1$ \rubric{1}.
        De kans op een tekort aan reserveonderdelen is in dat geval:
        \begin{align*}
            P(X > 12) = P(X \ge 13) = 1 - P(X \le 12)   &= 1 - \poissoncdf(\mu=8,1; k=12) \\
                                                        &\approx 0,0687. \rubric{2}
        \end{align*}

        Zodra optie 2 wordt doorgevoerd, wordt de transportcapaciteit verhoogt tot 13 reservemotoren per zes maanden.
        De kans op een tekort aan reserveonderdelen is in dat geval:
        \begin{align*}
            P(X > 13) = P(X \ge 14) = 1 - P(X \le 13)   &= 1 - \poissoncdf(\mu=9; k=13) \\
                                                        &\approx 0,0739. \rubric{2}
        \end{align*}

        Aangezien de kans op een tekort kleiner is als optie 1 wordt doorgevoerd ($0,0687 < 0,0739$), is deze optie ook een effectievere maatregel om het tekort te minimaliseren. \rubric{2}
    }

    \subquestion{9}{
        Stel nu dat er niet elke zes maanden een transport is van 12 reserveonderdelen, maar elke maand een transport van twee onderdelen.
        Het aantal storingen volgt nog steeds een Poissonproces met gemiddelde $\lambda=1,5$ per maand.
        Wat is het verwachte aantal reserveonderdelen van een enkel transport (dus na een maand) dat gebruikt moet worden om storingen mee op te lossen?
        Ga hierbij er van uit dat de huidige reservevoorraad leeg is.
    }
    \solution{
        Laat $Y$ het aantal reserveonderdelen zijn dat gebruikt moet worden op storingen mee op te lossen.
        De uitkomstenruimte van $Y$ bestaat uit $0, 1, 2$, aangezien er maximaal 2 reservemotoren gebruikt kunnen worden als een transport bestaat uit twee reservemotoren.\rubric{2}
        Deze kansvariabele $Y$ heeft de volgende kansfunctie:
        \begin{center}
            \begin{tabular}{cc}
                \toprule
                    {\bfseries Uitkomst $k$} & $P(Y=k)$\\
                \cmidrule{1-1} \cmidrule{2-2}
                    $0$ & $P(X=0) = \poissonpdf(\lambda=1,5; k=0) \approx 0,2231$ \\
                    $1$ & $P(X=1) = \poissonpdf(\lambda=1,5; k=1) \approx 0,3347$ \\
                    $2$ & $P(X\ge 2) = 1 - \poissoncdf(\lambda=1,5; k=1) \approx 0,4422$ \rubric{3} \\ 
                \bottomrule
            \end{tabular}
        \end{center}
        Het verwachte aantal reservemotoren dat wordt gebruikt om storingen mee te verhelpen is dan gelijk aantal
        de verwachtingswaarde van deze kansvariabele $Y$, oftewel:
        \begin{align*}
            E[Y]    &= 0 \cdot P(Y=0) + 1 \cdot P(Y=1) + 2 \cdot P(Y=2) \\
                    &\approx 0 \cdot 0,2231 + 1 \cdot 0,3347 + 2 \cdot 0,4422 \\
                    &\approx 1,2191 \rubric{3}
        \end{align*}
        Het verwachte aantal reservemotoren dat moet worden gebruikt om storingen te verhelpen is dus gelijk aan $1,2191$. \rubric{1}
    }
\end{question}