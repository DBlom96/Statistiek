\begin{question}{25}{
    Voor het vergaren van inlichtingen via de lucht worden zwermen van surveillance UAV's ingezet in vijandelijk gebied.
    De tijd $T$ (in minuten) totdat een zwerm gedetecteerd wordt, is een kansvariabele met de volgende kansdichtheidsfunctie:
    \[
        f(t) = \begin{cases} 
                    \frac{4}{27}\cdot t^2\cdot (3-t), &\text{ als } 0 \le t \le 3, \\
                                       0, &\text{ anders.} 
                \end{cases}
    \]
}
        \subquestion{4}{
            Toon met berekeningen (met de grafische rekenmachine) aan dat $f$ inderdaad voldoet aan de twee voorwaarden voor een kansdichtheidsfunctie.
        }
        \solution{
            Een functie $f$ kan dienen als een kansdichtheidsfunctie als geldt:

            \textbf{Voorwaarde 1:} de functie $f$ is niet-negatief voor alle waarden van $t$, oftewel \[f(t) \ge 0.\]
            
            Op het interval $0 \le t \le 3$ geldt dat $f(t) = \frac{4}{27}\cdot t^2\cdot (3-t)$.
            In dat geval geldt dat $t^2 \ge 0$ en $3 - t \ge 0$, en we hebben te maken met een vermenigvuldiging van niet-negatieve termen, dus $f(t) \ge 0$.
            Buiten het interval $0 \le t \le 3$ is $f(t) = 0$, dus is ook voldaan aan $f(t) \ge 0$. \rubric{2}

            \textbf{Voorwaarde 2:} de oppervlakte onder de grafiek van $f$ is gelijk aan 1, oftewel \[\int_{-\infty}^{\infty} f(t)\, dt=1.\]         
            
            Voor de tweede voorwaarde moeten we dus checken of de oppervlakte onder de grafiek van $f$ daadwerkelijk gelijk is aan 1:
            \begin{align*}
                \int_{-\infty}^{\infty} f(t)\, dt   &= \int_{0}^{3} \frac{4}{27}\cdot t^2 \cdot (3 - t)\, dt \\
                                                    &= \text{fnInt}(\frac{4}{27}\cdot x^2 \cdot (3 - x); x; 0; 3) \\
                                                    &= 1. \rubric{2}
            \end{align*} 
            Hiermee is aangetoond dat de gegeven functie $f$ voldoet aan de twee voorwaarden voor kansdichtheidsfuncties.
        }

        \subquestion{8}{
            Bereken de verwachtingswaarde en de standaardafwijking van $T$.
            Laat hierbij je berekeningen zien zonder gebruik te maken van het statistische menu van de grafische rekenmachine.
        }
        \solution{
            De verwachtingswaarde $E[T]$ berekenen we door middel van
            \begin{align*}
                E[T] = \int_{-\infty}^{\infty} t\cdot f(t)\, dt &= \int_{0}^{3} t \cdot \frac{4}{27}\cdot t^2 \cdot (3 - t)\, dt \\
                                                                &= \int_{0}^{3} \frac{4}{27}\cdot t^3 \cdot (3 - t)\, dt \\   
                                                                &= \text{fnInt}(\frac{4}{27}\cdot t^3 \cdot (3 - t); t; 0; 3) \\
                                                                &= 1,8. \rubric{3}          
            \end{align*}
            De standaardafwijking $\sigma(T)$ berekenen we door eerst de variantie $\Var{T}$ te bepalen, en van het resultaat de wortel te nemen:
            \begin{align*}
                \Var{T} = \int_{-\infty}^{\infty} (t-E[T])^2 \cdot f(t)\, dt    &= \int_{0}^{3} (t-1,8)^2 \cdot \frac{4}{27}\cdot t^2 \cdot (3 - t)\, dt \\
                                                                                &= \text{fnInt}(\frac{4}{27}\cdot (t-1,8)^2 \cdot t^2 \cdot (3 - t); t; 0; 3) \\
                                                                                &= 0,36. \rubric{3}
            \end{align*} 
            \begin{align*}
                \sigma(T) = \sqrt{\Var{T}} = \sqrt{0,36} = 0,6. \rubric{2}
            \end{align*}
        }

        \subquestion{4}{
            Bereken de mediaan van $T$.\\
            {\scriptsize \textbf{Hint:} de mediaan van een kansverdeling is de uitkomst $m$ waarvoor geldt dat $P(T \le m) = P(T \ge m) = 0,5$.}
        }
        \solution{
            Om de mediaan van $T$ te vinden moeten we een grenswaarde $m$ bepalen waarvoor geldt $P(T \le m) = \int_{-\infty}^{m} f(t)\, dt = 0,5$. \rubric{1}
            Deze vergelijking kunnen we oplossen door het functiemenu te openen en in te voeren:
            \begin{align*}
                y_1 &= \text{fnInt}(\frac{4}{27} \cdot t^2 \cdot (3 - t); t; 0; m) \\
                y_2 &= 0,5  \rubric{2}
            \end{align*}
            De optie ``intersect'' geeft een waarde van $m \approx 1,8428$. \rubric{1}

            }

        \subquestion{4}{
            Bereken het percentage dronezwermen dat binnen \'e\'en minuut wordt gedetecteerd.
        }
        \solution{
            We starten door de kans $P(T < 1)$ te berekenen:
            \[
                P(T < 1) = \int_{-\infty}^{1} f(t)\, dt = \int_{0}^{1} \frac{4}{27} \cdot t^2 \cdot (3-t)\, dt = \text{fnInt}(\frac{4}{27} \cdot t^2 \cdot (3-t); t; 0; 1) \approx 0,1111. 
            \]\rubric{3}
            Omrekenen naar percentages geeft dat $11,11\%$ van de dronezwermen binnen \'e\'en minuut wordt gedetecteerd. \rubric{1}
        }

        \subquestion{5}{
            Stel dat op een gegeven moment zes dronezwermen tegelijkertijd naar het vijandelijke gebied worden doorgestuurd.
            De missie is succesvol als er tenminste een zwerm drones is die meer dan \'e\'en minuut onopgemerkt blijft.
            Hoe groot is de kans op succes in deze surveillancemissie?
        }
        \solution{
            Laat $Y$ nu het aantal dronezwermen zijn dat meer dan \'e\'en minuut onopgemerkt blijft.
            In dat geval is $Y$ een binomiaal verdeelde kansvariabele met parameters $n = 6$ (aantal dronezwermen) en $p \approx 0,1111$ (kans dat een dronezwerm langer dan een minuut onopgemerkt blijft).\rubric{2}
            De kans op een succesvolle missie is gelijk aan de kans dat minstens een zwerm drones langer dan \'e\'en minuut onopgemerkt blijft, oftewel
            \[
                P(Y \ge 1) = 1 - P(Y=0) = 1 - \binompdf(n=6; p=0,1111; k=0) = 0,5067. \rubric{3}
            \]
        }
\end{question}