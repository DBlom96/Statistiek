\begin{question}{25}{
    Voor het vergaren van inlichtingen via de lucht worden zwermen van surveillance UAV's ingezet in vijandelijk gebied.
    De tijd $T$ (in minuten) totdat een zwerm gedetecteerd wordt is een kansvariabele met de volgende kansdichtheidsfunctie:
    \[
        f(t) = \begin{cases} 
                    \frac{4}{27}\cdot t^2\cdot (3-t), &\text{ als } 0 \le t \le 3, \\
                                       0, &\text{ anders.} 
                \end{cases}
    \]
}

        \subquestion{4}{
            Toon met berekeningen (met de grafische rekenmachine) aan dat $f$ inderdaad voldoet aan de twee voorwaarden voor een kansdichtheidsfunctie.
        }
        \solution{
            Een functie $f$ kan dienen als een kansdichtheidsfunctie als geldt:
            \begin{enumerate}
                \item De functie $f$ is niet-negatief voor alle waarden van $t$, oftewel $f(t) \ge 0$.
                \item De oppervlakte onder de grafiek van $f$ is gelijk aan 1, oftewel $\int_{-\infty}^{\infty} f(t)\, dt$
            \end{enumerate}
            De functie $f$ is gedefinieerd in twee gevallen.
            Op het interval $0 \le t \le 3$ geldt dat $f(t) = \frac{4}{27}\cdot t^2\cdot (3-t)$.
            In dat geval geldt dat $t^2 \ge 0$ en $3 - t \ge 0$, en we hebben te maken met een vermenigvuldiging van niet-negatieve termen, dus $f(t) \ge 0$.
            Buiten het interval $0 \le t \le 3$ is $f(t) = 0$, dus is ook voldaan aan $f(t) \ge 0$. \rubric{2}
            
            Voor de tweede voorwaarde moeten we checken of de oppervlakte onder de grafiek van $f$ daadwerkelijk gelijk is aan 1:
            \begin{align*}
                \int_{-\infty}^{\infty} f(t)\, dt   &= \int_{0}^{3} \frac{4}{27}\cdot t^2 \cdot (3 - t)\, dt \\
                                                    &= \text{fnInt}(\frac{4}{27}\cdot x^2 \cdot (3 - x); x; 0; 3) \\
                                                    &= 1.
            \end{align*} \rubric{2}
            Hiermee is aangetoond dat de gegeven functie $f$ voldoet aan de twee voorwaarden voor kansdichtheidsfuncties.
        }

        \subquestion{8}{
            Bereken de verwachtingswaarde en de standaardafwijking van $T$.
            Laat hierbij je berekeningen zien zonder gebruik te maken van het statistische menu van de grafische rekenmachine.
        }
        \solution{

        }

        \subquestion{4}{
            Bereken de mediaan van $T$.\\
            {\scriptsize \textbf{Hint:} de mediaan van een kansverdeling is de uitkomst $g$ waarvoor geldt dat $P(T \le g) = P(T \ge g) = 0,5$.}
        }
        \solution{

        }

        \subquestion{4}{
            Bereken het percentage dronezwermen dat binnen \'e\'en minuut wordt gedetecteerd.
        }
        \solution{

        }

        \subquestion{5}{
            Stel dat op een gegeven moment zes dronezwermen tegelijkertijd naar het vijandelijke gebied worden doorgestuurd.
            De missie is succesvol als er tenminste een zwerm drones is die meer dan \'e\'en minuut onopgemerkt blijft.
            Hoe groot is de kans op succes in deze surveillancemissie?
        }
        \solution{

        }


    

\end{question}