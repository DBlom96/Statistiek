\begin{question}{29}{
    De chauffeurs van de Transportgroep Defensie rijden met busjes tussen verschillende kazernes om kantoorartikelen te verplaatsen.
    Een van de chauffeurs, meneer de Wolf, heeft de specifieke taak gekregen om wekelijks tussen de KMA en het KIM te pendelen om bibliotheekboeken en IT-apparatuur te vervoeren. 
    Zijn reistijd in minuten (enkele reis) kan worden beschouwd als een normaal verdeelde kansvariabele $T$ met een gemiddelde $\mu = 140$ minuten en standaardafwijking $\sigma = 16$ minuten.
}

    \subquestion{4}{
        Bereken de kans dat hij in een willekeurige week er langer dan $2\frac{1}{2}$ uur over doet om van de KMA naar het KIM te rijden.
    }
    \solution{
        In een willekeurige week is zijn reistijd $T$ normaal verdeeld met gemiddelde $\mu = 140$ en standaardafwijking $\sigma=12$ minuten.
        De grenswaarde van $2\frac{1}{2}$ uur is omgerekend naar minuten gelijk aan 150 minuten. \rubric{1}
        
        De kans dat meneer de Wolf langer dan $2\frac{1}{2}$ uur erover doet om van de KMA naar het KIM te rijden is dus gelijk aan
        \begin{align*}
            P(T > 150) = \normalcdf(a=150; b=10^{99}; \mu=140; \sigma=16) \approx 0,2660. \rubric{3}
        \end{align*}
    }

    \subquestion{6}{
        Bereken (met behulp van je antwoord op vraag 2a) de kans dat hij in een half jaar (26 weken) minstens 10 keer langer dan $2\frac{1}{2}$ uur doet over zijn busrit.
    }
    \solution{
        Reistijden in verschillende werken kunnen we als onafhankelijk van elkaar beschouwen, omdat de reistijd in de ene week niets zegt over hoe lang je de week erna onderweg bent.\rubric{1}
        Het aantal keer $N$ dat meneer de Wolf in een half jaar meer dan $2\frac{1}{2}$ uur erover doet om van de KMA naar het KIM te rijden is dus binomiaal verdeeld met $n = 26$ en $p = 0,2660$. \rubric{1}
        Dit geeft een kans voor minstens 10 keer langer dan $2\frac{1}{2}$ uur van
        \begin{align*}
            P(N \ge 10) = 1 - P(N \le 9)    &= 1 - \binomcdf(n=26; p=0,2660; k=9) \\
                                            &\approx 0,1272. \rubric{3}
        \end{align*}
        Met $12, 72\%$ kans zal hij in een half jaar minstens 3 keer langer dan $2\frac{1}{2}$ uur over zijn busrit doen.\rubric{1}

    }

    \subquestion{8}{
        Wat is de kans dat in een jaar (52 weken) het gemiddelde van de reistijden van de 52 busritten groter is dan 2 uur en een kwartier?
    }
    \solution{
        Zoals gezegd zijn de reistijden in verschillende weken onafhankelijk van elkaar.
        Omdat de reistijden onafhankelijk zijn en normaal verdeeld met dezelfde $\mu$ en $\sigma$, kunnen we de centrale limietstelling gebruiken. \rubric{1}
        Die zegt dat de gemiddelde reistijd $\overline{T} = \frac{T_1+\ldots+T_{52}}{52}$ over 52 weken ook normaal verdeeld is,
        met verwachtingswaarde $\mu=130$ en standaardafwijking $\frac{\sigma}{\sqrt{52}} = \frac{12}{\sqrt{52}}$. \rubric{2}
        
        De grenswaarde van $2$ uur en een kwartier is omgerekend naar minuten gelijk aan 135 minuten. \rubric{1}

        De kans dat de gemiddelde reistijd in 52 weken langer dan 2 uur en een kwartier is, is dus gelijk aan
        \begin{align*}
            P(\overline{T} \ge 135) = \normalcdf(a=135; b=10^{99}; \mu=140; \sigma=\frac{16}{\sqrt{52}}) \approx 0,9879 \rubric{3}
        \end{align*} 
        Met $98,79\%$ kans zal hij in een jaar gemiddeld langer dan $2$ uur en een kwartier doen over zijn busrit.\rubric{1}
    }

    \subquestion{7}{
        Elke dinsdag vertrekt meneer de Wolf rond 8.00 uur `s ochtends vanaf de KMA.
        Om precies te zijn is zijn vertrektijd $V$ (gemeten in minuten na middernacht) normaal verdeeld met gemiddelde $\mu=479$ minuten (oftewel $7.59$ uur) en standaardafwijking $\sigma=4$ minuten.
        Hoe groot is de kans dat hij voor 10.00 uur aankomt op het KIM, gegeven dat zijn busrit $127$ minuten duurt? 
    }
    \solution{
        De aankomsttijd $A$ is de som van twee kansvariabelen, namelijk $A = V + T$, waarbij $V$ is de vertrektijd en $T$ is de reistijd.
        De aankomsttijd is in dit geval $A = V + 127$.\rubric{1}
        Als hij v\'o\'or 10.00 uur aankomt op het KIM, dan betekent dat dat $A < 600$. \rubric{1}
        Omdat $A = V + 127$, betekent dat ook dat $V = A - 127 = 600 - 127 = 473$ minuten.\rubric{1}
        De kans dat meneer de Wolf v\'o\'or 10.00 uur aankomt op het KIM is dus gelijk aan de kans dat hij v\'o\'or 7.53 uur vertrekt, oftewel:
        \begin{align*}
            P(A < 600) = P(V < 473) &= \normalcdf(a=-10^{99}; b=473; \mu=479;\sigma=4) \\
                                    &\approx 0,0668 \rubric{3}
        \end{align*}
        Met kans $6,68\%$ komt meneer de Wolf v\'o\'or 10.00 uur aan op het KIM.\rubric{1}

    }

    \subquestion{4}{
        Hoe laat moet meneer de Wolf `s ochtends uiterlijk vertrekken om te zorgen dat hij met kans $0,95$ v\'o\'or $10.00$ uur op het KIM aankomt? Rond af op hele minuten.
    }

    \solution{
        We hebben een gegeven aankomsttijd $a = 10 \cdot 60 = 600$, dus de enige factor die er nu toe doet om een uiterste vertrektijd te berekenen is de reistijd $T$.
        Hiervoor berekenen we eerst de waarde $t$ van de reistijd dusdanig dat $P(T \le t) = 0,95$. \rubric{1}

        Dit kunnen we doen met behulp van de ``invNorm'' functie op de grafische rekenmachine.
        \[
            t = \invnorm(opp=0,95; \mu=140; \sigma=16) \approx 166,3177.\rubric{1}
        \]
        Met $95\%$ kans duurt de reis korter dan $167$ minuten (naar boven afgerond op hele minuten). \rubric{1}
        De uiterste vertrektijd $v$ vinden we aan de hand van $v = a - t = 600 - 167 = 433$ minuten.
        Omgerekend naar uren geeft dit een tijdstip van $7.13$ uur. \rubric{1}
    }

\end{question}