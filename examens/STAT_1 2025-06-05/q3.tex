\begin{question}{29}{
    De chauffeurs van de Transportgroep Defensie rijden met busjes tussen verschillende kazernes om kantoorartikelen te verplaatsen.
    Een van de chauffeurs, meneer de Wolf, heeft de specifieke taak gekregen om wekelijks tussen de KMA en het KIM te pendelen om bibliotheekboeken en IT-apparatuur te vervoeren. 
    Zijn reistijd in minuten (enkele reis) kan worden beschouwd als een normaal verdeelde kansvariabele $T$ met een gemiddelde $\mu = 140$ minuten en standaardafwijking $\sigma = 16$ minuten.
}

    \subquestion{4}{
        Bereken de kans dat hij in een willekeurige week er langer dan $2\frac{1}{2}$ uur over doet om van de KMA naar het KIM te rijden.
    }
    \solution{
        In een willekeurige week is zijn reistijd $T$ normaal verdeeld met gemiddelde $\mu = 130$ en standaardafwijking $\sigma=12$ minuten. \rubric{1}
        
        De kans dat meneer de Wolf langer dan $2\frac{1}{2}$ uur -- oftewel 150 minuten -- erover doet om van de KMA naar het KIM te rijden is gelijk aan
        \begin{align*}
            P(T > 150) = \normalcdf(a=150; b=10^{99}; \mu=130; \sigma=12) \approx 0,0478. \rubric{3}
        \end{align*}
    }

    \subquestion{5}{
        Bereken (met behulp van je antwoord op vraag 2a) de kans dat hij in een half jaar (26 weken) minstens 3 keer langer dan $2\frac{1}{2}$ uur doet over zijn busrit.
    }
    \solution{
        Reistijden in verschillende werken kunnen we als onafhankelijk van elkaar beschouwen, omdat de reistijd in de ene week niks zegt over hoe lang je de week erna onderweg bent.\rubric{1}
        Het aantal keer $N$ dat meneer de Wolf in een half jaar meer dan $2\frac{1}{2}$ erover doet om van de KMA naar het KIM te rijden is binomiaal verdeeld met $n = 26$ en $p = 0,0478$. \rubric{1}
        Dit geeft een kans voor minstens 3 keer langer dan $2\frac{1}{2}$ uur van
        \begin{align*}
            P(N \ge 3) = 1 - P(N \le 2) &= 1 - \binomcdf(n=26; p=0,0478; k=2) \\
                                        &\approx 0,1257. \rubric{3}
        \end{align*}
        Met $12,57\%$ kans zal hij in een half jaar minstens 3 keer langer dan $2\frac{1}{2}$ uur over zijn busrit doet.

    }

    \subquestion{6}{
        Hoe groot is de kans dat hij gedurende een jaar (52 weken) gemiddeld langer dan 2 uur en een kwartier doet over zijn busritten.
    }
    \solution{
        Zoals gezegd zijn de reistijden in verschillende weken onafhankelijk van elkaar.
        Omdat de reistijden onafhankelijk zijn en normaal verdeeld met dezelfde $\mu$ en $\sigma$, kunnen we volgens de centrale limietstelling zeggen dat de gemiddeld reistijd $\overline{T}$ over 52 weken ook normaal verdeeld is met verwachtingswaarde $\mu$ en standaardafwijking $\frac{\sigma}{\sqrt{52}}$.
        
    }

    \subquestion{8}{
        Elke dinsdag vertrekt meneer de Wolf rond 8.00 uur `s ochtends vanaf de KMA.
        Om precies te zijn is zijn vertrektijd $V$ normaal verdeeld met gemiddelde 7:59 en een standaardafwijking van $4$ minuten.
        Hoe groot is de kans dat hij voor 10.00 uur aankomt op het KIM? 
    }
    \solution{
        
    }

    \subquestion{6}{
        Hoe laat moet meneer de Wolf `s ochtends uiterlijk vertrekken om te zorgen dat hij met kans $0,95$ v\'o\'or $10.00$ uur op het KIM aankomt? Rond af op hele minuten.
    }
    \solution{

    }

\end{question}