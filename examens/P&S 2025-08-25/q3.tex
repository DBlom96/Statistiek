\begin{enquestion}{20}{
    A coastal rescue team is conducting search operations for missing swimmers lost at sea.
    Based on previous data and ocean current modeling, the joint probability density function (PDF) for the coordinates ($X$, $Y$) of a swimmer's location (in kilometers relative to the shore) is given by:
    \begin{align*}
        f_{X,Y}(x,y) =  \begin{cases}
                            C \cdot (3x+2y), & \text{if } 0 \le x \le 2 \text{ and } 0 \le x \le 3 \\
                            0, & \text{otherwise,}
                        \end{cases}
    \end{align*}
    where $C$ is a constant.
}

    \subquestion{6}{Determine the value of $C$ so that $f_{X,Y}(x,y)$ satisfies the properties of a valid joint probability density function.}
    \solution{
        The joint probability density function $f_{X,Y}(x,y)$ is a valid joint probability density function if the following two conditions hold
        \begin{enumerate}
            \item First condition: $f_{X,Y}(x,y)\ge 0$ for all possible values of $x$ and $y$, and \rubric{1}
            \item Second condition: $\int_{-\infty}^{\infty} \int_{-\infty}^{\infty} f_{X,Y}(x,y)\dx\dy=1$. \rubric{1}
        \end{enumerate}
        The first condition holds if and only if $C(3x+2y)$ for $0\le x\le 2$ and $0 \le y \le 3$, so we need $C \ge 0$.
        The second condition holds if
        \begin{align*}
            1   &= \int_{-\infty}^{\infty} \int_{-\infty}^{\infty} f_{X,Y}(x,y)\dx\dy \\
                &= \int_0^3 \int_0^2 C \cdot (3x+2y)\dx\dy \\
                &= C \cdot \int_0^3 \left[\int_0^2 (3x+2y) dx\right] \dy \\
                &= C \cdot \int_0^3 \left[\frac{3}{2}x^2 +2xy\right]_0^2 \dy \\
                &= C \cdot \int_0^3 6+4y \dy \\
                &= C \cdot \left[6y + 2y^2 \right]_0^3 \\
                &= C \cdot (18 + 18) = 36 \cdot C \rubric{3}
        \end{align*}
        The constant $C$ should be equal to $\frac{1}{36}$ in order for $f_{X,Y}(x,y)$ to satisfy the properties of a valid joint probability density function.
        \points{4}
    }
    
    \subquestion{6}{Compute the marginal PDFs $f_X(x)$ and $f_Y(y)$, for $X$ and $Y$ respectively.}
    \solution{
        \begin{itemize}
            \item The marginal PDF of $X$ is given by 
            \begin{align*}
                f_{X}(x) = \int_{-\infty}^{\infty} f_{X,Y} \dy &= \int_0^3 \frac{1}{36} \cdot (3x+2y) \dy \\
                                                              &= \left[\frac{1}{12}xy + \frac{1}{36}y^2\right]_{0}^3 \\
                                                              &= \frac{1}{4}x+\frac{1}{4} \rubric{3}
            \end{align*}
            \item The marginal PDF of $Y$ is given by 
            \begin{align*}
                f_{Y}(y) = \int_{-\infty}^{\infty} f_{X,Y} \dx &= \int_0^2 \frac{1}{36} \cdot (3x+2y) \dx \\
                                                              &= \left[\frac{1}{24}x^2 + \frac{1}{18}xy\right]_{0}^2 \\
                                                              &= \frac{1}{9}y+\frac{1}{6} \rubric{3}
            \end{align*}
        \end{itemize}
    }
    \subquestion{8}{Are $X$ and $Y$ independent random variables? Justify your answer with a computation.}
    \solution{
        To show whether or not $X$ and $Y$ are independent random variables, we need to compute their covariance and check if it is equal to zero (independence) or not (not independence).
        We can compute the covariance as follows:
        \[
            \Cov{X,Y} = E[XY] - E[X]\cdot E[Y]. \rubric{1}
        \]
        
        We first need to compute these quantities:
        \begin{align*}
            E[X] = \int_{-\infty}^{\infty} xf_{X}(x) \dx &= \int_{0}^{2} x \cdot \left(\frac{1}{4}x+\frac{1}{4}\right) \dx  \\
                                                        &= \int_{0}^{2} \frac{1}{4}x^2+\frac{1}{4}x \dx  \\
                                                        &= \left[\frac{1}{12}x^3 + \frac{1}{8}x^2\right]_{0}^{2} = \frac{7}{6}. \rubric{2}
        \end{align*}

        \begin{align*}
            E[Y] = \int_{-\infty}^{\infty} yf_{Y}(y) \dy &= \int_{0}^{3} y \cdot \left(\frac{1}{9}y+\frac{1}{6}\right) \dy  \\
                                                        &= \int_{0}^{3} \frac{1}{9}y^2+\frac{1}{6}y \dy  \\
                                                        &= \left[\frac{1}{27}y^3 + \frac{1}{12}y^2\right]_{0}^{3} = \frac{7}{4}. \rubric{1}
        \end{align*}

        \begin{align*}
            E[XY] = \int_{-\infty}^{\infty} \int_{-\infty}^{\infty} xyf_{X,Y}(x,y) \dx\dy &= \int_{0}^{3} \int_{0}^{2} \left( \frac{1}{12}x^2y + \frac{1}{18}xy^2\right)\dx\dy  \\
                                                        &= \int_0^3 \left[\frac{1}{36}x^3y + \frac{1}{36}x^2y^2\right]_{0}^{2} \dy \\
                                                        &= \int_0^3 \frac{2}{9}y + \frac{1}{9}y^2 \dy \\
                                                        &= \left[\frac{1}{9}y^2 + \frac{1}{27}y^3\right]_0^3 = 2. \rubric{2}
        \end{align*}

        The covariance of $X$ and $Y$ is equal to $\Cov{X}{Y} = E[XY] - E[X] \cdot E[Y] = 2 - \frac{7}{6} \cdot \frac{7}{4} = -\frac{1}{24} \neq 0$. \rubric{1}
        Hence, $X$ and $Y$ cannot be independent random variables. \rubric{1}
    }
\end{enquestion}