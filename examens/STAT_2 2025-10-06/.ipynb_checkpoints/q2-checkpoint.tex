\begin{question}{20}{
    Een Oekraïens militaire inlichtingencentrum heeft het vermoeden dat er een bepaald patroon zit in het aantal Shahed-drones dat gebruikt wordt in Russische drone-aanvallen.
    Om dit te onderzoeken, hebben ze de gegevens verzameld over het aantal drones dat gebruikt werd in $171$ aanvallen.
     
    \begin{center}
        \renewcommand{\arraystretch}{1.25}
        \begin{tabular}{cc}
            \toprule
                \textbf{Aantal drones} & \textbf{Frequentie} \\
            \midrule
                $1$ & $18$ \\
                $2$ & $23$ \\
                $3$ & $34$ \\
                $4$ & $27$ \\
                $5$ & $31$ \\
                $\geq 6$ & $38$ \\
            \bottomrule
        \end{tabular}
    \end{center}

}
    \subquestion{10}{
        Toets of de verdeling van het aantal drones significant afwijkt van een uniforme verdeling over deze zes categorieën.
        Bepaal de toetsuitslag aan de hand van de $p$-waarde en kies als significantieniveau $\alpha = 0.02$.
    }

    \solution{
        We voeren een chikwadraat toets voor aanpassing met een significantieniveau van $\alpha = 0.02$.

        We beginnen met het definiëren van de nulhypothese $H_0$ en de alternatieve hypothese $H_1$:
        \begin{itemize}
            \item[$H_0$: ] het aantal drones in Russische drone-aanvallen is uniform verdeeld \rubric{1}
            \item[$H_1$: ] het aantal drones in Russische drone-aanvallen is NIET uniform verdeeld \rubric{1}
        \end{itemize}
        Allereerst berekenen we op basis van de discrete uniforme kansverdeling (evenveel kans op iedere mogelijke categorie) de verwachte frequenties:
        \begin{center}
            \begin{tabular}{ccc}
                \toprule
                    \textbf{Categorie} & \textbf{Observed} & \textbf{Expected} \\
                \midrule
                    $1$ & $18$ & $28.5$ \\
                    $2$ & $23$ & $28.5$ \\
                    $3$ & $34$ & $28.5$ \\
                    $4$ & $37$ & $28.5$ \\
                    $5$ & $31$ & $28.5$ \\
                    $\geq 6$ & $38$ & $28.5$ \\ 
                \bottomrule
            \end{tabular}\rubric{2}
        \end{center}
    
        Vervolgens berekenen we de geobserveerde toetsingsgrootheid $\chi^2$ als volgt:
       \begin{align*}
            \chi^2 &= \frac{(O_{1} - E_{1})^2}{E_{1}} + \frac{(O_{2} - E_{2})^2}{E_{2}} + \frac{(O_{3} - E_{3})^2}{E_{3}} + \frac{(O_{4} - E_{4})^2}{E_{4}} + \frac{(O_{5} - E_{5})^2}{E_{5}} + \frac{(O_{6} - E_{6})^2}{E_{6}}\\
                &= \frac{(18 - 28.5)^2}{28.5} + \frac{(23 - 28.5)^2}{28.5} + \frac{(34 - 28.5)^2}{28.5} + \frac{(27 - 28.5)^2}{28.5} + \frac{(31 - 28.5)^2}{28.5} + \frac{(38 - 28.5)^2}{28.5}\\
                &\approx 9.4561
    \end{align*}
    
        De theoretische toetsingsgrootheid $X^2$ volgt onder de nulhypothese een $\chi^2$-verdeling met $\text{df}= 6 - 1 - 0 = 5$ vrijheidsgraden. \rubric{1}
        De $p$-waarde berekenen we als de rechteroverschrijdingskans van onze geobserveerde toetsingsgrootheid $\chi^2$ op basis van de chikwadraatverdeling met $5$ vrijheidsgraden:
        \begin{align*}
            p = P(\chi^2 > 9.4561) &= \chi^2\text{cdf}(\text{lower}=9.4561; \text{upper}=10^{99}; \text{df}=5) \\
                                             &\approx 0.0922.   
        \end{align*}
        Omdat $p > \alpha = 0.02$, wordt de nulhypothese $H_0$ aangenomen.
        Er is onvoldoende reden om de aanname te verwerpen dat de geobserveerde data tot stand zijn gekomen als trekkingen van de uniforme kansverdeling.\rubric{1}
    }

    \subquestion{5}{
        Voer de toets opnieuw uit, maar bepaal nu de toetsuitslag aan de hand van het kritieke gebied.
    }
    \solution{
        Deze toets heeft veelal dezelfde elementen als het antwoord bij vraag a. Het verschil zit hem in de manier waarop tot de conclusie wordt gekomen. \rubric{1}
        Bij een chikwadraattoets is het kritieke gebied altijd van de vorm $[g, \infty)$, oftewel de toets is altijd rechtszijdig. \rubric{1}
        
        Deze grens $g$ voldoet aan de volgende vergelijking:
        \begin{align*}
            P(X^2 \geq g) = \chi^2\text{cdf}(\text{lower}=g; \text{upper}=10^{99}; \text{df}=5) = \alpha = 0.02
        \end{align*}
        Deze vergelijking kunnen we oplossen door gebruik te maken van de solver optie op de grafische rekenmachine:
        \begin{align*}
            &y_1: \qquad \chi^2\text{cdf}(\text{lower}=X; \text{upper}=10^{99}; \text{df}=5) \\
            &y_2: \qquad 0.02 \rubric{1}
        \end{align*}
        De oplossing voor deze vergelijking is $g \approx 13.3882$. \rubric{1}
        De geobserveerde toetsingsgrootheid $\chi^2 = 10.7017$ ligt dus niet in het kritieke gebied, waardoor de nulhypothese $H_0$ wordt aangenomen.
        Er is onvoldoende reden om de aanname te verwerpen dat de geobserveerde data tot stand zijn gekomen als trekkingen van de uniforme kansverdeling.\rubric{1}
    }

    \subquestion{5}{
        Verklaar de toetsuitslag aan de hand van de frequenties in de bovenstaande tabel en de verwachte frequenties volgens een uniforme verdeling.
    }
    \solution{
        Als we kijken naar de geobserveerde frequenties, dan zien we dat deze in de buurt liggen van de verwachte frequenties volgens de uniforme verdeling. \rubric{2}
        Het is dus niet zo gek dat de nulhypothese $H_0$ wordt aangenomen, omdat de toetsingsgrootheid $\chi^2$ kleiner is naarmate de observed en expected frequenties meer op elkaar lijken.\rubric{3}
    }
\end{question}