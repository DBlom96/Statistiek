\begin{question}{30}{
    Het Ministerie van Defensie wil onderzoeken of er een verband is tussen het OPCO waar een militair deel van uitmaakt en de mate waarin hij/zij tevreden is met zijn/haar huidige gevechtstenue.
    De tevredenheid met de uitrusting wordt gemeten met een enquête en kan twee waarden aannemen: ``tevreden'' en ``niet tevreden''.
    \vspace{1ex}
    \begin{center}
        \renewcommand{\arraystretch}{1.25}
        \begin{tabular}{cccc}
            \toprule
                & \textbf{Tevreden} & \textbf{Niet tevreden} \\
            \midrule
    			\textbf{CLAS} & $250$ & $150$ \\ 
			    \textbf{CLRS} & $200$ & $200$ \\ 
			    \textbf{CZSK} & $220$ & $180$ \\ 
                \textbf{KMAR} & $180$ & $220$ \\
            \bottomrule
        \end{tabular}
    \end{center}
}

    \subquestion{4}{
        Welk type hypothesetoets dienen we uit te voeren om aan te tonen of er een verband is tussen OPCO en mate van tevredenheid?
        Formuleer de nulhypothese $H_0$ en de alternatieve hypothese $H_1$ van de bijbehorende hypothesetoets.
    }
    \solution{
        Omdat we willen weten of de mate van tevredenheid significant afwijkt tussen militairen van verschillende OPCO's, dienen we een chikwadraattoets voor onafhankelijkheid uit te voeren. \rubric{2}
        Dit kan ook omdat we te maken hebben met twee categorische variabelen!
        
        De bijbehorende nul- en alternatieve hypothese zijn in dit geval gelijk aan
        \begin{itemize}
            \item[$H_0$:] de verdeling van de mate van tevredenheid is onafhankelijk van het OPCO van de militairen. \rubric{1}
            \item[$H_1$:] de verdeling van de mate van tevredenheid is WEL afhankelijk van het OPCO van de militairen. \rubric{1}
        \end{itemize}
    }

    \subquestion{11}{Bepaal de $p$-waarde van de desbetreffende hypothesetoets.}
    \solution{
        We voeren een chikwadraat toets voor onafhankelijkheid uit om de bovenstaande hypotheses te toetsen.
        Allereerst berekenen we op basis van de aanname van onafhankelijkheid de verwachte frequenties:

        \begin{minipage}{0.45\textwidth}
            \begin{table}[H]
                \centering
                % \caption{Geobserveerde frequenties (observed)}
                \begin{tabular}{c|cc|c}
                    \multicolumn{4}{c}{\textbf{Geobserveerde frequenties (observed)}} \\
                    \toprule
                        & \textbf{T} & \textbf{NT}  & \textbf{Totaal} \\
                    \midrule
                        \textbf{CLAS} & $250$ & $150$ & $400$ \\ 
                        \textbf{CLRS} & $200$ & $200$ & $400$ \\ 
                        \textbf{CZSK} & $220$ & $180$ & $400$ \\ 
                        \textbf{KMAR} & $180$ & $220$ & $400$ \\ 
                    \midrule
                        \textbf{Totaal} & $850$ & $750$ & $1600$ \\
                    \bottomrule
                \end{tabular}
            \end{table}
        \end{minipage}
        \hfill
        \begin{minipage}{0.45\textwidth}
            \begin{table}[H]
                \centering
                % \caption{Verwachte frequenties \\ (expected)}
                \begin{tabular}{c|cc|c}
                    \multicolumn{4}{c}{\textbf{Verwachte frequenties (expected)}} \\
                    \toprule
                        & \textbf{T}  & \textbf{NT}  & \textbf{Totaal} \\
                    \midrule
    			        \textbf{CLAS} & $212.5$ & $187.5$ & $400$ \\ 
                        \textbf{CLRS} & $212.5$ & $187.5$ & $400$ \\ 
                        \textbf{CZSK} & $212.5$ & $187.5$ & $400$ \\ 
                        \textbf{KMAR} & $212.5$ & $187.5$ & $400$ \\ 
                    \midrule
                        \textbf{Totaal} & $850$ & $750$ & $1600$ \\
                    \bottomrule
                \end{tabular}
            \end{table}
        \end{minipage}
        
        Vervolgens berekenen we de toetsingsgrootheid $X^2$ als volgt:
        \begin{align*}
            \chi^2  &= \frac{(O_{1, 1} - E_{1,1})^2}{E_{1, 1}} + \frac{(O_{1, 2} - E_{1,2})^2}{E_{1, 2}} + \ldots + \frac{(O_{4, 2} - E_{4,2})^2}{E_{4, 2}} \\
                    &= \frac{(250 - 212.5)^2}{212.5} + \frac{(150 - 187.5)^2}{187.5} + \ldots + \frac{(220 - 187.5)^2}{187.5} \\
                    &\approx 26.8549.
        \end{align*}
        De toetsingsgrootheid $X^2$ volgt onder de nulhypothese een $\chi^2$-verdeling met $\text{df}= (\#\text{rijen}-1)\cdot(\#\text{kolommen}-1) = (4-1)\cdot(2-1) = 3$ vrijheidsgraden.
        \begin{align*}
            p   &= P(\chi^2 > 26.8549) \\ 
                &= \chi^2\text{cdf}(\text{lower}=26.8549; \text{upper}=10^{99}; \text{df}=3) \\
                &\approx 0.   
        \end{align*}
    }

    \subquestion{5}{Geef de conclusie van de hypothesetoets (op basis van een significantieniveau $\alpha = 0.05$) met behulp van de berekende $p$-waarde.}
    \solution{
        Uit de hypothesetoets volgt een bizar kleine $p$-waarde van nagenoeg $0$, oftewel rechteroverschrijdingskans.
         grote rechteroverschrijdingskans $p \approx 0.2059$.
        Omdat $p << \alpha = 0.05$, geldt dat $H_0$ wordt verworpen. \rubric{2}
        We kunnen dus de conclusie trekken dat de categorische variabelen OPCO en mate van tevredenheid niet onafhankelijk zijn van elkaar. \rubric{1}
        
        Als we kijken naar de tabel van geobserveerde frequenties, dan zien we dat de verdeling tevreden-niet tevreden erg verschilt tussen de verschillende OPCO's.\rubric{1}
        Dat duidt erop dat bijvoorbeeld militairen van de KMAR vaker ontevreden zijn dan bijvoorbeeld landmachters. 
        Hierdoor zijn er ook grote verschillen tussen de geobserveerde en verwachte frequenties.\rubric{1}
    }

    \subquestion{10}{
        Naast de vraag of de mate van tevredenheid afhangt van het OPCO, is het ministerie ook geïnteresseerd in het percentage tevreden militairen.
        Bepaal een \SI{95}{\percent}-betrouwbaarheidsinterval voor het percentage tevreden militairen (over alle OPCO's samen).
    }
    \solution{
        Om een betrouwbaarheidsinterval te bepalen van een fractie (of percentage) gebruiken we de Clopper-Pearson methode. \rubric{1}
        Laat $X$ het aantal tevreden militairen zijn in de steekproef.
        Aangezien de steekproefomvang gelijk is aan $1600$ militairen, is $X$ binomiaal verdeeld met $n = 400$ en een onbekende succeskans $p$. \rubric{1}
        De steekproefuitkomst is dat er $k = 850$ tevreden militairen zijn. \rubric{1}
        Omdat de gewenste betrouwbaarheid \SI{95}{\percent} is, geldt dat $\alpha = 0.05$.

        De Clopper-Pearson methode werkt als volgt:
        \begin{enumerate}
            \item Bepaal de succeskans $p_1$ waarvoor geldt dat de linkeroverschrijdingskans van de uitkomst $k=850$ gelijk is aan $\alpha/2$, oftewel $P(X \leq 850) = \alpha/2 = 0.025$.
            Voer hiervoor in het solver menu van de grafische rekenmachine in:
            \begin{align*}
                y_1 &= \binomcdf(n=1600; p_1=X; k=850) \\
                y_2 &= 0.025 \rubric{2}
            \end{align*}
            De solver optie geeft een waarde van $p_1 \approx 0.5559$.\rubric{1}
            \item Bepaal de succeskans $p_2$ waarvoor geldt dat de rechteroverschrijdingskans van de uitkomst $k=850$ gelijk is aan $\alpha/2$, oftewel $P(X \geq 850) = 1 - P(X \le 849) = \alpha/2 = 0.025$.
            Voer hiervoor in het solver menu van de grafische rekenmachine:
            \begin{align*}
                y_1 &= 1 - \binomcdf(n=1600; p_1=X; k=849) \\
                y_2 &= 0.025 \rubric{2}
            \end{align*}
            De solver optie geeft een waarde van $p_2 \approx 0.5064$.\rubric{1}
        \end{enumerate}
        We vinden het Clopper-Pearson interval door de twee gevonden waarden als grenzen te nemen, oftewel het \SI{90}{\percent}-betrouwbaarheidsinterval voor de fractie tevreden militairen loopt van \SI{50.64}{\percent} tot \SI{55.59}{\percent}. \rubric{1}
    }
\end{question}