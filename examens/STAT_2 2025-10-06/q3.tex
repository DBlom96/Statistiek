\begin{question}{30}{
    Het Ministerie van Defensie wil onderzoeken of er een verband is tussen het OPCO waar een militair deel van uitmaakt en de mate waarin hij/zij tevreden is met zijn/haar huidige gevechtstenue.
    De tevredenheid met de uitrusting wordt gemeten met een enquête en kan twee waarden aannemen: ``tevreden'' en ``niet tevreden''.
    \vspace{1ex}
    \begin{center}
        \renewcommand{\arraystretch}{1.25}
        \begin{tabular}{cccc}
            \toprule
                & \textbf{Tevreden} & \textbf{Niet tevreden} \\
            \midrule
    			\textbf{CLAS} & $250$ & $150$ \\ 
			    \textbf{CLRS} & $200$ & $200$ \\ 
			    \textbf{CZSK} & $220$ & $180$ \\ 
                \textbf{KMAR} & $180$ & $220$ \\
            \bottomrule
        \end{tabular}
    \end{center}
}

    \subquestion{4}{
        Welk type hypothesetoets dienen we uit te voeren om aan te tonen of er een verband is tussen OPCO en mate van tevredenheid?
        Formuleer de nulhypothese $H_0$ en de alternatieve hypothese $H_1$ van de bijbehorende hypothesetoets.
    }
    \solution{
        Omdat we willen weten of de mate van tevredenheid significant afwijkt tussen militairen van verschillende OPCO's, dienen we een $\chi^2$-toets (chikwadraat) voor onafhankelijkheid uit te voeren. \rubric{2}
        Dit kan ook omdat we te maken hebben met twee categorische variabelen!
        
        De bijbehorende nul- en alternatieve hypothese zijn in dit geval gelijk aan
        \begin{align*}
            H_0&: \text{ de verdeling van de mate van tevredenheid } \\
               & \qquad \qquad \text{is onafhankelijk van het OPCO van de militairen.} \rubric{1} \\
            H_1&: \text{ e verdeling van de mate van tevredenheid } \\
               & \qquad \qquad \text{is wel afhankelijk van het OPCO van de militairen.} \rubric{1}
        \end{align*}
    }

    \subquestion{11}{Bepaal de $p$-waarde van de desbetreffende hypothesetoets.}
    \solution{
        Om een chi-kwadraattoets voor onafhankelijkheid uit te voeren, moeten we eerst de verwachte frequenties uitrekenen. Onder de aanname dat $H_0$ waar is, is deze voor elk van de negen cellen te berekenen als $\frac{\text{rijtotaal} \cdot \text{kolomtotaal}}{\text{totaal}}$.
        De verwachte frequenties zijn te berekenen met behulp van de formule:
        \[
            E_{ij} = \frac{\text{rijtotaal}_{i} \cdot \text{kolomtotaal}_{j}}{\text{totaal}}
        \]

        \begin{center}
            \begin{minipage}{0.45\textwidth}
                \begin{table}[H]
                    \centering
                    \caption{``Observed'' frequenties}
                    \begin{tabular}{c|cc|c}
                        \toprule
                            & C & A & Totaal \\
                        \midrule
                            Leiderschap & $52$ & $22$ & $74$ \\ 
                            Avontuur & $23$ & $9$ & $32$ \\ 
                            Vaderland dienen & $38$ & $28$ & $66$ \\
                        \midrule
                            Totaal & $113$ & $59$ & $172$ \\
                        \bottomrule
                    \end{tabular}
                \end{table}
            \end{minipage}
            \hfill
            \begin{minipage}{0.45\textwidth}
                \begin{table}[H]
                    \centering
                    \caption{``Expected'' frequenties}
                    \begin{tabular}{cc|c}
                        \toprule
                            C & A & Totaal \\
                        \midrule
                             $48,6163$ & $25,3837$ & $74$ \\ 
                             $21,0233$ & $10,9767$ & $32$ \\ 
                             $43,3605$ & $22,6395$ & $66$ \\ 
                        \midrule
                             $113$ & $59$ & $172$ \\
                        \bottomrule
                    \end{tabular}
                \end{table}
            \end{minipage}\rubric{4}
        \end{center} 
        
        We berekenen de toetsingsgrootheid $\chi^2$ als volgt, waarbij $O_{ij} / E_{ij}$ de geobserveerde / verwachte frequentie is in rij $i$, kolom $j$:        
        \begin{align*}
            \chi^2 &= \frac{(O_{1, 1} - E_{1,1})^2}{E_{1, 1}} + \frac{(O_{1, 2} - E_{1,2})^2}{E_{1, 2}} + \ldots + \frac{(O_{3, 2} - E_{3,2})^2}{E_{3, 2}} \\
                    &= \frac{(52 - 48,6163)^2}{48,6163} + \frac{(22 - 25,3837)^2}{25,3837} + \ldots + \frac{(28 - 22,6395)^2}{22,6395} \\
                    &\approx 3,1603. \rubric{2}
        \end{align*}

        De bijbehorende $p$-waarde is de rechteroverschrijdingskans $P(X > \chi^2)$, waarbij $X^2$ de toetsingsgrootheid van de chikwadraattoets voor onafhankelijkheid. \rubric{1}
        De toetsingsgrootheid $X^2$ volgt onder de nulhypothese een $\chi^2$-verdeling met $\text{df}= (\#\text{rijen}-1)\cdot(\#\text{kolommen}-1) = (3-1)\cdot(2-1) = 2$ vrijheidsgraden. \rubric{1}

        \begin{align*}
            p   &= P(\chi^2 > 3.1603) \\
                &= \chi^2\text{cdf}(\text{lower}=3.1603; \text{upper}=10^{99}; \text{df}=2) \\
                &\approx 0.2059.  \rubric{2}
        \end{align*}
        Omdat $p > \alpha$, wordt de nulhypothese $H_0$ aangenomen.
        Er is onvoldoende reden om de aanname te verwerpen dat de twee variabelen onafhankelijk zijn van elkaar. \rubric{1}
    }

    \subquestion{5}{Geef de conclusie van de hypothesetoets (op basis van een significantieniveau $\alpha = 0,05$) met behulp van de berekende $p$-waarde.}
    \solution{
        Uit de hypothesetoets volgt een redelijk grote rechteroverschrijdingskans $p \approx 0,2059$.
        Omdat $p > \alpha = 0,05$, geldt dat $H_0$ wordt aangenomen. \rubric{2}
        We kunnen dus de conclusie trekken dat de verdeling van primaire motivaties om officier te worden niet significant verschilt tussen cadetten en adelborsten. \rubric{1}
        
        Als we kijken naar de tabel van geobserveerde en verwachte frequenties, dan zien we dat deze redelijk dichtbij elkaar in de buurt liggen.
        Er zijn geen grote uitschieters te bekennen in de verwachte frequenties ten opzichte van de daadwerkelijk geobserveerde frequenties.\rubric{2}
    }

    \subquestion{10}{
        Naast de vraag of de mate van tevredenheid afhangt van het OPCO, is het ministerie ook geïnteresseerd in het percentage tevreden militairen.
        Bepaal een \SI{95}{\percent}-betrouwbaarheidsinterval voor het percentage tevreden militairen (over alle OPCO's samen).
    }
\end{question}