\begin{question}{20}{
    Om het hoofd te kunnen bieden aan toekomstige dreigingen, wordt onderzoek gedaan naar high-performance materialen voor het materieel van de toekomst.
    Onderdeel van dit onderzoek is een test van een nieuwe metaallegering, die beter bestand zou moeten zijn tegen zware explosies.
    Voor deze test worden een aantal samples gemaakt van de legering waarvan de treksterkte wordt gemeten (in $N/mm^2$).
    De waargenomen aantallen zijn als volgt:
    \[
        931, 978, 954, 962, 984, 976, 923, 940, 988, 937
    \]
    Neem aan dat de treksterkte een normale verdeling volgt.
}

    \subquestion{5}{        
       Bereken van de gemeten waarden het steekproefgemiddelde en de steekproefstandaardafwijking.
    }
    \solution{

    }
    
    \subquestion{7}{
        Bereken een \SI{90}{\percent}-betrouwbaarheidsinterval voor de gemiddelde treksterkte $\mu$ van de nieuwe metaallegering, op grond van de bovengenoemde steekproefresultaten.
        Laat hierbij duidelijk je berekeningen zien en maak geen gebruik van de optie TESTS/Interval op de grafische rekenmachine.
        Rond het interval af op gehele getallen zodanig dat de betrouwbaarheid gewaarborgd blijft.

    }
    \solution{
    
    }

    \subquestion{6}{
        De onderzoekers beweren dat de nieuwe metaallegering een gemiddelde treksterkte heeft van minstens $940 N/mm^2$.
        Toets deze bewering met een geschikte hypothesetoets.
        Bepaal de toetsuitslag aan de hand van het kritieke gebied op basis van bovenstaande steekproef.
        Kies in dit geval voor een significantieniveau van $\alpha = 0.05$.
    }
    \solution{
    
    }

    \subquestion{2}{
        Leg in eigen woorden uit wat de toetsuitslag betekent voor de gemiddelde treksterkte van het nieuwe materiaal.

    }
    \solution{

    }
\end{question}