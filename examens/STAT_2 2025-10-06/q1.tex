\begin{question}{20}{
    Om het hoofd te kunnen bieden aan toekomstige dreigingen, wordt onderzoek gedaan naar high-performance materialen voor het materieel van de toekomst.
    Onderdeel van dit onderzoek is een test van een nieuwe metaallegering, die beter bestand zou moeten zijn tegen zware explosies.
    Voor deze test worden een aantal samples gemaakt van de legering waarvan de treksterkte wordt gemeten (in $N/mm^2$).
    De waargenomen aantallen zijn als volgt:
    \[
        931, 978, 954, 962, 984, 976, 923, 940, 988, 937
    \]
    Neem aan dat de treksterkte een normale verdeling volgt.
}

    \subquestion{5}{        
       Bereken van de gemeten waarden het steekproefgemiddelde en de steekproefstandaardafwijking.
    }
    \solution{
        We berekenen het steekproefgemiddelde $\overline{x}$ als volgt:
        \begin{align*}
            \overline{x} = \frac{x_1+x_2+\ldots+x_n}{n} = \frac{ 931 + 978 + \ldots + 937 }{ 10 } \approx 957.3. \rubric{2}
        \end{align*}
    

        We berekenen de steekproefvariantie $s^2$ als volgt:
        \begin{align*}
            s^2 &= \frac{ (x_1 - \overline{x})^2 + (x_2 - \overline{x})^2 + \ldots + (x_n - \overline{x})^2 }{ n - 1 } \\
                &= \frac{ (931 - 957.3)^2 + (978 - 957.3)^2 + \ldots + (937 - 957.3)^2 }{ 10 - 1 } \\
                &\approx 560.6778. \rubric{2}
        \end{align*}
    

        We berekenen de steekproefstandaardafwijking $s$ door te wortel van de steekproefvariantie $s^2$ te nemen:
        \begin{align*}
            s = \sqrt{ s^2 } = \sqrt{ 560.6778 } \approx 23.6786. \rubric{1}
        \end{align*}
    }
    
    \subquestion{7}{
        Bereken een \SI{90}{\percent}-betrouwbaarheidsinterval voor de gemiddelde treksterkte $\mu$ van de nieuwe metaallegering, op grond van de bovengenoemde steekproefresultaten.
        Laat hierbij duidelijk je berekeningen zien en maak geen gebruik van de optie TESTS/Interval op de grafische rekenmachine.
        Rond het interval af op gehele getallen zodanig dat de betrouwbaarheid gewaarborgd blijft.

    }
    \solution{
        Gegeven is dat de treksterkte van de metaallegering een normale verdeling volgt met een onbekende verwachtingswaarde $\mu$ en een onbekende standaardafwijking $\sigma$.
        Volgens de centrale limietstelling geldt dat de gemiddelde treksterkte van de metaallegering dan normaal verdeeld is met parameters $\mu$ en $\frac{\sigma}{\sqrt{10}}$. \rubric{1}
        Omdat $\sigma$ onbekend is en de steekproefgrootte $n = 10 < 30$ is, moeten we gebruik maken van de $t$-verdeling. \rubric{1}
        De $t$-waarde die hoort bij \SI{90}{\percent} betrouwbaarheid, oftewel $\alpha = 0.1$, is (in het geval van tweezijdige intervallen) gelijk aan
        \[
            t = \invt(\text{opp}=1-\frac{\alpha}{2}; \text{df}=n-1) = \invt(\text{opp}=0.95; \text{df}=9) \approx 1.8331. \rubric{2}
        \]
        Het \SI{90}{\percent}-betrouwbaarheidsinterval wordt dan gegeven door
        \begin{align*}
            [\overline{x} - t \cdot \frac{ s }{ \sqrt{ n } }; \overline{x} + t \cdot \frac{ s }{ \sqrt{ n } }] \\
            &= [957.3 - 1.8331 \cdot \frac{ 23.6786 }{ \sqrt{ 10 } }; 957.3 + 1.8331 \cdot \frac{ 23.6786 }{ \sqrt{ 10 } }] \\
            &= [943.574; 971.026]. \rubric{2}
        \end{align*}
        Om de betrouwbaarheid te waarborgen, moeten we het interval naar buiten afronden, oftewel de ondergrens naar beneden afronden en de bovengrens naar boven.
        In andere woorden, het \SI{90}{\percent}-betrouwbaarheidsinterval voor de gemiddelde treksterkte $\mu$ loopt van $943$ tot $972$. \rubric{1} 
    }

    \subquestion{6}{
        De onderzoekers beweren dat de nieuwe metaallegering een gemiddelde treksterkte heeft van minstens $940 N/mm^2$.
        Toets deze bewering met een geschikte hypothesetoets.
        Bepaal de toetsuitslag aan de hand van het kritieke gebied op basis van bovenstaande steekproef.
        Kies in dit geval voor een significantieniveau van $\alpha = 0.05$.
    }
    \solution{
        In deze hypothesetoets hebben we te maken met een $t$-toets met nul- en alternatieve hypothese als volgt:
        \begin{align*}
            H_{0}: \mu &\geq 940 \tag{gemiddelde voldoet aan de bewering} \\
            H_{1}: \mu &< 940 \tag{gemiddelde voldoet NIET aan de bewering} \rubric{1}
        \end{align*}

        In dit geval werken we met een linkszijdige toets (alternatieve hypothese heeft vorm $<$), dus het kritieke gebied is van de vorm $(-\infty, g]$. \rubric{1}
        We willen de grens $g$ bepalen zodanig dat de kans op een type-I fout ($H_0$ verwerpen terwijl deze juist waar is) gelijk is aan $\alpha=0,05$.

        Onder de nulhypothese is de gemiddelde treksterkte $\overline{X}$ van 10 willekeurige monsters van de metaallegering normaal verdeeld met verwachtingswaarde $\mu = 940$ en standaardafwijking $\frac{\sigma}{\sqrt{10}}$.
        Omdat $\sigma$ onbekend is en $n = 10 < 30$, gebruiken we opnieuw de $t$-verdeling en $s = 23.6786$. \rubric{1}
        Er geldt dus, omdat we eenzijdig toetsen, dat 
        \[
            t = \invt(opp=1-\alpha; df=n-1) = \invt(opp=0.95; df=9) \approx 1.8331. \rubric{1}
        \]
        (merk op dat de $t$-waarde anders is, omdat we met een ander betrouwbaarheidsniveau werken!).
        
        De bijbehorende grenswaarde vinden we nu met
        \[
            g = \mu - t \cdot \frac{s}{\sqrt{10}} = 940 - 1.8331 \cdot \frac{23.6786}{\sqrt{10}} \approx 926.2741. \rubric{1}
        \]

        Het steekproefgemiddelde $\overline{x} \approx 957.3$ is groter dan deze bovengrens $g \approx 926.2741$ van het kritieke gebied, dus $H_0$ wordt niet verworpen. \rubric{1} 
    }

    \subquestion{2}{
        Leg in eigen woorden uit wat de toetsuitslag betekent voor de gemiddelde treksterkte van het nieuwe materiaal.

    }
    \solution{
        De toetsuitslag betekent dus dat $H_0$ niet wordt verworpen.\rubric{1}
        In andere woorden, op basis van de steekproef is er onvoldoende bewijs om de bewering van de onderzoekers (dat de gemiddelde treksterkte minstens $940$ $N/mm^2$ is) te ontkrachten. \rubric{1}
    }
\end{question}