\begin{question}{20}{
    Een Oekraïens militaire inlichtingencentrum heeft het vermoeden dat er een bepaald patroon zit in het aantal Shahed-drones dat gebruikt wordt in Russische drone-aanvallen.
    Om dit te onderzoeken, hebben ze de gegevens verzameld over het aantal drones dat gebruikt werd in $100$ aanvallen.
     
    \begin{center}
        \renewcommand{\arraystretch}{1.25}
        \begin{tabular}{cc}
            \toprule
                \textbf{Aantal drones} & \textbf{Frequentie} \\
            \midrule
                $1$ & $18$ \\
                $2$ & $23$ \\
                $3$ & $34$ \\
                $4$ & $27$ \\
                $5$ & $31$ \\
                $\geq 6$ & $38$ \\
            \bottomrule
        \end{tabular}
    \end{center}

}
    \subquestion{10}{
        Toets of de verdeling van het aantal drones significant afwijkt van een uniforme verdeling over deze zes categorieën.
        Bepaal de toetsuitslag aan de hand van de $p$-waarde en kies als significantieniveau $\alpha = 0,02$.
    }

    \solution{

    }

    \subquestion{5}{
        Voer de toets opnieuw uit, maar bepaal nu de toetsuitslag aan de hand van het kritieke gebied.
    }
    \solution{

    }

    \subquestion{5}{
        Verklaar de toetsuitslag aan de hand van de frequenties in de bovenstaande tabel en de verwachte frequenties volgens een uniforme verdeling.
    }
    \solution{
    
    }
\end{question}