\begin{question}{28}{
    De Koninklijke Marine is geïnteresseerd in het bepalen van een betrouwbare onderhoudsstrategie voor de maritieme NH90-gevechtshelikopters.
    Hiervoor is het belangrijk om data te verzamelen van de \emph{mean time between failures} (MTBF), oftewel de gemiddelde tijd tussen twee faalmomenten van een helikopteronderdeel.
    Om te onderzoeken wat kritieke onderdelen zijn, worden de \emph{time between failures} (TBF) van de motor ($X$) en rotorbladen ($Y$) van tien NH90-gevechtshelikopters gemeten.

    De volgende data zijn verzameld over de time between failures van motoren en rotorbladen van NH90-gevechtshelikopters.
    
    \begin{center}
        \begin{tabular}{c|cccccccccc}
            \toprule
                TBF motoren (uren) & $1185$ & $1175$ & $1195$ & $1180$ & $1195$ & $1190$ & $1185$ & $1215$ & $1175$ & $1205$ \\ 
            \midrule
                TBF rotorbladen (uren) & $1180$ & $1205$ & $1190$ & $1210$ & $1175$ & $1200$ & $1225$ & $1195$ & $1185$ & $1215$ \\
            \bottomrule
        \end{tabular}
    \end{center}
    \vspace{1em}

    De centrale vraag is nu om te toetsen of de MTBF van de motor significant lager is dan die van de rotorbladen, oftewel dat de motor sneller faalt dan de rotorbladen.
    Voor beide steekproeven kan worden aangenomen dat de tijden tussen faalmomenten normaal verdeeld zijn.
}

    \subquestion{8}{
        Bepaal voor beide populaties de steekproefgemiddelden ($x$ en $y$) en de steekproefvarianties ($s_X^2$ en $s_Y^2$).
    }
    \solution{

        \textbf{Time between failures voor motoren:}

        We berekenen het steekproefgemiddelde $\overline{x}$ als volgt:
        \begin{align*}
            \overline{x} = \frac{x_1+x_2+\ldots+x_n}{n} = \frac{ 1185 + 1175 + \ldots + 1205 }{ 10 } = 1190. \rubric{2}
        \end{align*}
    

        We berekenen de steekproefvariantie $s_X^2$ als volgt:
        \begin{align*}
            s_X^2 &= \frac{ (x_1 - \overline{x})^2 + (x_2 - \overline{x})^2 + \ldots + (x_n - \overline{x})^2 }{ n - 1 } \\
                &= \frac{ (1185 - 1190)^2 + (1175 - 1190)^2 + \ldots + (1205 - 1190)^2 }{ 10 - 1 } \\
                &\approx 166.6667. \rubric{2}
        \end{align*}
        
        \textbf{Time between failures voor rotorbladen:}

        We berekenen het steekproefgemiddelde $\overline{y}$ als volgt:
        \begin{align*}
            \overline{y} = \frac{y_1+y_2+\ldots+y_n}{n} = \frac{ 1180 + 1205 + \ldots + 1215 }{ 10 } = 1198.\rubric{2}
        \end{align*}
    

        We berekenen de steekproefvariantie $s_x^2$ als volgt:
        \begin{align*}
            s_Y^2 &= \frac{ (y_1 - \overline{y})^2 + (y_2 - \overline{y})^2 + \ldots + (y_n - \overline{y})^2 }{ n - 1 } \\
                &= \frac{ (1180 - 1198)^2 + (1205 - 1198)^2 + \ldots + (1215 - 1198)^2 }{ 10 - 1 } \\
                &\approx 256.6667.\rubric{2}
        \end{align*}
    }

    \subquestion{10}{
        Voer een $F$-toets uit om te bepalen of de varianties van de MTBF's van de twee populaties als gelijk kunnen worden beschouwd ($\sigma_X^2 = \sigma_Y^2$).
        Gebruik hiervoor een significantieniveau van $\alpha = 0.05$ en bepaal de toetsuitslag op basis van het kritieke gebied.
    }
    \solution{
        In de vraag staat dat we aan mogen nemen dat voor de time between failures $X$ en $Y$ voor respectievelijk motoren en rotorbladen geldt dat
        $X \sim N(\mu_X=?; \sigma_X=?)$ en $Y \sim N(\mu_Y=?; \sigma_Y=?)$.

        We toetsen op gelijke varianties, oftewel
        \begin{align*}
            H_0: \quad \sigma_X^2 = \sigma_Y^2 \\
            H_1: \quad \sigma_X^2 \neq \sigma_Y^2 \rubric{2}
        \end{align*}

        Verder is gegeven dat we mogen werken met een significantieniveau $\alpha=0.05$, en data is al verzameld voor beide populaties.
        De toetsingsgrootheid voor een $F$-toets is gelijk aan
        \[
            F = \frac{S_X^2}{S_Y^2}, 
        \]
        en volgt een $F(n-1, m-1)$-verdeling, oftewel een $F(9,9)$-verdeling. \rubric{1}

        De geobserveerde toetsingsgrootheid is gelijk aan
        \[
            f = \frac{s_X^2}{s_Y^2} = \frac{166.6667}{256.6667} = 0.6494. \rubric{2}
        \]
        
        Omdat een $F$-toets altijd tweezijdig toetst, is het kritieke gebied van de vorm $(-\infty; g_1]$ en $[g_2; \infty)$, waarbij de grenzen $g_1$ en $g_2$ bepaald kunnen worden met de $F(9,9)$-verdeling.
        \begin{align*}
            &\fcdf(\text{lower}=0; \text{upper}=g_1; \text{df1}=9; \text{df2}=9)=\alpha/2=0.025 \rightarrow g_1 \approx 0.2484\\
            &\fcdf(\text{lower}=g_2; \text{upper}=10^{99}; \text{df1}=9; \text{df2}=9)=\alpha/2=0.025 \rightarrow g_2 \approx 4.0260 \rubric{2}
        \end{align*}
        Het kritieke gebied is dus $(-\infty; 0.2484]$ en $[4.0260; \infty)$. \rubric{1}
        De berekende $f = 0.6494$ ligt dus niet in het kritieke gebied, dus we kunnen de nulhypothese $H_0$ niet verwerpen. \rubric{1}
        Er is onvoldoende bewijs om op basis van deze steekproef de aanname van gelijke varianties te verwerpen. \rubric{1}
    }

    \subquestion{10}{
        Bepaal met behulp van een onafhankelijke $t$-toets of de MTBF van de motor significant lager is dan die van de rotorbladen.
        Gebruik hiervoor opnieuw een significantieniveau van $\alpha = 0.05$, en bepaal de toetsuitslag op basis van de $p$-waarde.
    }
    \solution{
        We toetsen of de gemiddelde time to failure $\mu_X$ voor de motoren significant lager is dan de gemiddelde time to failure $\mu_Y$ voor de rotorbladen.
        De hypotheses kunnen we daarom als volgt defini\"eren:
        \begin{align*}
            H_0: \quad \mu_X \geq \mu_Y \text{ (niet significant lager) } \\
            H_1: \quad \mu_X < \mu_Y \text{ (wel significant lager) } \rubric{2}
        \end{align*}

        Verder is gegeven dat we mogen werken met een significantieniveau $\alpha=0.05$, en data is al verzameld voor beide populaties.

        Op basis van ons antwoord bij vraag (b) mogen we nu aannemen dat $\sigma^2 = \sigma_X^2 = \sigma_Y^2$.
        Dat betekent dat we met de pooled variance mogen werken als schatting voor de gemeenschappelijke onbekende variantie $\sigma^2$. \rubric{1}
        
        \begin{align*}
            s_P^2 = \frac{(n-1)\cdot s_X^2 + (m-1) \cdot s_Y^2}{n-1+m-1} = \frac{9 \cdot 166.6667 + 9 \cdot 256.6667}{18} \approx 211.6667. \rubric{1}
        \end{align*}
        
        De toetsingsgrootheid van de bijbehorende $t$-toets is gelijk aan
        \[
            t = \frac{(\overline{x}-\overline{y}) - (\mu_X - \mu_Y)}{\sqrt{\frac{s_P^2}{n} + \frac{s_P^2}{m}}} = \frac{(1190-1198) - 0}{\sqrt{\frac{211.6667}{10} + \frac{211.6667}{10}}}\approx -1.2296, 
        \]
        en komt uit een $t$-verdeling met $\text{df}=18$ vrijheidsgraden. \rubric{2}
        Omdat we linkszijdig toetsen, is de $p$-waarde gelijk aan de linkeroverschrijdingskans van deze geobserveerde toetsingsgrootheid $t$:
        \begin{align*}
            p = P(T \leq t) &= \tcdf(\text{lower}=-10^{99}, \text{upper}=t; \text{df}=n+m-2) \\
                            &= \tcdf(-10^{99}; -1.2296, 18) \\
                            &\approx 0.1173 \rubric{2}
        \end{align*}
        
        Deze $p$-waarde is groter dan het significantieniveau $\alpha = 0.05$, dus $H_0$ wordt geaccepteerd.\rubric{1}
        Er is op basis van deze steekproeven onvoldoende reden om aan te nemen dat de MTBF van motoren significant lager is dan de MTBF van rotorbladen.\rubric{1}
    }
\end{question}