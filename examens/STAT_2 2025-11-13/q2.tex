\begin{question}{30}{
    Een luchtmachteenheid is een nieuw type radar aan het testen voor het detecteren van vijandelijke drones.
    Fabrikant Thales claimt dat de radar een succeskans van \SI{70}{\percent} heeft om een drone te detecteren (onafhankelijk van andere drones).
    Om deze claim te testen, worden er $1000$ onafhankelijke tests uitgevoerd. In elke test worden vier drones op het systeem afgestuurd en geteld hoeveel van de vier drones gedetecteerd worden.
    De gegevens zijn weergegeven in onderstaande frequentietabel:
    \begin{center}
        \renewcommand{\arraystretch}{0.75}
        \begin{tabular}{cc}
            \toprule
                \textbf{Aantal drones} & \textbf{Frequentie} \\
            \midrule
                $0$ & $15$ \\
                $1$ & $105$ \\
                $2$ & $290$ \\
                $3$ & $360$ \\
                $4$ & $230$ \\
            \bottomrule
        \end{tabular}
    \end{center}

    Om de claim van de fabrikant te toetsen, wordt een chikwadraat aanpassingstoets uitgevoerd.
}
\vspace{-1cm}
    \subquestion{4}{
        Welke kansverdeling volgt het aantal gedetecteerde drones in één enkele test met een radar?
        Geef daarnaast specifieke waardes van de bijbehorende parameters.
    }

    \solution{
        Laat $X$ het aantal gedetecteerde drones zijn in één enkele test met de radar.
        Omdat we een ``aantal successen'' (detecties) tellen uit een eindig aantal onafhankelijke Bernoulli-experimenten (aantal drones), betreft het een binomiale kansverdeling. \rubric{2}
        
        Het aantal Bernoulli-experimenten $n = 4$, want er doen vier drones mee per test.\rubric{1}

        Verder is de succeskans $p = 0.7$ (\SI{70}{\percent} detectiekans) per drone. \rubric{1}
    
        {
            \itshape Noot: de waarde van $n$ is NIET gelijk aan $1000$. Dit getal geeft alleen aan hoe vaak een realisatie van een Binomiaal$(n=4,p=0.7)$ verdeelde kansvariabele wordt gemeten.
        }
    }

    \subquestion{4}{
        Formuleer de nulhypothese $H_0$ en de alternatieve $H_1$ van deze hypothesetoets.
        Wat zou in deze context de betekenis zijn van het verwerpen van de nulhypothese?
    }

    \solution{
        De nulhypothese $H_0$ en de alternatieve hypothesetoets $H_1$ kunnen als volgt worden geformuleerd:
        \begin{description}
            \item[$H_0$:] het aantal gedetecteerde drones $X$ volgt een binomiale verdeling met parameters $n = 4$ en $p = 0.7$ \rubric{1}
            \item[$H_1$:] het aantal gedetecteerde drones $X$ volgt NIET een binomiale verdeling met parameters $n = 4$ en $p = 0.7$ \rubric{1}
        \end{description}
        Het verwerpen van de nulhypothese $H_0$ betekent in dit geval dat $X$ niet binomiaal verdeeld met parameters $n = 4$ en $p = 0.7$.\rubric{1}
        Het is echter nog steeds mogelijk dat $X$ binomiaal verdeeld is, maar dan moet gelden dat de succeskans $p \neq 0.7$.\rubric{1}
    }

    \subquestion{3}{
        Bereken de verwachte (``expected'') frequenties van het aantal gedetecteerde drones uitgaande van de nulhypothese $H_0$.
    }
    \solution{
        Om de verwachte frequenties te bepalen, moeten we gebruik maken van het feit dat er $1000$ onafhankelijke tests zijn uitgevoerd.
        Onder de nulhypothese $H_0$ is het aantal gedetecteerde drones $X$ binomiaal verdeeld met parameters $n = 4$ en $p = 0.7$. \rubric{1}
        
        {\small
            \begin{center}
                \renewcommand{\arraystretch}{1.25}
                \begin{tabular}{ccc}
                    \toprule
                        \textbf{Aantal drones} & \textbf{Observed} & \textbf{Expected} \\
                    \midrule
                        $0$ & $15$  & $1000 \cdot \binompdf(n=4, p=0.7, k=0) = 8.1$   \\
                        $1$ & $105$ & $1000 \cdot \binompdf(n=4, p=0.7, k=1) = 75.6$  \\
                        $2$ & $290$ & $1000 \cdot \binompdf(n=4, p=0.7, k=2) = 264.6$ \\
                        $3$ & $360$ & $1000 \cdot \binompdf(n=4, p=0.7, k=3) = 411.6$ \\
                        $4$ & $230$ & $1000 \cdot \binompdf(n=4, p=0.7, k=4) = 240.1$ \rubric{2} \\
                    \bottomrule
                \end{tabular}
            \end{center}
        }
    }

    \subquestion{6}{
        Bereken de toetsingsgrootheid en de $p$-waarde op basis van de gegeven frequenties.
    }
    \solution{
        We berekenen de toetsingsgrootheid $\chi^2$ als volgt:

        \begin{align*}
            \chi^2 &= \frac{(O_{0} - E_{0})^2}{E_{0}} + \frac{(O_{1} - E_{1})^2}{E_{1}} + \ldots + \frac{(O_{4} - E_{4})^2}{E_{4}}\\
                    &= \frac{(15 - 8.1)^2}{8.1} + \frac{(105 - 75.6)^2}{75.6} + \ldots + \frac{(230 - 240.1)^2}{240.1}\\
                    &\approx 26.643 
        \end{align*}
    
        De theoretische toetsingsgrootheid $X^2$ volgt onder de nulhypothese een $\chi^2$-verdeling met $\text{df}= \#\textrm{categorieën} - 1 = 4$ vrijheidsgraden.
        De $p$-waarde behorende bij onze toetsingsgrootheid is daarom gelijk aan        \begin{align*}
            p = P(\chi^2 > 26.643) &= \chi^2\text{cdf}(\text{lower}=26.643; \text{upper}=10^{99}; \text{df}=4) \\
                                             &\approx 0.   
        \end{align*}
    }

    \subquestion{5}{
        Formuleer een conclusie voor deze hypothesetoets (op basis van een significantieniveau $\alpha = 0.05$) in de originele context van het probleem, en verklaar deze aan de hand van de geobserveerde en verwachte frequenties.
    }
    \solution{
        De $p$-waarde is extreem klein.
        Omdat $p < \alpha$, wordt de nulhypothese $H_0$ verworpen. \rubric{1}
        Er is voldoende reden om aan te nemen dat het aantal gedetecteerde drones niet binomiaal verdeeld is met $n = 4$ en $p = 0.7$.\rubric{2}

        Als we kijken naar de geobserveerde en verwachte frequenties, dan zijn lage uitkomsten (0 t/m 2) vaker geobserveerd dan verwacht, en hoge uitkomsten ($3$ of $4$) juist minder vaak dan verwacht. \rubric{1}
        Het is dus waarschijnlijk dat de succeskans kleiner is dan de geclaimde $p = 0.7$.\rubric{1}
    }
    
    \subquestion{8}{
        Bereken een \SI{95}{\percent}-betrouwbaarheidsinterval voor de succeskans $p$ met de Clopper-Pearson methode.
        Wat zegt dit over de claim van Thales van \SI{70}{\percent} kans op detectie?

        {
            \itshape \textbf{Hint:} gebruik dat $1000$ onafhankelijke realisaties van een binomiale kansvariabele met $n = 4$ in feite neerkomt op $4000$ onafhankelijke Bernoulli-experimenten.
        }
    }

    \solution{
        Om het totaal aantal successen te tellen bij $1000$ onafhankelijke waarnemingen van een binomiale kansvariabele met $n = 4$ en $p = ?$ kijken we eigenlijk naar een binomiale kansvariabele met $n = 4000$ en $p = ?$. 
        Op basis van de tabel van geobserveerde frequenties vinden we dat het totaal aantal detecties (uit $4000$) gelijk is aan
        \begin{align*}
            0 \cdot 15 + 1 \cdot 105 + 2 \cdot 290 + 3 \cdot 360 + 4 \cdot 230 = 2685. \rubric{1}
        \end{align*} 

         Omdat de gewenste betrouwbaarheid \SI{95}{\percent} is, geldt dat $\alpha = 0.05$.

        De Clopper-Pearson methode werkt als volgt:
        \begin{enumerate}
            \item Bepaal de succeskans $p_1$ waarvoor geldt dat de linkeroverschrijdingskans van de uitkomst $k=2685$ gelijk is aan $\alpha/2$, oftewel $P(X \leq 2685) = \alpha/2 = 0.025$.
            Voer hiervoor in het solver menu van de grafische rekenmachine in:
            \begin{align*}
                y_1 &= \binomcdf(n=4000; p_1=X; k=2685) \\
                y_2 &= 0.025 \rubric{1}
            \end{align*}
            De solver optie geeft een waarde van $p_1 \approx 0.6858$.\rubric{1}
            \item Bepaal de succeskans $p_2$ waarvoor geldt dat de rechteroverschrijdingskans van de uitkomst $k=850$ gelijk is aan $\alpha/2$, oftewel $P(X \geq 2685) = 1 - P(X \le 2684) = \alpha/2 = 0.025$.
            Voer hiervoor in het solver menu van de grafische rekenmachine:
            \begin{align*}
                y_1 &= 1 - \binomcdf(n=4000; p_1=X; k=2684) \\
                y_2 &= 0.025 \rubric{2}
            \end{align*}
            De solver optie geeft een waarde van $p_2 \approx 0.6564$.\rubric{1}
        \end{enumerate}
        We vinden het Clopper-Pearson interval door de twee gevonden waarden als grenzen te nemen, oftewel het \SI{95}{\percent}-betrouwbaarheidsinterval voor de detectiekans $p$ van een drone door de nieuwe radar
        is gelijk aan $[0.6564, 0.6858]$. \rubric{1}

        We zien dat de geclaimde succeskans $p = 0.7$ niet in dit interval ligt.
        We kunnen dus met \SI{95}{\percent} betrouwbaarheid concluderen dat de succeskans lager ligt dan de geclaimde \SI{70}{\percent}. \rubric{1}
    }
\end{question}