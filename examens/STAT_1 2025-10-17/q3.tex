\begin{question}{30}{
    Bij een grenscontrole in de buurt van Enschede doet de Koninklijke Marechaussee een steekproef om drugssmokkel per auto te detecteren.
    Hierbij wordt op een dag uit de voorbijgaande auto's een steekproef van $n = 29$ auto's aselect gekozen.
    De doorzoektijd $T$ van deze auto's is uniform verdeeld tussen $11$ en $17$ minuten.
} 
    \subquestion{4}{
        Wat zijn de verwachtingswaarde $E[T]$ en de standaardafwijking $\sigma(T)$ van de doorzoektijd van een willekeurige auto?
    }

    \solution{

    }

    \subquestion{4}{
        Wat is de kans dat de doorzoeking van een willekeurige auto langer dan $15$ minuten duurt?
    }

    \solution{
        
    }

    \subquestion{6}{
        Bereken dat minstens de helft van de doorzoekingen sneller is afgerond dan $15$ minuten.
    }

    \solution{

    }

    \subquestion{8}{
        Wat is de kans dat de totale doorzoektijd over de $n=29$ geselecteerde auto's langer is dan acht uur?
    }

    \solution{

    }

    \subquestion{8}{
        De Marechaussee kan de doorzoeking van auto's mogelijk verkorten door gebruik te maken van drugshonden.
        Hierbij wordt ervanuit gegaan dat de doorzoektijd nog steeds uniform verdeeld zal zijn, maar verschoven naar links (oftewel van $[11, 17]$ naar $[11-x, 17-x]$ voor een bepaalde waarde van $x$).
        Voor welke waarde van $x$ geldt dat de kans dat de $n=29$ geselecteerde auto's binnen zeven uur kunnen worden gedetecteerd nu groter is dan \SI{90}{\percent}.
    }

    \solution{

    }

\end{question}