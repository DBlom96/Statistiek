\begin{question}{25}{
    De dreiging van verstoringen van onze kritieke infrastructuur op zee is een reeël en urgent probleem.
    De Nederlandse defensie wil deze dreiging het hoofd bieden door het inzetten van patrouilleschepen om spionageschepen van vijandelijke mogendheden
    te detecteren op een populaire vaarroute in de Noordzee.

    Uit historische data is op te maken dat het aantal te detecteren spionageschepen $X$ per dag de volgende discrete kansverdeling volgt:

    \begin{center}
        \begin{tabular}{ccccccccc}
            \multicolumn{9}{c}{\textbf{Aantal spionageschepen per dag}} \\
            \toprule
                $k$ & $0$ & $1$ & $2$ & $3$ & $4$ & $5$ & $6$ & $7$ \\
            \midrule
                $f(k) = P(X=k)$ & $0.06$ & $0.13$ & $0.16$ & $0.11$ & $0.32$ & $0.16$ & $0.06$ \\
            \bottomrule
        \end{tabular}
    \end{center}
}    
    \subquestion{3}{
        Teken het naalddiagram van deze discrete kansverdeling.
    }

    \solution{
    
    }

    \subquestion{4}{
        Toon aan dat deze kansverdeling voldoet aan de twee voorwaarden waaraan een discrete kansverdeling moet voldoen.
    }

    \subquestion{7}{
        Bereken de verwachtingswaarde $E[X]$ en de standaardafwijking $\sigma(X)$ van het aantal spionageschepen per dag.
    }
    \solution{

    }

    \subquestion{4}{
        Bereken de kans dat er op een willekeurige dag minstens $4$ spionageschepen worden gedetecteerd.
    }

    \solution{

    }

    \subquestion{7}{
        Het probleem met spionageschepen is dat we aan het begin van de dag niet weten hoeveel spionageschepen er die dag zullen passeren.
        Voor ieder spionageschip geldt dat er minstens één patrouilleschip nodig is om het terug te begeleiden naar internationale wateren, maar het inzetten van een patrouilleschip is duur.
        Er wordt besloten om het aantal patrouilleschepen $Y$ per dag in te zetten volgens een binomiale verdeling met $n = 8$ en $p = 0.5$.
        Wat is de kans dat er op een willekeurige dag voldoende patrouilleschepen zijn om alle spionageschepen die dag te kunnen detecteren en terug te begeleiden?     
        \textit{Hint:} maak een soortgelijke kanstabel voor deze nieuwe discrete kansvariabele $Y$.
    }

    \solution{

    }
\end{question}