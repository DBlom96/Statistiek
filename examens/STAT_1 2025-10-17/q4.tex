\begin{question}{20}{
    Sanquin houdt een reclamecampagne onder de burgerbevolking om bloed te komen doneren voor gewonde soldaten.
    Historisch gezien komen er tijdens zo'n campagne per uur $3$ donaties binnen.
    Het aantal donaties per uur kan worden gemodelleerd aan de hand van een Poissonproces.

    De bloedgroepen die vóórkomen onder de burgerbevolking zijn als volgt:
    \begin{center}
        \begin{tabular}{cccc}
            \toprule
                \textbf{O} & \textbf{A} & \textbf{B} & \textbf{AB} \\
            \midrule
                \SI{46}{\percent} & \SI{40}{\percent} & \SI{10}{\percent} & \SI{4}{\percent}\\
            \bottomrule
        \end{tabular}
    \end{center}
    
    Aangenomen wordt dat de donaties die binnenkomen ook deze verdeling van bloedgroepen volgt.
} 
    \subquestion{2}{
        Wat is het verwachte aantal donaties van universele donors (bloedgroep $O$) in het komende uur?
    }

    \solution{
        Gegeven is dat het aantal donaties per uur tijdens de campagne gemiddeld op $3$ ligt.
        Als we mogen aannemen dat de donaties proportioneel binnenkomen volgens de verdeling van bloedgroepen, dan houdt dat in dat er gemiddeld $0.46 \cdot 3 = 1.38$ donaties per uur binnenkomen. \rubric{2}
    }

    \subquestion{6}{
        In het veldhospitaal liggen nu vijf gewonde soldaten met bloedgroep $A$. 
        Deze soldaten kunnen bloed ontvangen van bloedgroep $O$ of bloedgroep $A$.
        Wat is de kans dat alle soldaten bloed kunnen ontvangen binnen de komende twee uur?
    }

    \solution{
        Opnieuw geldt dat de donaties volgens een Poissonproces aankomen.
        Merk op dat de respectievelijk \SI{46}{\percent} en \SI{40}{\percent} van de donaties van bloedgroepen $O$ en $A$ komen, oftewel \SI{86}{\percent} van de donaties. \rubric{1}
        In andere woorden, de donaties van bloed van bloedgroepen $O$ en $A$ komen aan volgens een Poissonproces met gemiddelde $\lambda = 0.86 \cdot 3 = 2.58$ donaties per uur. \rubric{1}
        Alle soldaten kunnen bloed ontvangen indien er minstens zoveel donaties de komende twee uur binnenkomen. \rubric{1}
        Stel nu dat $X$ het aantal donaties van bloedgroepen $O$ en $A$ samen is in de komende twee uur.
        Dan geldt dat $X$ Poisson verdeeld is met parameter $\mu = \lambda \cdot t = 2.58 \cdot 2 = 5.16$. \rubric{1}
        De kans dat er genoeg donaties binnenkomen om alle gewonde soldaten te helpen is dus gelijk aan
        \begin{align*}
            P(X \geq 5) = 1 - P(X \leq 4) = 1 - \poissoncdf(\mu = 5.16, k=4) \approx 0.5871. \rubric{2}
        \end{align*}

    }

    \subquestion{4}{
        Soldaten van bloedgroep $B$ kunnen bloed ontvangen van mensen met bloedgroep $O$ of bloedgroep $B$.
        Leg zonder berekening uit of de kans dat vijf gewonde soldaten met bloedgroep $B$ binnen twee uur bloed kunnen ontvangen groter of kleiner is dan je antwoord bij vraag 4b.
    }

    \solution{
        Merk op dat de bloedgroep $B$ minder vaak voorkomt dan bloedgroep $A$ (\SI{10}{\percent} < \SI{40}{\percent}). \rubric{1}
        Dit houdt dus in, indien de donaties proportioneel zijn over de bloedgroepen, dat er minder donaties binnenkomen om mensen met bloedgroep $B$ te bedienen. \rubric{1}
        De kans dat er genoeg donaties binnenkomen om vijf gewonde soldaten met bloedgroep $B$ te helpen is dan dus ook kleiner dan de kans dat er genoeg donaties binnenkomen voor vijf gewonde soldaten met bloedgroep $A$. \rubric{2}
    }

    \subquestion{8}{
        Na hoeveel donaties is de kans groter dan \SI{95}{\percent} dat vijf gewonde soldaten met bloedgroep $O$ een bloeddonatie kunnen ontvangen? Pati\"enten met bloedgroep $O$ kunnen alleen bloed ontvangen van bloedgroep $O$. (Hierbij bekijken we donaties van alle bloedgroepen, dus ook onbruikbare bloedgroepen $A$, $B$ en $AB$)
    }

    \solution{
        In deze situatie hebben we te maken met een discrete kansvariabele $X$ die het aantal bruikbare donaties telt voor pati\"enten met bloedgroep $O$. \rubric{1}
        Merk op dat \SI{46}{\percent} van de donaties van bloedgroep $O$ zijn, en dus bruikbaar voor pati\"enten van bloedgroep $O$.

        De kansvariabele $X$ dus binomiaal verdeeld is met als parameters een nader te bepalen $n$ en een succeskans $p = 0.46$. \rubric{2}

        We willen nu de waarde van $n$ bepalen zodanig dat
        \begin{align*}
            P(X \geq 5) = 1 - P(X \leq 4) = 1 - \binomcdf(n?, p=0.46, k=4) \geq 0.95. \rubric{2}
        \end{align*}

        Voer deze vergelijking in in de tabel optie van de grafische rekenmachine (merk op: $n$ moet een geheel getal zijn):
        \begin{itemize}
            \item[] $y_1: 1 - \binomcdf(n?, p=0.86, k=4)$
            \item[] $y_2: 0.95$ \rubric{1}
        \end{itemize}

        Uit de tabel volgt dan dat:
        \begin{itemize}
            \item[] $n = 17: P(X \geq 5) = 0.94953$
            \item[] $n = 18: P(X \geq 5) = 0.96581$ \rubric{1}
        \end{itemize}
        
        Er moeten dus minstens $18$ donaties binnenkomen om minstens \SI{95}{\percent} kans te hebben op minstens vijf donaties van bloedgroep $O$. \rubric{1}
    }

\end{question}