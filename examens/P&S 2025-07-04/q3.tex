\begin{enquestion}{15}{
    A naval research unit investigates the signal strength (in decibels, dB) of sonar pulses reflected from a newly designed stealth submarine hull.

    Under standard conditions, the mean reflected signal strength from a conventional hull is $\mu_0 = 65$ dB, with unknown standard deviation $\sigma$ dB.

    To assess whether the new stealth design reduces detectability, a test series is conducted. Signal strength is measured in 10 independent trials with the following results (in dB):

    \[
    \{63.2,\ 64.1,\ 62.5,\ 63.8,\ 65.0,\ 64.7,\ 63.5,\ 62.9,\ 63.6,\ 64.2\}
    \]

    Assume the measurements follow a normal distribution. Use a significance level of $\alpha = 0.05$.
}

    \subquestion{4}{Determine the mean and standard deviation of this sample.}
    \ensolution{
        We can compute the sample mean and the sample variance as follows:
        \[
            \overline{x} = \frac{x_{1}+x_{2}+\ldots+x_{n}}{n} = \frac{63.2+64.1+\ldots+64.2}{10} = \frac{637.5}{10} = 63.75
        \]
        \rubric{1}
        \begin{align*}
            s^2     &= \frac{1}{n-1}\sum_{i=1}^n (x_i - \overline{x})^2 \\
                    &= \frac{1}{10-1} \cdot \left( (63.2-63.75)^2 + (64.1-63.75)^2 + \ldots + (64.2-63.75)^2\right) \\
                    &\approx 0.6072 \rubric{2}
        \end{align*}
        

        The sample standard deviation is then obtained by taking the square root of the sample variance:
        \[
            s = \sqrt{0.62} \approx 0.7792 \rubric{1}
        \]
    }
    
    \subquestion{3}{
        State the null and alternative hypotheses, and explain the direction of the test.
    }
    \ensolution{
        Since we aim at testing whether the mean reflected signal strength is still $\mu_0 = 65$ or if it has decreased, the null hypothesis $H_0$ and the alternative hypothesis $H_1$ of this test can be formulated as follows:
        \begin{align*}
            H_0&: \mu = 65 \quad \text (no change in reflectivity) \rubric{1} \\
            H_1&: \mu < 65 \quad \text{(stealth hull reduces reflectivity)} \rubric{1}
        \end{align*}
        This is a \textbf{one-sided} left-tailed test.
    }   
    \subquestion{9}{Perform the hypothesis test for the mean reflected signal strength $\mu$ based on the critical region.}

    \ensolution{
        Since the sample size $n=10$ is smaller than 30, and the population standard deviation $\sigma$ is unknown, we need to resort to the $t$-distribution.\rubric{1}
        This $t$-distribution has $\text{df}=n-1=9$ degrees of freedom.\rubric{1}

        We want to compute a critical region for the population mean $\mu$ with significance level $\alpha = 0.05$.
        Since the hypothesis is left-tailed, the critical region is of the form $(-\infty, g]$, where \rubric{1}
        The $t$-value is equal to
        \[
            t = \invt(\text{area}=1 - \alpha; \text{df}=n-1) = \invt(\text{area}=0.95; \text{df}=9) = 1.8331. \rubric{2}
        \]
        In particular, we can compute the boundary $g$ as follows
        \begin{align*}
            g   &= \mu - t \cdot \frac{s}{\sqrt{n}} \\
                &= 65 - 1.8331 \cdot \frac{0.7792}{\sqrt{10}} \\
                &\approx 64.5483 \rubric{2}
        \end{align*}  
        
        The critical region is therefore be given by $(-\infty; 64.5483]$.\rubric{1}
        The sample mean $\overline{x}=63.75$ lies in the critical region, hence we reject the null hypothesis $H_0$.\rubric{1}
        }

    
    \subquestion{4}{Calculate the probability of a Type-II error $\beta$ if the true reflected signal strength is actually normally distributed with $\mu = 64.5$ en $\sigma=0.8$ dB (for a single observation).}

    \ensolution{
        We computed in the previous subquestion the critical region, which was equal to $(-\infty; 64.5483]$.
        Therefore, we need to compute the probability that given $\mu=64$ en $\sigma=0.8$, we get a value inside the acceptable region. \rubric{1}

        In other words:
        \begin{align*}
            \beta   &= P(\overline{X} < 64.5483 \mid \mu = 64.5) \\
                    &= \normalcdf(\text{lower}=-10^{99}; \text{upper}=64.5483; \mu=64.5; \sigma=\frac{0.8}{\sqrt{10}}) \\
                    &\approx 0.5757 \rubric{2}
        \end{align*}

        So, $\beta \approx 0.5757$: a $57,57\%$ chance of accepting the null hypothesis while it is incorrect in reality. \rubric{1}
=    }
    \end{enquestion}
