\begin{enquestion}{18}{
    In a test for the endurance of soldiers, research is conducted on the relationship between the load weight carried by the soldier ($X$) and the time needed for completing a 3 km speed march ($Y$).
    The following data were collected on 12 soldiers:

    \begin{center}
        \begin{tabular}{c|cccccccccccc}
            \toprule
                $\mathbf{X}$ & $12$ & $14$ & $16$ & $18$ & $20$ & $22$ & $24$ & $26$ & $28$ & $30$ & $15$ & $25$ \\
            \midrule
                $\mathbf{Y}$ & $16.75$ & $16.29$ & $17.97$ & $19.78$ & $17.65$ & $18.15$ & $21.37$ & $20.65$ & $19.3$ & $21.31$ & $16.05$ & $18.55$ \\
            \bottomrule
        \end{tabular}
    \end{center}

    The research team believes that there is a linear relationship between the load weight carried by the soldier and the time needed for completing the speed march.
}
    
    \subquestion{8}{
        Calculate the least-squares estimates $\hat{\beta}_0$ and $\hat{\beta}_{1}$ for the slope and intercept of the linear regression line $Y = \beta_0 + \beta_1 X$, where $Y$ is the speed march time, and $X$ is the load weight carried.
    }

    \ensolution{
        We start by generating a table based on the data, the sample sums and means can then be used to calculate the least-squares estimated $\hat{\beta}_0$ and $\hat{\beta}_{1}$:
        \begin{center}
            \begin{tabular}{ccccc}
                \toprule
                    $X$ & $Y$ & $XY$ & $X^2$ & $Y^2$ \\
                \midrule
                    $12$ & $16.75$ & $201.0$ & $144$ & $280.5625$ \\
                    $14$ & $16.29$ & $228.06$ & $196$ & $265.3641$ \\
                    $16$ & $17.97$ & $287.52$ & $256$ & $322.9209$ \\
                    $18$ & $19.78$ & $356.04$ & $324$ & $391.2484$ \\
                    $20$ & $17.65$ & $353$ & $400$ & $311.5225$ \\
                    $22$ & $18.15$ & $399.3$ & $484$ & $329.4225$ \\
                    $24$ & $21.37$ & $512.88$ & $576$ & $456.6769$ \\
                    $26$ & $20.65$ & $536.9$ & $676$ & $426.4225$ \\
                    $28$ & $19.30$ & $540.4$ & $784$ & $372.49$ \\
                    $30$ & $21.31$ & $639.3$ & $900$ & $454.1161$ \\
                    $15$ & $16.05$ & $240.75$ & $225$ & $257.6025$ \\
                    $25$ & $18.55$ & $463.75$ & $625$ & $344.1025$ \\
                \midrule
                    $\overline{X} = 20.8333$ & $\overline{Y} = 18.6517$ & $\overline{XY} = 396.5750$ & $\overline{X^2} = 465.8333$ & $\overline{Y^2} = 351.0376$ \rubric{4}\\
                \bottomrule
            \end{tabular}
        \end{center}
        
        
        Using the standard formulas for the least-squares estimates, we get
        \begin{align*}
            \hat{\beta}_1 &= \frac{\overline{XY} - \overline{X} \cdot \overline{Y}}{\overline{X^2} - (\overline{X})^2} \\
                    &= \frac{396.575 - 20.833 \cdot 18.652}{465.833 - (20.833)^2} \\
                    &= \frac{7.999}{31.806} = 0.2515 \\
            \hat{\beta}_0 &= \overline{Y} - \beta_1 \cdot \overline{X} \\
            &= 18.652 - 0.2515 \cdot 20.833 \\
            &= 13.4124.\rubric{3}
        \end{align*}
        The equation of the regression line is thus equal to $Y = 13.4124 + 0.2515X$. \rubric{1}
    }
    
    \subquestion{4}{
        Interpret the slope $\beta_{1}$ of the regression model in the context of this exercise. 
        What does the slope suggest about the relationship between load weight and speed march time?
    }
    \ensolution{
        In the context of this question, the slope $\beta_{1}$ of the regression tells us by how much more time it takes to finish a 3km speed march if the load increases by one kilogram. \rubric{1}
        Since the least-squares estimate $\hat{\beta}_1$ of the slope is positive, this suggests that the speed march time increases whenever the load weight increases. \rubric{2}
        Specifically, for every additional kilogram of load weight, the predicted time to complete the speed march increases by around a quarter of a minute (15 seconds).\rubric{1}
    }

    \subquestion{6}{
        Calculate the correlation coefficient $R(X,Y)$ between the load weight and speed march time. 
        Based on the value of the correlation coefficient, what can you conclude about the strength and direction of the relationship between the two variables?
    }
    \ensolution{
        The correlation coefficient $R(X,Y)$ can be computed as follows:
        \begin{align*}
            R(X, Y) &= \frac{\overline{X \cdot Y} - \overline{X} \cdot \overline{Y}}{\sqrt{(\overline{X}^2 - \overline{X^2}) \cdot (\overline{Y}^2 - \overline{Y^2})}} \\
            &= \frac{396.575 - 20.833 \cdot 18.652}{\sqrt{(20.833^2 - 465.833) \cdot (18.652^2 - 351.038)}} \\
            &= \frac{7.999}{10.014} \\
            &\approx 0.7987.\rubric{4}
        \end{align*}
        
        The value of the correlation coefficient is positive and relative close to one, which suggests a strong positive correlation between load weight and speed march time.
        This makes sense, as speed marches are going to be harder once you have to carry a higher load weight with you.\rubric{2}
    }
\end{enquestion}