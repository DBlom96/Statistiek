\question{3}{Binnen het NAVO-bondgenootschap wordt over een periode van zes maanden door een internationaal cybersecurity team inlichtingen uitgewisseld over cyberaanvallen gericht op de Baltische staten: Estland, Letland en Litouwen.
Hierbij worden de cyberaanvallen van drie belangrijke types gemonitord, namelijk phishing aanvallen, het injecteren van malware en DDoS-aanvallen (Distributed Denial-of-Service).
De verzamelde data ziet er als volgt uit:

\begin{center}
    \resizebox{0.95\textwidth}{!}{
        \begin{tabular}{c|ccc|c}
            \toprule
                Land & Phishing & Malware & DDoS & \textbf{Totaal}\\
            \midrule 
                Estland & 120 & 200 & 140 & 460 \\
                Letland & 150 & 170 & 180 & 500 \\
                Litouwen & 100 & 220 & 160 & 480 \\
            \midrule
                \textbf{Totaal} & 89,6\% & 10,4\% & 100\% \\
            \bottomrule
        \end{tabular}
    }
\end{center}
Het cybersecurity team wil testen of de verdeling van het type inkomende cyberaanvallen significant verschilt tussen de drie Baltische staten.
}

  \subquestion{[3] Formuleer de nulhypothese $H_0$ en de alternatieve hypothese $H_1$ van de bijbehorende hypothesetoets.}
    \answer{

    }

    \subquestion{[10] Voer een chi-kwadraattoets voor onafhankelijkheid uit met significantieniveau $\alpha=0,05$ en bereken de bijbehorende $p$-waarde.}
    \answer{

    }

    \subquestion{[7] Kunnen we concluderen dat de verdeling van het type inkomende cyberaanvallen significant verschilt voor de drie Baltische staten?}
    \answer{

    }
