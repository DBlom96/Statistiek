\begin{question}{30}{
    Een luchtmacht onderzoekt de nauwkeurigheid van GPS-geleide raketten in een omgeving van elektronische oorlogsvoering.
    De raketten worden gericht op een vijandelijke luchtmachtbasis.
    Tijdens de vlucht worden de raketten ``gedesori\"enteerd'' omdat de vijandelijke troepen gebruik maken van GPS jammers om precisiewapens te verstoren.
    Onderzocht wordt nu wat het verband is tussen de afstand (in km) tussen de eerste locatie waar de raket gejamd wordt en het doelwit ($X$) en de afstand (in km) tussen het doelwit en het daadwerkelijke inslagpunt van de raket ($Y$).
    Van tien raketten worden data verzameld over deze twee variabelen.
    \begin{center}
        \begin{tabular}{c|cccccccccc}
            \toprule
                \textbf{$X$} & $30$ & $50$  & $70$  & $90$  & $110$  & $130$ & $150$ & $170$ & $190$ & $210$ \\ 
                \textbf{$Y$} & $0.7$ & $1.8$ & $2.4$ & $3.3$ & $4.8$ & $5.3$ & $6.6$ & $7.1$ & $8.4$ & $9.2$ \\
            \bottomrule
        \end{tabular}
    \end{center}
}

    \subquestion{5}{Teken de gegevens uit bovenstaande tabel in een spreidingsdiagram.} 
    \solution{

    }

    \subquestion{6}{Bereken Pearson's correlatieco\"efficient voor $X$ en $Y$. 
    Bepaal of er sprake is van een lineaire correlatie en leg in woorden uit wat de betekenis is van de grootte en het teken van de correlatieco\"efficient.
    }
    \solution{

    }

    \subquestion{8}{Bereken de regressielijn $Y = aX + b$ met behulp van de tabel hierboven.}
    \solution{

    }

    \subquestion{3}{Geef een statistisch verantwoorde voorspelling van de afstand tot het doelwit waarop een raket inslaat die op $2,5$ kilometer van het doelwit voor het eerst GPS jamming ondervindt.}
    \solution{

    }

    \subquestion{8}{Bepaal een $95\%$-voorspellingsinterval voor de afstand tussen doelwit en inslagpunt als die op $120$ kilometer van het doelwit voor het eerst GPS-jamming ondervindt.}
    \solution{

    }

\end{question}