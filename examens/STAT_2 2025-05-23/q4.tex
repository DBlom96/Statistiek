\begin{question}{25}{
    Een marinefregat opereert in een gebied waar regelmatig vijandelijke onderzeeboten actief zijn.
    Het fregat gebruikt hiervoor sonar en magnetische sensoren om onderzeeboten te detecteren.
    De bemanning wil analyseren hoe vaak ze per dag succesvolle detecties uitvoeren en of hun waarnemingen een Poisson-verdeling volgen.

    Gedurende een periode wordt het aantal gedetecteerde vijandelijke onderzeeboten geteld.
    \begin{center}
        \begin{tabular}{cc}
            \toprule
                \textbf{Aantal gedetecteerde onderzeeboten} & \textbf{Aantal dagen} \\
            \midrule
                $0$ & $40$ \\
                $1$ & $80$ \\
                $2$ & $100$ \\
                $3$ & $85$ \\
                $4$ & $35$ \\
                $5$ & $15$ \\
                $6$ & $7$ \\
                $7$ & $3$ \\
            \midrule
                \textbf{Totaal:} & 365\\
            \bottomrule
        \end{tabular}
    \end{center}
}
    \subquestion{5}{
        Bereken met behulp van de gegevens uit de tabel het aantal dagen waarop er data is verzameld en het gemiddelde aantal gedetecteerde vijandelijke onderzeeboten per dag.
    }
    \solution{

    }

    \subquestion{10}{
        Voer een geschikte hypothesetoets uit om de bewering te toetsen dat het aantal gedetecteerde onderzeeboten per dag een Poisson-verdeling volgt met gemiddelde $\lambda=2$.
        Bepaal de toetsuitslag door het berekenen van de $p$-waarde, met significantieniveau $\alpha = 0,05$.
    } 
    \solution{

    }

    \subquestion{10}{
        De marine definieert een dag met hoge vijandelijke onderzeese activiteit als een dag waarop minstens vier onderzeeboten worden gedetecteerd.
        Bereken met de methode van Clopper-Pearson een $95\%$-betrouwbaarheidsinterval voor de kans $p$ dat er op een willekeurige dag hoge vijandelijke activiteit is. 
    }
    \solution{

    }
\end{question}