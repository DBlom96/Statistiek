\begin{question}{25}{
    Een marinefregat opereert in een gebied waar regelmatig vijandelijke onderzeeboten actief zijn.
    Het fregat gebruikt hiervoor sonar en magnetische sensoren om onderzeeboten te detecteren.
    De bemanning wil analyseren hoe vaak ze per dag succesvolle detecties uitvoeren en of hun waarnemingen een Poissonverdeling volgen.

    Gedurende een periode wordt het aantal gedetecteerde vijandelijke onderzeeboten geteld.
    \begin{center}
        \begin{tabular}{cc}
            \toprule
                \textbf{Aantal gedetecteerde} & \multirow{2}{*}{\textbf{Aantal dagen}} \\
                \textbf{onderzeeboten}        & \\
            \midrule
                $0$ & $40$ \\
                $1$ & $80$ \\
                $2$ & $100$ \\
                $3$ & $85$ \\
                $4$ & $35$ \\
                $5$ & $15$ \\
                $6$ & $7$ \\
                $7$ & $3$ \\
            % \midrule
            %     \textbf{Totaal:} & 365\\
            \bottomrule
        \end{tabular}
    \end{center}
}
    \subquestion{5}{
        Bereken met behulp van de gegevens uit de tabel het aantal dagen waarop er data is verzameld en het gemiddelde aantal gedetecteerde vijandelijke onderzeeboten per dag.
    }
    \solution{
        Het aantal dagen $n$ waarop er data is verzameld is gelijk aan
        \[
            n = 40 + 80 + 100 + 85 + 35 + 15 + 7 + 3 = 365 \rubric{1}
        \]
        Het totaal aantal detecties is gelijk aan
        \[
            \text{\#detecties} = 0 \cdot 40 + 1 \cdot 80 + 2 \cdot 100 + 3 \cdot 85 + 4 \cdot 35 + 5 \cdot 15 + 6 \cdot 7 + 7 \cdot 3 = 813 \rubric{2}
        \]
        Het gemiddelde aantal detecties is dus gelijk aan
        \[
            \text{gemiddelde} = \frac{\text{\#detecties}}{n} = \frac{813}{365} \approx 2.2274. \rubric{2}
        \]
    }

    \subquestion{12}{
        Toets of het aantal gedetecteerde onderzeeboten per dag $X$ is te beschouwen als een Poisson verdeelde kansvariabele.
        Schrijf hierbij specifiek de nul- en alternatieve hypothese uit en bepaal de toetsuitslag aan de hand van het berekenen van een $p$-waarde.
        Kies als betrouwbaarheid $95\%$ en gebruik in je berekening je antwoord op vraag 4a.
    } 
    \solution{
        Om deze bewering te toetsen, moeten we werken met de volgende hypothesen:
        \begin{align*}
            H_{0}: \text{ de kansvariabele $X$ volgt een Poissonverdeling.} \\
            H_{1}: \text{ de kansvariabele $X$ volgt NIET een Poissonverdeling.}\rubric{2}
        \end{align*}

        Omdat we een toets willen uitvoeren om te testen of een kansvariabele een bepaalde kansverdeling volgt, moeten we een chi-kwadraattoets voor aanpassing gebruiken.
        Om de toetsingsgrootheid $\chi^2$ te berekenen, hebben we allereerst de verwachte frequenties nodig als we zouden werken met een Poissonverdeling met parameter $\lambda=\frac{813}{365}$. \rubric{1}
        
            \begin{center}
                \scriptsize
                \begin{tabular}{ccc}
                    \toprule
                        \textbf{Aantal gedetecteerde} & \textbf{Aantal dagen} & \textbf{Aantal dagen} \\
                        \textbf{onderzeeboten}        & \textbf{(observed)} & \textbf{(expected)}\\
                    \midrule
                        $0$ & $40$ & $365 \cdot \poissonpdf(\lambda=\frac{813}{365}; k=0) = 39.3502$\\
                        $1$ & $80$ & $365 \cdot \poissonpdf(\lambda=\frac{813}{365}; k=1) = 87.6484$\\
                        $2$ & $100$ & $365 \cdot \poissonpdf(\lambda=\frac{813}{365}; k=2) = 97.6140 $\\
                        $3$ & $85$ & $365 \cdot \poissonpdf(\lambda=\frac{813}{365}; k=3) = 72.4750$\\
                        $4$ & $35$ & $365 \cdot \poissonpdf(\lambda=\frac{813}{365}; k=4) = 40.3577$\\
                        $5$ & $15$ & $365 \cdot \poissonpdf(\lambda=\frac{813}{365}; k=5) = 17.9785$\\
                        $6$ & $7$ & $365 \cdot \poissonpdf(\lambda=\frac{813}{365}; k=6) = 6.6742$\\
                        $7$ & $3$ & $365 \cdot (1 - \poissoncdf(\lambda=\frac{813}{365}; k=6)) = 2.9020$\\ 
                    \bottomrule
                \end{tabular}\rubric{3}
            \end{center}
        
        Omdat er verwachte frequenties zijn die kleiner dan 5 zijn (vuistregel), moeten we rijen gaan samenvoegen, namelijk de laatste twee rijen:
        \begin{center}
            \footnotesize
            \begin{tabular}{ccc}
                \toprule
                    \textbf{Aantal gedetecteerde} & \textbf{Aantal dagen} & \textbf{Aantal dagen} \\
                    \textbf{onderzeeboten}        & \textbf{(observed)} & \textbf{(expected)}\\
                \midrule
                    $0$ & $40$ & $39.3502$\\
                    $1$ & $80$ & $87.6484$\\
                    $2$ & $100$ & $97.6140 $\\
                    $3$ & $85$ & $72.4750$\\
                    $4$ & $35$ & $40.3577$\\
                    $5$ & $15$ & $17.9785$\\
                    $\ge 6$ & $10$ & $9.5763$\\
                \bottomrule
            \end{tabular}\rubric{1}
        \end{center}
        De toetsingsgrootheid kunnen we nu bepalen als volgt:
        \begin{align*}
            \chi^2  &= \frac{(O_{0} - E_{0})}{E_{0}} + \frac{(O_{1} - E_{1})}{E_{1}} + \ldots + \frac{(O_{\ge 6} - E_{\ge 6})}{E_{\ge 6}} \\
                    &= \frac{(40 - 39.3502)}{39.3502} + \frac{(80 - 87.6484)}{87.6484} + \ldots + \frac{(10 - 9.5763)}{9.5763} \\
                    &\approx 4.1245 \rubric{2}
        \end{align*} 
        Aangenomen dat de nulhypothese waar zou zijn, is deze toetsingsgrootheid getrokken uit een chi-kwadraatverdeling met aantal vrijheidsgraden 
        \[
            df = \#\text{categorie\"en} - 1 - \#\text{geschatte parameters} = 7 - 1 - 1 = 5.
        \]
        De $p$-waarde die hoort bij deze uitkomst is 
        \[
            p = P(\chi^2 > 4.1245) = \chi^2\text{cdf}(4.1245; 10^{99}; 5) \approx 0.5316. \rubric{2}
        \]
        Tevens is er voor een betrouwbaarheidsniveau van $95\%$ ($\alpha=0.05$) gekozen.
        Omdat $p > \alpha$, wordt $H_0$ niet verworpen. Er is onvoldoende bewijs om de bewering te verwerpen dat de kansvariabele een Poissonverdeling zou volgen.\rubric{1}
    }

    \subquestion{8}{
        De marine definieert een dag met hoge vijandelijke onderzeese activiteit als een dag waarop minstens vier onderzeeboten worden gedetecteerd.
        Bereken met de methode van Clopper-Pearson een $95\%$-betrouwbaarheidsinterval voor de kans $p$ dat er op een willekeurige dag hoge vijandelijke activiteit is. 
    }
    \solution{
        Volgens de gegevens van de marine zijn op $35+15+7+3=60$ van de $365$ dagen minstens vier onderzeeboten gedetecteerd.\rubric{1}
        Een puntschatting voor de kans dat er op een willekeurige dag hoge vijandelijke onderzeese activiteit is, is gelijk aan $\hat{p} = \frac{60}{365}=0.1644$. \rubric{1}

        Voor het $95\%$-betrouwbaarheidsinterval (dus $\alpha=0.05$) berekenen we de grenzen aan de hand van
        \begin{enumerate}
            \item We berekenen $p_1$ door de vergelijking $P(X \le 60) = \frac{\alpha}{2} = 0.025$ op te lossen. Hieruit volgt:
            \[
                \binomcdf(n=365; p_1=?; k=60) = 0.025 \rightarrow p_1 = 0.2065 \rubric{2}
            \]

            \item We berekenen $p_2$ door de vergelijking $P(X \ge 60) = 1 - P(X \le 59) = \frac{\alpha}{2} = 0.025$ op te lossen. Hieruit volgt:
            \[
                1 - \binomcdf(n=365; p_2=?; k=59) = 0.025 \rightarrow p_2 = 0.1278 \rubric{3}
            \]
        \end{enumerate}
        Dit levert het 95\% Clopper-Pearson betrouwbaarheidsinterval $[0.1278; 0.2065]$ voor de kans $p$ dat er op een willekeurige dag hoge vijandelijke onderzeese activiteit is. \rubric{1}
    }
\end{question}