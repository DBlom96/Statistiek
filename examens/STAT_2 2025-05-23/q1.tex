\begin{question}{25}{
    Op de Koninklijke Militaire Academie wordt onderzocht of een nieuw type gevechtstraining op basis van virtual reality de reactietijd van soldaten onder crisisomstandigheden significant verbetert.
    Soldaten worden ondergedompeld in een realistische VR-simulatie waarin ze snel moeten reageren en vijandelijke dreigingen moeten zien te identificeren.

    Om de effectiviteit te testen, meten instructeurs de reactietijd (in milliseconden) van 18 cadetten na 6 weken VR-training en vergelijken dit met de gemiddelde reactietijd van 420 ms bij conventionele trainingsmethoden.
    Uit de steekproef blijkt dat de gemiddelde reactietijd na de VR-training gelijk is aan 390 milliseconden met een standaardafwijking van 40 milliseconden.
    Er wordt aangenomen dat de reactietijd van een willekeurige cadet een normaal verdeelde kansvariabele is.
    }

    \subquestion{5}{
        Bereken op grond van deze steekproef een $98\%$-betrouwbaarheidsinterval voor de gemiddelde reactietijd van een cadet.
        Rond de grenzen van dit interval af op gehele milliseconden, zodanig dat de betrouwbaarheid gewaarborgd blijft.
    }
    \solution{
        Laat $X \sim N(\mu=?; \sigma=?)$ de reactietijd zijn van een willekeurige cadet.
        Volgens de centrale limietstelling is de gemiddelde reactietijd $\overline{X} = \frac{X_1 + X_2 + \ldots + X_{18}}{18}$ van 18 cadetten ook normaal verdeeld, met verwachtingswaarde $\mu$ en standaardafwijking $\frac{\sigma}{\sqrt{18}}$. \rubric{1}
        Omdat naast de verwachtingswaarde $\mu$ ook de standaardafwijking $\sigma$ onbekend is (en bovendien de steekproefgrootte $n = 18 < 30$), moet de $t$-verdeling worden gebruikt.\rubric{1}
        De $t$-waarde die hoort bij $98\%$ betrouwbaarheid, oftewel $\alpha = 0.02$, is (in het geval van tweezijdige intervallen) gelijk aantal
        \[
            t = \invt(opp=1-\frac{\alpha}{2}; df=n-1) = \invt(opp=0.99; df=17) \approx 2.5669. \rubric{1}
        \]
        Omdat $\sigma$ dus onbekend is, gebruiken we de steekproefstandaardafwijking $s = 40$ als schatter.
        Het betrouwbaarheidsinterval worden dan gegeven door
        \begin{align*}
            [\overline{x} - t \cdot \frac{s}{\sqrt{n}}; \overline{x} + t \cdot \frac{s}{\sqrt{n}}] 
                &= [390 - 2.5669 \cdot \frac{40}{\sqrt{18}}; 390 + 2.5669 \cdot \frac{40}{\sqrt{18}}] \\
                &= [365.7990; 414.2010] \rubric{2}
        \end{align*}
        Om de betrouwbaarheid te waarborgen, mag het interval niet kleiner worden, dus naar buiten afronden: $[365; 415]$. 
    }

    \subquestion{3}{
        Een van de instructeurs die verantwoordelijk is voor de VR-training beweert dat de gemiddelde reactietijd van cadetten daalt tot hoogstens $380$ milliseconden.
        De overige instructeurs willen dit niet meteen voor waar aannemen en willen dit graag statistisch verantwoord toetsen.
        Formuleer de bijbehorende nul- en alternatieve hypothese van een geschikte hypothesetoets.
    } 
    \solution{
        In deze hypothesetoets gaan we uit van een gegeven gemiddelde van $420$ milliseconden voor het conventionele trainingsregime.
        In dat geval hebben we te maken met een $t$-toets met nul- en alternatieve hypothese als volgt: \rubric{1}
        \begin{align*}
            H_{0}: \mu &\le 380 \tag{nulhypothese} \\ \rubric{1}
            H_{1}: \mu &> 380 \tag{alternatieve hypothese} \rubric{1}
        \end{align*}
    }

    \subquestion{8}{
        Voer de hypothesetoets uit. 
        Bepaal de toetsuitslag door het berekenen van een kritiek gebied op basis van de gegeven steekproef van 18 cadetten met significantieniveau $\alpha=0,05$.
    }
    \solution{
        In dit werken we met een rechtszijdige toets (alternatieve hypothese heeft vorm $>$), dus het kritieke gebied is van de vorm $(g, \infty)$.\rubric{1}
        We willen de (kleinste) grens $g$ zoeken zodanig dat de kans op een type-I fout ($H_0$ verwerpen terwijl die waar is) kleiner is dan $\alpha=0.05$.

        Onder de nulhypothese is de gemiddelde reactietijd $\overline{X}$ van 18 willekeurige gekozen cadetten normaal verdeeld met verwachtingswaarde $380$ en standaardafwijking $\frac{\sigma}{\sqrt{n}}$.\rubric{1}
        Omdat $\sigma$ onbekend is en $n=18<30$, gebruiken we opnieuw de $t$-verdeling en $s = 40$. \rubric{1}
        Er geldt dus, omdat we eenzijdig toetsen, dat 
        \[
            t = \invt(opp=1-\alpha; df=n-1) = \invt(opp=0.95; df=17) \approx 1.7396. \rubric{1}
        \]

        De bijbehorende grenswaarde vinden we nu met
        \[
            g = \mu + t \cdot \frac{s}{\sqrt{18}} = 380 + 1.7396 \cdot \frac{40}{\sqrt{18}} \approx 396.4012. \rubric{1}
        \]

        De gemeten waarde van $390$ is kleiner dan deze ondergrens $g \approx 396.4012$ van het kritieke gebied, dus $H_0$ wordt niet verworpen. 
        Dit betekent dat met deze toets niet kan worden aangetoond dat de gemiddelde reactietijd hoger wordt door de VR-gevechtstraining. \rubric{2}
    }

    \subquestion{4}{
        Bepaal opnieuw de toetsuitslag, nu door middel van berekening van de $p$-waarde.
    }

    \solution{
        Nog steeds geldt dat onder de nulhypothese geldt dat de gemiddelde reactietijd $\overline{X}$ van 18 willekeurige gekozen cadetten normaal verdeeld met verwachtingswaarde $380$ en standaardafwijking $\frac{\sigma}{\sqrt{n}}$.
        Omdat $\sigma$ onbekend is en $n=18<30$, gebruiken we opnieuw de $t$-verdeling en $s = 40$. \rubric{1}
        De $t$-waarde die hoort we een uitkomst van $\overline{x}=390$ is gelijk aan
        \[
            t = \frac{\overline{x} - \mu}{\frac{s}{\sqrt{n}}} = \frac{390 - 380}{\frac{40}{\sqrt{18}}} \approx 1.0607. \rubric{1}
        \]
        Er geldt dus, omdat we eenzijdig toetsen, dat we de rechteroverschrijdingskans willen bepalen:
        \[
            p = P(\overline{X} > t) = \text{tcdf}(lower=1.0607; upper=10^{99}; df=17) \approx 0.1518. \rubric{1}
        \]

        Aangezien deze overschrijdingskans $p$ groter is dan $\alpha$, zal $H_0$ opnieuw niet worden verworpen. \rubric{1}
    }

    \subquestion{5}{
        Stel nu dat de instructeurs niet de reactietijd van een individuele cadet willen meten, maar van een team van zes cadetten. 
        De teamscore wordt bepaald door de som te nemen van de reactietijden van de zes cadetten in het team.
        Elk team wil dus een zo laag mogelijke teamscore bereiken.

        Op dit moment staat het record op $x$ milliseconden.
        Voor welke waarde van $x$ is de kans op verbreking van het record gelijk aan $5\%$?
    }
    \solution{
        We noemen nu $Y$ de totale som van reactietijden van de zes cadetten in een team, dat is volgens de centrale limietstelling normaal verdeeld met verwachtingswaarde $6 \cdot \mu$ en standaardafwijking $\sqrt{6}\cdot\sigma$.
        We gebruiken hiervoor de schattingen $6\cdot\overline{x}=6\cdot390=2340$ ms en $\sqrt{6}\cdot s=\sqrt{6}\cdot 40 \approx 97.9796$ ms. \rubric{1}
        We willen nu de waarde $x$ vinden waarvoor geldt:
        \[
            P(Y \le x) = 0.05. \rubric{1}
        \]

        Omdat $n = 6 < 30$, werken we nog steeds met de $t$-verdeling, de $t$-waarde die daarbij hoort is
        \[
            t = \invt(opp=0.05; df=n-1) = \invt(opp=0.05; df=5) \approx -2.0150. \rubric{1}
        \]
        De grenswaarde van het interval is dan gelijk aan
        \[
            g = \mu + t \cdot s = 2340 - 2.0150 \cdot 97.9796 \approx 2142.5664 \rubric{1}
        \]
        Dit betekent dat het huidige record op deze grenswaarde $g = 2142.5664$ ms staat, dan zal een team van zes cadetten met $5\%$ kans het record kunnen breken. \rubric{1} 
    }

\end{question}