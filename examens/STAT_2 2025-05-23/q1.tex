\begin{question}{25}{
    De Koninklijke Militaire Academie onderzoekt of een nieuw type gevechtstraining op basis van virtual reality de reactietijd van soldaten onder crisisomstandigheden significant verbetert.
    Soldaten worden ondergedompeld in een realistische VR-simulatie waarin ze snel moeten reageren en vijandelijke dreigingen moeten zien te identificeren.

    Om de effectiviteit te testen, meten instructeurs de reactietijd (in milliseconden) van 18 cadetten na 6 weken VR-training en vergelijken dit met de gemiddelde reactietijd van 420 ms bij conventionele trainingsmethoden.
    Uit de steekproef blijkt dat de gemiddelde reactietijd na de VR-training gelijk is aan 390 milliseconden met een standaardafwijking van 40 milliseconden.
    }

    \subquestion{5}{
        Bereken op grond van deze steekproef een $95\%$-betrouwbaarheidsinterval voor de gemiddelde reactietijd van een cadet.
        Rond de grenzen van dit interval af op gehele milliseconden, zodanig dat de betrouwbaarheid gewaarborgd blijft.
    }
    \solution{

    }

    \subquestion{3}{
        De instructeurs willen de bewering dat de VR-training de reactietijd van cadetten significant verlaagt statistisch verantwoord toetsen.
        Formuleer de bijbehorende nul- en alternatieve hypothese van deze hypothesetoets.
    } 
    \solution{

    }

    \subquestion{8}{
        Voer de hypothesetoets uit. 
        Bepaal de toetsuitslag door het berekenen van een kritiek gebied op basis van de gegeven steekproef van 18 cadetten met significantieniveau $\alpha=0,02$.
    }
    \solution{

    }

    \subquestion{4}{
        Leg in eigen woorden uit wat de uitslag van deze toets betekent voor de gemiddelde reactietijd van een cadet na VR-training.
    }
    \solution{

    }

    \subquestion{5}{
        Stel nu dat de instructeurs niet de reactietijd van een individuele cadet willen meten, maar van een team van zes cadetten. 
        De teamscore wordt bepaald door de som te nemen van de reactietijden van de zes cadetten in het team.
        Elk team wil dus een zo laag mogelijke teamscore bereiken.

        Op dit moment staat het record op $x$ milliseconden.
        Voor welke waarde van $x$ is de kans op verbreking van het record gelijk aan $5\%$?
    }
    \solution{

    }

\end{question}