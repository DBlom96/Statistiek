\begin{question}{22}{
    Een onderzoeksteam van het Maritime Warfare Center (MWC) onderzoekt de signaalsterkte (in decibel) van sonarpulsen die worden gereflecteerd door een nieuw type onderzeeboot.
    Het huidige ontwerp heeft een gemiddelde gereflecteerde signaalsterkte van $\mu = 65$ decibel.

    Het MWC wil toetsen of het nieuw ontwerp de detecteerbaarheid reduceert, oftewel dat de gereflecteerde signaalsterkte significant lager ligt bij het nieuwe ontwerp.
    In een steekproef van tien onafhankelijke metingen is bij het nieuwe ontwerp de gereflecteerde signaalsterkte in decibel gemeten met gemiddelde $63,75$ decibel en standaardafwijking $s = 0,78$.
    Je mag aannemen dat de signaalsterktes normaal verdeeld zijn met onbekende standaardafwijking $\sigma$.
    
    Gebruik voor de hypothesetoets een significantieniveau van $\alpha=0,05$.
}
    \subquestion{4}{
       Definieer de nulhypothese en de alternatieve hypothese van de hypothesetoets.
       Verklaar het gekozen type (tweezijdig, linkszijdig of rechtszijdig) van de toets.
    }
    \solution{
        Aangezien we willen toetsen of de gemiddelde gereflecteerde signaalsterkte significant lager ligt dan $\mu = 65$, kunnen we de hypothesetoets als volgt defini\"eren:
        \begin{align*}
            H_0&: \mu \ge 65 \quad \text{(geen significante vermindering)} \rubric{2} \\
            H_1&: \mu < 65 \quad \text{(wel een significante vermindering)} \rubric{1}
        \end{align*}
        Dit is een linkszijdige toets, omdat de nulhypothese uitgaat van een status-quo (geen verbetering) en de alternatieve hypothese juist wel een verbetering aanduidt.
         \rubric{1}
    }
    
    \subquestion{4}{
       Voer de bijbehorende hypothesetoets uit met behulp van het kritieke gebied.
    }
    \solution{
        Aangezien de steekproefgrootte $n = 10$ kleiner is dan $30$ en de standaardafwijking $\sigma$ onbekend is, moeten we de $t$-verdeling gebruiken. \rubric{1}
        Deze $t$-verdeling heeft $\text{df}=n-1=9$ vrijheidsgraden.\rubric{1}

        De toetsingsgrootheid van onze hypothesetoets is gelijk aan
        \begin{align*}
            t = \frac{
        \end{align*}
        We willen een kritiek gebied bepalen voor het populatiegemiddelde $\mu$ met significantieniveau $\alpha = 0,05$.
        Dit doen we aan de hand van de $t$-waarde
        \[
            t = \invt(\text{area}=1 - \alpha; \text{df}=n-1) = \invt(\text{area}=0.95; \text{df}=9) = 1.8331. \rubric{2}
        \]
        Since the hypothesis is left-tailed, the critical region is of the form $(-\infty, g]$. \rubric{1}
        In particular, we can compute the boundary $g$ as follows
        \begin{align*}
            g   &= \mu - t \cdot \frac{s}{\sqrt{n}} \\
                &= 65 - 1.8331 \cdot \frac{0.7792}{\sqrt{10}} \\
                &\approx 64.5483 \rubric{2}
        \end{align*}  
        
        The critical region is therefore be given by $(-\infty; 64.5483]$.\rubric{1}
        The sample mean (test statistic) $\overline{x}=63.75$ lies in the critical region, hence we reject the null hypothesis $H_0$.
        Based on the selected sample, there is sufficient evidence to believe that the new stealth design indeed reduces detectability.\rubric{1}
        }

    
    \subquestion{4}{Calculate the probability of a Type-II error $\beta$ if the true reflected signal strength is actually normally distributed with $\mu = 64.5$ en $\sigma=0.8$ dB (for a single observation).}

    \solution{
        We computed in the previous subquestion the critical region, which was equal to $(-\infty; 64.5483]$.
        Therefore, we need to compute the probability that given $\mu=64$ en $\sigma=0.8$, we get a value inside the acceptable region. \rubric{1}

        In other words:
        \begin{align*}
            \beta   &= P(\overline{X} < 64.5483 \mid \mu = 64.5) \\
                    &= \normalcdf(\text{lower}=-10^{99}; \text{upper}=64.5483; \mu=64.5; \sigma=\frac{0.8}{\sqrt{10}}) \\
                    &\approx 0.5757 \rubric{2}
        \end{align*}

        So, $\beta \approx 0.5757$: a $57,57\%$ chance of accepting the null hypothesis while it is incorrect in reality. \rubric{1}
    }
\end{question}