\begin{question}{20}{
    Een onderzoeksteam van het Maritime Warfare Center (MWC) onderzoekt de signaalsterkte (in decibel) van sonarpulsen die worden gereflecteerd door een nieuw type onderzeeboot.
    Het huidige (oude) ontwerp heeft een gemiddelde gereflecteerde signaalsterkte van $\mu = \mu_0 = 65$ decibel.

    Het MWC wil toetsen of bij het nieuwe ontwerp de gemiddelde signaalsterkte significant verschilt van het oude ontwerp.
    In een steekproef van tien onafhankelijke metingen met het nieuwe ontwerp is de gereflecteerde signaalsterkte gemiddeld $63,75$ decibel met een standaardafwijking van $0,78$.
    Je mag aannemen dat de signaalsterktes normaal verdeeld zijn met onbekende standaardafwijking $\sigma$.
    
    Gebruik voor de hypothesetoets een significantieniveau van $\alpha=0,05$.
}
    \subquestion{5}{
       Definieer de nulhypothese en de alternatieve hypothese van de hypothesetoets.
       Verklaar het gekozen type (tweezijdig, linkszijdig of rechtszijdig) van de toets.
    }
    \solution{
        Aangezien we willen toetsen of de gemiddelde gereflecteerde signaalsterkte significant afwijkt van $\mu = \mu_0 = 65$, kunnen we de hypothesetoets als volgt defini\"eren:
        \begin{align*}
            H_0&: \mu = 65 \quad \text{(geen significant verschil)} \rubric{2} \\
            H_1&: \mu \neq 65 \quad \text{(wel een significant verschil)} \rubric{1}
        \end{align*}
        Dit is een tweezijdige toets, omdat de nulhypothese uitgaat van een status-quo (geen verschil) en de alternatieve hypothese juist wel een verschil aanduidt.\rubric{2}
    }
    
    \subquestion{9}{
       Voer de bijbehorende hypothesetoets uit met behulp van het kritieke gebied.
    }
    \solution{
        We mogen uitgaan van een significantieniveau van $\alpha = 0,05$ en verder is gegeven uit de steekproefdata dat $\overline{x} = 63,75$ en $s = 0,78$.
        Uit de centrale limietstelling volgt onder $H_0$ dat de toetsingsgrootheid $\overline{X}$ normaal verdeeld is met gemiddelde $\mu = \mu_0 = 65$ en standaardafwijking $\frac{\sigma}{\sqrt{n}}$. \rubric{1}
        
        Aangezien de steekproefgrootte $n = 10 < 30$ en de (populatie)standaardafwijking $\sigma$ onbekend is, moeten we de $t$-verdeling gebruiken. \rubric{1}
        
        Deze $t$-verdeling heeft $\text{df}=n-1=9$ vrijheidsgraden.\rubric{1}

        De bijbehorende $t$-waarde (tweezijdig toetsen) is gelijk aan 
        \[
            t = \invt(\text{opp}=1 - \alpha/2; \text{df}=n-1) = \invt(\text{opp}=0.975; \text{df}=9) \approx 2,2622. \rubric{2}
        \]
        
        Hieruit volgt een voorspellingsinterval voor $\overline{X}$ op basis van deze steekproef gelijk aan
        \begin{align*}
            & [\mu_0 - t \cdot \frac{s}{\sqrt{n}}; \mu_0 + t \cdot \frac{s}{\sqrt{n}}] \\
            = & [65 - 2,2622 \cdot \frac{0,78}{\sqrt{10}}; 65 + 2,2622 \cdot \frac{0,78}{\sqrt{10}}] \\
            \approx & [64,4420; 65,5580] \rubric{2}
        \end{align*}  
        
        Het gemeten steekproefgemiddelde $\overline{x} = 63,75$ ligt niet in dit voorspellingsinterval -- dus wel in het kritieke gebied -- dus verwerpen we de nulhypothese.\rubric{1}
        Er is voldoende reden om aan te nemen dat de gemiddelde gereflecteerde signaalsterkte significant afwijkt van het nieuwe ontwerp.\rubric{1}
    }

    
    \subquestion{6}{
        Bereken de kans op een type-II fout $\beta$ als geldt dat het daadwerkelijke populatiegemiddelde van het nieuwe ontwerp gelijk is aan $\mu = 64,5$ decibel, en dat de standaardafwijking $\sigma$ gelijk is aan $0,8$.
    }
    \solution{
        Bij een type-II fout wordt $H_0$ aangenomen, terwijl $H_1$ waar is. \rubric{1}
        In dit geval geldt dat $H_1$ waar is, omdat $\mu = 64,5 \neq 65$. \rubric{1}
        We willen dus de kans bepalen dat de toetsingsgrootheid een waarde in het acceptatiegebied van $H_0$ aanneemt (oftewel $[64,4420; 65,5580]$), gegeven $\mu = 64,5$. \rubric{1}

        In andere woorden:
        \begin{align*}
            \beta   &= P(64,4420 \le \overline{X} \le 65,5580 \mid \mu = 64.5) \\
                    &= \normalcdf(\text{lower}=64,4420 ; \text{upper}=65,5580; \mu=64.5; \sigma=\frac{0.8}{\sqrt{10}}) \\
                    &\approx 0,5907. \rubric{2}
        \end{align*}

        De kans op een type-II fout is dus gelijk aan $\beta \approx 0,5907$, oftewel $59\%$ kans dat de nulhypothese wordt aangenomen terwijl deze in werkelijkheid fout is.\rubric{1}
    }
\end{question}