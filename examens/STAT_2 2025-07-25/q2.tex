\begin{question}{28}{
    Bij een onderzoek binnen de Koninklijke Luchtmacht wordt gekeken naar de effectiviteit van de training van piloten van gevechtsvliegtuigen.
    Tijdens de training neemt een piloot plaats in een speciale centrifuge, waarbij hij of zij aan hoge G-krachten wordt blootgesteld.
    Hierbij wordt geoefend met de Anti-G Straining Maneuver (AGSM), een speciale ademhalingstechniek die helpt om langer weerstand te kunnen bieden aan de hoge G-krachten.

    De focus van het onderzoek ligt op het bepalen van de maximale G-belasting totdat een piloot verstoringen ervaart in het gezichtsveld, zoals tunnelvisie of tijdelijke blackouts.
    Hierbij zijn voor twee populaties steekproeven genomen, namelijk voor de populatie F-16 vliegers (populatie $A$) en voor de populatie F-35 vliegers (populatie $B$).

    \begin{center}
            \begin{tabular}{c|cccccccccc}
                \toprule
                    \textbf{Maximale G-belasting populatie $A$ (in G)} & $7,4$ & $8,1$ & $8,3$ & $7,8$ & $7,9$ & $8,0$ & $7,7$ & $8,2$ & $7,6$ & $8,0$ \\
                    \textbf{Maximale G-belasting populatie $B$ (in G)} & $8,3$ & $8,7$ & $8,9$ & $8,2$ & $8,5$ & $8,4$ & $8,8$ & $8,6$ & $8,3$ & $8,5$ \\
                \bottomrule
            \end{tabular}
        \end{center}

    Het doel van het onderzoek is om te toetsen of de gemiddelde maximale G-belasting significant hoger is voor F-35 vliegers. 
    Voor beide populaties wordt aangenomen dat de maximale G-belasting van een willekeurige vlieger normaal verdeeld is met onbekende verwachtingswaarde en standaardafwijking.
}
    \subquestion{10}{
        Bepaal voor beide populaties de steekproefgemiddeldes en de steekproefvarianties.
    }
    \solution{
        We berekenen het steekproefgemiddelde $\overline{x}$ voor de steekproef uit populatie $A$ als volgt.
        De steekproefgrootte is gelijk aan $n = 10$.
        \begin{align*}
            \overline{x} = \frac{x_1+x_2+\ldots+x_n}{n} = \frac{ 7,4 + 8,1 + \ldots + 8,0 }{ 10 } \approx 7,9 \rubric{2}
        \end{align*}
    
        We berekenen de steekproefvariantie $s_A^2$ als volgt:
        \begin{align*}
            s_A^2   &= \frac{ (x_1 - \overline{x})^2 + (x_2 - \overline{x})^2 + \ldots + (x_n - \overline{x})^2 }{ n - 1 } \\
                    &= \frac{ (7,4 - 7,9)^2 + (8,1 - 7,9)^2 + \ldots + (8,0 - 7,9)^2 }{ 10 - 1 } \\
                    &\approx 0,0778. \rubric{3}
        \end{align*}
        Op eenzelfde manier berekenen we het steekproefgemiddelde $\overline{y}$ voor de steekproef uit populatie $B$ en de steekproefvariantie $s_B^2$.
        De steekproefgrootte is gelijk aan $m = 10$.
        \begin{align*}
            \overline{y} = \frac{y_1+y_2+\ldots+y_m}{m} = \frac{ 8,3 + 8,7 + \ldots + 8,5 }{ 10 } \approx 8,52. \rubric{2}
        \end{align*}
    
        We berekenen de steekproefvariantie $s^2$ als volgt:
        \begin{align*}
            s^2 &= \frac{ (y_1 - \overline{y})^2 + (y_2 - \overline{y})^2 + \ldots + (y_m - \overline{y})^2 }{ m - 1 } \\
                &= \frac{ (8,3 - 8,52)^2 + (8,7 - 8,52)^2 + \ldots + (8,5 - 8,52)^2 }{ 10 - 1 } \\
                &\approx 0,0529. \rubric{3}
        \end{align*}
    }

    \subquestion{8}{
        Voer een $F$-toets uit en bepaal met behulp van de $p$-waarde of de varianties in de maximale G-belasting gelijk is voor beide populaties. Kies voor het significantieniveau $\alpha = 0,05$.
    }
    \solution{
        We noteren de populaties F-16 en F-35 vliegers respectievelijk met $A$ en $B$.
        In de vraag staat dat we aan mogen nemen dat $X_A \sim N(\mu_A=?; \sigma_A=?)$ en $X_B \sim N(\mu_B=?; \sigma_A=?)$.

        We toetsen op gelijke varianties, oftewel
        \begin{align*}
            H_0: \quad \sigma_A^2 = \sigma_B^2 \\
            H_1: \quad \sigma_A^2 \neq \sigma_B^2 \rubric{2}
        \end{align*}

        Verder is gegeven dat we mogen werken met een significantieniveau $\alpha=0,05$, en data is al verzameld voor beide populaties.
        De toetsingsgrootheid voor een $F$-toets is gelijk aan
        \[
            F = \frac{S_A^2}{S_B^2}, 
        \]
        en volgt een $F(n-1, m-1)$-verdeling, oftewel een $F(9,9)$-verdeling. \rubric{1}
        De geobserveerde toetsingsgrootheid is gelijk aan
        \[
            f = \frac{s_A^2}{s_B^2} = \frac{0,0778}{0,0529} \approx 1,4707. \rubric{1}
        \]
        
        De $p$-waarde (rechteroverschrijdingskans) van deze toetsingsgrootheid $f$ is gelijk aan
        \begin{align*}
            p = P(F \ge f) = \text{Fcdf}{\text{lower}=1,4707; \text{upper}=10^{99}; \text{df1}=9; \text{df2}=9} \approx 0,2874.\rubric{2}
        \end{align*}
        Aangezien de $p$-waarde groter is dan $\alpha/2$ (tweezijdige toets!), wordt de nulhypothese $H_0$ aangenomen.\rubric{1}
        Er is onvoldoende bewijs om op basis van deze steekproef de aanname van gelijke varianties te verwerpen. \rubric{1}
    }

    \subquestion{10}{
        Bepaal met behulp van een onafhankelijke $t$-toets of gemiddeld de maximale G-belasting significant hoger is voor F-35 vliegers.
        Kies opnieuw voor het significantieniveau $\alpha = 0,05$.
    }
    \solution{
        We toetsen of het gemiddelde $\mu_B$ voor F-35 vliegers significant hoger is dan het gemiddelde $\mu_A$ voor F-16 vliegers.
        De hypotheses kunnen we daarom als volgt defini\"eren:
        \begin{align*}
            H_0: \quad \mu_A \ge \mu_B \text{ (niet significant hoger) } \\
            H_1: \quad \mu_A < \mu_B \text{ (wel significant hoger) } \rubric{2}
        \end{align*}

        Verder is gegeven dat we mogen werken met een significantieniveau $\alpha=0,05$, en data is al verzameld voor beide populaties.

        Op basis van ons antwoord bij vraag (b) mogen we aannemen dat $\sigma = \sigma_A = \sigma_B$.
        Dat betekent dat we met de pooled variance mogen werken als schatting voor de gemeenschappelijke onbekende $\sigma$.
        
        \begin{align*}
            s_P^2 = \frac{(n-1)\cdot s_A^2 + (m-1) \cdot s_B^2}{n-1+m-1} = \frac{9\cdot 0,0778 + 9\cdot 0,0529}{18} \approx 0,0653. \rubric{3}
        \end{align*}
        
        De toetsingsgrootheid van de bijbehorende $t$-toets is gelijk aan
        \[
            t = \frac{(\overline{x}-\overline{y}) - (\mu_A - \mu_B)}{\sqrt{\frac{s_P^2}{n} + \frac{s_P^2}{m}}} \approx -5,4239, 
        \]
        en komt uit een $t$-verdeling met $\text{df}=18$ vrijheidsgraden. \rubric{2}
        Omdat we linkszijdig toetsen, is het kritieke gebied van de vorm $(-\infty; g]$.
        Deze kritieke grens kunnen we met de $t$-verdeling bepalen, namelijk
        \[
            g = \invt(\text{opp} = \alpha; \text{df}=n+m-2) = \invt(\text{opp}=0,05; \text{df}=18) \approx -1,7341 \rubric{1}
        \]
        
        De toetsingsgrootheid $t$ ligt in het kritieke gebied, dus $H_0$ moet worden verworpen.
        Er is op basis van deze steekproeven voldoende reden om aan te nemen dat de maximale G-belasting van F-35 vliegers significant hoger is dan die van F-16 vliegers. \rubric{2}
    }    

\end{question}