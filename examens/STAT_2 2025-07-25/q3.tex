\begin{question}{20}{
    Bij de sollicitatiegesprekken voor cadetten en adelborsten wordt door de aanstellingscommissie aan de aspirant-officier gevraagd naar zijn of haar primaire motivatie om officier te willen worden.
    De antwoorden van de aspiranten zijn ruwweg in te delen in vier categorie\"en: leiderschap, avontuur, doorgroeimogelijkheden en dienen aan het vaderland.
    
    Naar aanleiding van de sollicitatiegesprekken kunnen de antwoorden als volgt worden samengevat:

    \begin{center}
        \renewcommand{\arraystretch}{1.25}
        \begin{tabular}{c|cc|c}
            \toprule
                & Cadetten & Adelborsten & \\
            \midrule
                Leiderschap & $52$ & $22$ & $74$ \\
                Avontuur & $23$ & $9$ & $32$ \\
                Doorgroeimogelijkheden & $17$ & $11$ & $28$ \\
                Vaderland dienen & $38$ & $28$ & $56$ \\
            \midrule
                Totaal      & $130$ & $70$ & $200$ \\
            \bottomrule
        \end{tabular}
    \end{center}
    C-NLDA wil inzicht krijgen in of de primaire motivatie van cadetten significant afwijkt van die van adelborsten.
}

    \subquestion{4}{
        Welk type hypothesetoets dienen we hiervoor uit te voeren? 
        Formuleer de nulhypothese $H_0$ en de alternatieve hypothese $H_1$ van de bijbehorende hypothesetoets.
    }
    \solution{
        Omdat we willen weten of de primaire motivatie van cadetten significant afwijkt van die van de adelborsten, dienen we een $\chi^2$-toets (chikwadraat) voor onafhankelijkheid uit te voeren. \rubric{2}
        Dit kan ook omdat we te maken hebben met twee categorische variabelen!
        
        De bijbehorende nul- en alternatieve hypothese zijn in dit geval gelijk aan
        \begin{align*}
            H_0&: \text{ de verdeling van de primaire motivaties van cadetten } \\
               & \qquad \qquad \text{is onafhankelijk van het type aspirant-officieren.} \rubric{1} \\
            H_1&: \text{ de verdeling van de primaire motivaties van cadetten } \\
               & \qquad \qquad \text{is WEL afhankelijk van het type aspirant-officieren.} \rubric{1}
        \end{align*}
    }

    \subquestion{11}{Bepaal de $p$-waarde van de desbetreffende hypothesetoets.}
    \solution{
        Om een chi-kwadraattoets voor onafhankelijkheid uit te voeren, moeten we eerst de verwachte frequenties uitrekenen. Onder de aanname dat $H_0$ waar is, is deze voor elk van de negen cellen te berekenen als $\frac{\text{rijtotaal} \cdot \text{kolomtotaal}}{\text{totaal}}$.
        De verwachte frequenties zijn te berekenen met behulp van de formule:
        \[
            E_{ij} = \frac{\text{rijtotaal}_{i} \cdot \text{kolomtotaal}_{j}}{\text{totaal}}
        \]

        \begin{center}
            \begin{minipage}{0.45\textwidth}
                \begin{table}[H]
                    \renewcommand{\arraystretch}{1.25}
                    \centering
                    {\bfseries Observed} \\[1ex]
                        \begin{tabular}{c|cc|c}
                            \toprule
                                & Cadetten & Adelborsten & Totaal \\
                            \midrule
                                Leiderschap & $52$ & $22$ & $74$ \\
                                Avontuur & $23$ & $9$ & $32$ \\
                                Doorgroei & $17$ & $11$ & $28$ \\
                                Dienen & $38$ & $28$ & $56$ \\
                            \midrule
                                Totaal      & $130$ & $70$ & $200$ \\
                            \bottomrule
                        \end{tabular}
                    
                \end{table}
            \end{minipage}
            \hfill
            \begin{minipage}{0.45\textwidth}    
                \begin{table}[H]
                    \renewcommand{\arraystretch}{1.25}
                    \centering
                    {\bfseries Expected} \\[1ex]
                        \begin{tabular}{cc|c}
                        \toprule
                                Cadetten & Adelborsten & Totaal \\
                            \midrule
                                $\frac{130 \cdot 74}{200} = 48,1$ & $\frac{70 \cdot 74}{200} = 25,9$ & $74$ \\
                                $\frac{130 \cdot 32}{200} = 20,8$ & $\frac{70 \cdot 32}{200} = 11,2$ & $32$ \\
                                $\frac{130 \cdot 28}{200} = 18,2$ & $\frac{70 \cdot 28}{200} = 9,8$ & $28$ \\
                                $\frac{130 \cdot 56}{200} = 42,9$ & $\frac{70 \cdot 56}{200} = 23,1$ & $56$ \\
                            \midrule
                                $130$ & $70$ & $200$ \\
                            \bottomrule
                        \end{tabular}
                    
                \end{table}
            \end{minipage}\rubric{4}
        \end{center} 
        
        We berekenen de toetsingsgrootheid $\chi^2$ als volgt, waarbij $O_{ij} / E_{ij}$ de geobserveerde / verwachte frequentie is in rij $i$, kolom $j$:        
        \begin{align*}
            \chi^2  &= \frac{(O_{11} - E_{11})^2}{E_{11}} + \frac{(O_{12} - E_{12})^2}{E_{12}} + \ldots + \frac{(O_{42} - E_{42})^2}{E_{42}} \\
                    &= \frac{(52 - 48,1)^2}{48,1} + \frac{(22-25,9)^2}{25,9} + \ldots + \frac{(28 - 23,1)^2}{23,1} \\
                    &\approx 3,3934.\rubric{3}
        \end{align*}

        De bijbehorende $p$-waarde is de rechteroverschrijdingskans $P(X > \chi^2)$, waarbij $X^2$ de toetsingsgrootheid van de chikwadraattoets voor onafhankelijkheid. \rubric{1}
        Deze toetsingsgrootheid heeft $\text{df} = (\text{\#rijen}-1)\cdot(\text{\#kolommen}-1)=(4-1)\cdot(2-1) = 3$ vrijheidsgraden.\rubric{1}
        \begin{align*}
            p = P(X > \chi^2 \approx 3,3934) &= \chi^2\text{cdf}(\text{lower}=3,3934; \text{upper}=10^{99}; \text{df}=3) \\
                                        &\approx 0,3348. \rubric{2}
        \end{align*}
    }

    \subquestion{5}{Geef de conclusie van de hypothesetoets (op basis van een significantieniveau $\alpha = 0,05$) met behulp van de berekende $p$-waarde.}
    \solution{
        Uit de hypothesetoets volgt een redelijk grote rechteroverschrijdingskans $p \approx 0,3348$.
        Omdat $p > \alpha = 0,05$, geldt dat $H_0$ wordt aangenomen. \rubric{2}
        We kunnen dus de conclusie trekken dat de verdeling van primaire motivaties om officier te worden niet significant verschilt tussen cadetten en adelborsten. \rubric{1}
        
        Als we kijken naar de tabel van geobserveerde en verwachte frequenties, dan zien we dat deze redelijk dichtbij elkaar in de buurt liggen.
        Er zijn geen grote uitschieters te bekennen in de verwachte frequenties ten opzichte van de daadwerkelijk geobserveerde data.\rubric{2}
    }
\end{question}