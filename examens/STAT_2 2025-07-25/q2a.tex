\begin{question}{30}{
    De afdeling Dienstencentrum Human Resources (DCHR) van Defensie wil nagaan of het wervingsproces effectief en vlot verloopt.
    De doelstelling is dat sollicitanten binnen \textbf{45 dagen} (exclusief veiligheidsonderzoek) na hun schriftelijke sollicitatie een formeel aanstellingsvoorstel ontvangen.
    
    Om de huidige omstandigheden van werving en selectie te onderzoeken, zijn doorlooptijden van dertien recent aangestelde burgermedewerkers verzameld (in dagen):
    \begin{align*}
        39, 42, 44, 48, 47, 43, 31, 46, 40, 42, 43, 49, 44
    \end{align*}
    Neem aan dat de doorlooptijden normaal verdeeld zijn, elke procedure met dezelfde verwachtingswaarde en standaardafwijking.
}
    \subquestion{7}{
        Bereken het steekproefgemiddelde en de steekproefstandaardafwijking van de doorlooptijden van sollicitatieprocedures.
    }
    \solution{
        We berekenen het steekproefgemiddelde $\overline{x}$ als volgt:
        \begin{align*}
            \overline{x} = \frac{x_1+x_2+\ldots+x_n}{n} = \frac{ 39 + 42 + \ldots + 44 }{ 13 } \approx 42,9231.\rubric{3}
        \end{align*}
    

        We berekenen de steekproefvariantie $s^2$ als volgt:
        \begin{align*}
            s^2 &= \frac{ (x_1 - \overline{x})^2 + (x_2 - \overline{x})^2 + \ldots + (x_n - \overline{x})^2 }{ n - 1 } \\
                &= \frac{ (39 - 42,9231)^2 + (42 - 42,9231)^2 + \ldots + (44 - 42,9231)^2 }{ 13 - 1 } \\
                &\approx 21,5769. \rubric{3}
        \end{align*}
    

        We berekenen de steekproefstandaardafwijking $s$ door te wortel van de steekproefvariantie $s^2$ te nemen:
        \begin{align*}
            s = \sqrt{ s^2 } = \sqrt{ 21,5769 } \approx 4,6451. \rubric{1}
        \end{align*}
    }

    \subquestion{6}{
        Bereken een \SI{94}{\percent}-betrouwbaarheidsinterval voor de gemiddelde doorlooptijd $\mu$ van sollicitatieprocessen, op grond van bovengenoemde steekproef,
        zonder daarbij gebruik te maken van de optie TESTS/Interval van de grafische rekenmachine.
        Rond de grenzen van het interval af op gehele dagen, dusdanig dat het betrouwbaarheidsniveau gewaarborgd blijft.
    }
    \solution{
        Laat $X \sim N(\mu=?; \sigma=?)$ de doorlooptijd zijn van een willekeurige sollicitatieprocedure.
        Volgens de centrale limietstelling is de gemiddelde doorlooptijd $\overline{X} = \frac{X_1 + X_2 + \ldots + X_{13}}{13}$ van 13 procedures ook normaal verdeeld, met onbekende verwachtingswaarde $\mu$ en standaardafwijking $\frac{\sigma}{\sqrt{13}}$. \rubric{1}
        Omdat naast de verwachtingswaarde $\mu$ ook de standaardafwijking $\sigma$ onbekend is (en bovendien de steekproefgrootte $n = 13 < 30$), moet de $t$-verdeling worden gebruikt.\rubric{1}
        Omdat $\sigma$ onbekend is en de steekproefgrootte $n = 13 < 30$ is, moeten we gebruik maken van de $t$-verdeling.
        De $t$-waarde die hoort bij \SI{94}{\percent} betrouwbaarheid, oftewel $\alpha = 0.06$, is (in het geval van tweezijdige intervallen) gelijk aan
        \[
            t = \invt(\text{opp}=1-\frac{\alpha}{2}; \text{df}=n-1) = \invt(\text{opp}=0,97; \text{df}=12) \approx 2,0764.\rubric{1}
        \]
        Het \SI{94}{\percent}-betrouwbaarheidsinterval wordt dan gegeven door
        \begin{align*}
            &[\overline{x} - t \cdot \frac{ s }{ \sqrt{ n } }; \overline{x} + t \cdot \frac{ s }{ \sqrt{ n } }] \\
            &= [42,9231 - 2,0764 \cdot \frac{ 4,6451 }{ \sqrt{ 13 } }; 42,9231 + 2,0764 \cdot \frac{ 4,6451 }{ \sqrt{ 13 } }] \\
            &= [40,2480; 45,5982].\rubric{2}
        \end{align*}
        Om de betrouwbaarheid te waarborgen, mag het interval niet kleiner worden, dus naar buiten afronden: $[40; 46]$. 
        Met \SI{94}{\percent} betrouwbaarheid ligt de gemiddelde doorlooptijd van een sollicitatieprocedure tussen de $40$ en $46$ dagen.\rubric{1}
    }

    \subquestion{9}{
        Toets op basis van de gegeven steekproef of de gemiddelde doorlooptijd van sollicitatieprocedures voldoet aan de bovengenoemde doelstelling van maximaal 45 dagen.
        Geef je conclusie op basis van de $p$-waarde.
        Kies in dit geval als onbetrouwbaarheid $\alpha = 0,05$.
        Leg in simpele bewoordingen uit wat de uitslag van deze toets betekent voor de doorlooptijd van sollicitatieprocedures binnen Defensie.
    }
    \solution{
        In deze hypothesetoets hebben we te maken met een $t$-toets met nul- en alternatieve hypothese als volgt: \rubric{1}
        \begin{align*}
            H_{0}: \mu &\le 45 \tag{gemiddelde voldoet aan de doelstelling} \\ \rubric{1}
            H_{1}: \mu &> 45 \tag{gemiddelde voldoet NIET aan de doelstelling} \rubric{1}
        \end{align*}

        In dit geval werken we met een rechtszijdige toets (alternatieve hypothese heeft vorm $>$), dus het kritieke gebied is van de vorm $[g, \infty)$.\rubric{1}
        We willen de (kleinste) grens $g$ zoeken zodanig dat de kans op een type-I fout ($H_0$ verwerpen terwijl die waar is) kleiner is dan $\alpha=0,05$.

        Onder de nulhypothese is de gemiddelde doorlooptijd $\overline{X}$ van 13 willekeurige sollicitatieprocedures normaal verdeeld met verwachtingswaarde $\mu = 45$ en standaardafwijking $\frac{\sigma}{\sqrt{n}}$.\rubric{1}
        Omdat $\sigma$ onbekend is en $n = 18 < 30$, gebruiken we opnieuw de $t$-verdeling en $s = 4,6451$. \rubric{1}
        Er geldt dus, omdat we eenzijdig toetsen, dat 
        \[
            t = \invt(opp=1-\alpha; df=n-1) = \invt(opp=0,95; df=12) \approx 2,1788. \rubric{1}
        \]
        (merk op dat de $t$-waarde anders is, omdat we met een ander betrouwbaarheidsniveau werken!).
        
        De bijbehorende grenswaarde vinden we nu met
        \[
            g = \mu + t \cdot \frac{s}{\sqrt{18}} = 45 + 2,1788 \cdot \frac{4,6451}{\sqrt{13}} \approx 47,8070. \rubric{1}
        \]

        Het steekproefgemiddelde $\overline{x} \approx 45,9231$ is kleiner dan deze ondergrens $g \approx 47,8070$ van het kritieke gebied, dus $H_0$ wordt niet verworpen. 
        Dit betekent dat met deze toets niet kan worden aangetoond dat de gemiddelde doorlooptijd buiten de gestelde doelstelling valt. \rubric{2}
    } 
    
    \subquestion{8}{
        Het Dienstencentrum Human Resources breidt haar onderzoek uit.
        In een grootschalige onderzoek met $386$ burgermedewerkers, blijkt dat $102$ burgermedewerkers binnen 45 dagen een aanstellingsvoorstel heeft ontvangen.
        Bereken met de Clopper-Pearson methode een \SI{95}{\percent}-betrouwbaarheidsinterval voor het percentage burgermedewerkers dat binnen 45 dagen een aanstellingsvoorstel heeft ontvangen.
    }
    \solution{
       Laat $X$ het aantal burgermedewerkers zijn dat binnen $45$ dagen een aanstellingsvoorstel heeft ontvangen. 
        Aangezien de steekproefomvang gelijk is aan $386$ kiesgerechtigde Nederlanders, is $X$ binomiaal verdeeld met $n=386$ en nog onbekende succeskans $p$. \rubric{1}
    
        We bepalen een \SI{95}{\percent}-betrouwbaarheidsinterval voor $p$ met behulp van de Clopper-Pearson methode.
        Omdat de gewenste betrouwbaarheid \SI{95}{\percent} is, geldt dat $\alpha = 0,05$.

        De Clopper-Pearson methode werkt als volgt:
        \begin{enumerate}
            \item Bepaal de succeskans $p_1$ waarvoor geldt dat de linkeroverschrijdingskans van de uitkomst $k=102$ gelijk is aan $\alpha/2$, oftewel $P(X \le 102) = \alpha/2 = 0,025$.
            Voer hiervoor in het functiescherm van de grafische rekenmachine:
            \begin{align*}
                y_1 &= \binomcdf(n=386; p_1=X; k=102) \\
                y_2 &= 0,025 \rubric{2}
            \end{align*}
            De solver optie geeft een waarde van $p_1 \approx 0,3112$.\rubric{1}
            \item Bepaal de succeskans $p_2$ waarvoor geldt dat de rechteroverschrijdingskans van de uitkomst $k=102$ gelijk is aan $\alpha/2$, oftewel $P(X \ge 102) = 1 - P(X \le 101) = \alpha/2 = 0,025$.
            Voer hiervoor in het functiescherm van de grafische rekenmachine:
            \begin{align*}
                y_1 &= 1 - \binomcdf(n=386; p_1=X; k=101) \\
                y_2 &= 0,025 \rubric{2}
            \end{align*}
            De solver optie geeft een waarde van $p_2 \approx 0,2209$.\rubric{1}
        \end{enumerate}
        We vinden het Clopper-Pearson interval door de twee gevonden waarden als grenzen te nemen, oftewel met \SI{95}{\percent}-betrouwbaarheid ligt het percentage burgermedewerkers dat binnen $45$ dagen een aanstellingsvoorstel heeft ontvangen tussen ongeveer \SI{22}{\percent} en \SI{31}{\percent}. \rubric{1}
    }  

\end{question}